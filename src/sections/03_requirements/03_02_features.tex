\subsection{Features}

Die in Kapitel 3.1 beschriebenen User Stories werden in drei Features gruppiert.
Diese drei Features bilden Grundlage für Aufteilung der nachfolgenden Kapitel Konzept und Technologie Evaluation.
Es wurde die folgenden drei Features definiert:

\begin{table}[h]
    \centering
    \begin{tabular}{|l|p{15cm}|}
        \hline
        \textbf{Id} & \textbf{Feature}                        \\
        \hline
        F01         & Native iOS Applikation                 \\
        \hline
        F02         & Sprachsynthese                          \\
        \hline
        F03         & Gegensprechanlage                       \\
        \hline
    \end{tabular}\label{tab:features}
\end{table}

Das Feature F01 ''Native iOS Applikation'' bildet die Grundlage für die Neuentwicklung der mobilen Applikation.
Es beinhaltet den Aufbau der neuen Applikation und die Migration bestehender Funktionalität aus dem Vorgängerprojekt.
Das Feature F02 ''Sprachsynthese'' baut auf F01 auf und fügt die Sprachsynthese für empfangene Benachrichtigungen hinzu.
Mit dem Feature F03 ''Gegensprechanlage'' wird schliesslich das Kernstück der neuen Funktionalität umgesetzt.
Dieses Feature beinhaltet Integration von Peer-To-Peer Sprachverbindungen in das System.
Nach der Umsetzung dieses Features kann das Praxisrufsystem als Gegensprechanlage verwendet werden.
Die korrekte Umsetzung aller Anforderungen wird durch Funktionstests sichergestellt.
Dazu werden Testszenarien definiert, welche Ausgangslage, Testschritte und die erwarteten Resultate definieren.
Die Szenarien sind in Anhang C aufgeführt.
Die Resultate der letzten Ausführung der Testszenarien sind im Kapitel 8.2 festgehalten.

\clearpage
