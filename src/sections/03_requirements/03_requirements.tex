\section{Anforderungen}\label{sec:anforderungen}

Dieses Kapitel beschreibt die Anforderungen an das Praxisrufsystem, welche zu Beginn des Projektes erarbeitet werden.
Die Anforderungen werden aus fachlicher der Stakeholder festgehalten.
Dazu werden zunächst Stakeholdergruppen identifiziert und anschliessend pro Gruppe User Stories definiert.
Jede User Story beschreibt eine Anfroderung welche eine Stakeholdergruppe an das System hat.
Anhand der User Stories werden anschliessen Features definiert, welche User Stories unter einem gemeinsamen Titel zusammenfassen.

\subsection{User Stories}

Im folgenden Kapitel werden die Anforderungen aus Sicht der Stakeholder dokumentiert.
Im Vorgängerprojekt wurden drei Stakeholdergruppen, welche für dieses Projekt übernommen werden.
Die Gruppe Praxismitarbeitende verwendet das Praxisrufsystem, um in der Praxis zu kommunizieren.
Neben Praxismitarbeitenden, arbeitet auch die Rolle des Praxisverantwortlichen mit dem Praxisrufsystem.
Praxisverantwortliche sind dafür verantwortlich, die Konfiguration des Praxisrufsystems vorzunehmen.
Als dritte Stakeholdergruppe hat der Auftraggeber ein Interesse daran, dass gewisse Rahmenbedingungen gesetzt und eingehalten werden.\cite{ip5}
Im Folgenden werden die Anforderungen pro Stakeholdergruppe tabellarisch aufgelistet.
\\

\subsubsection{Praxismitarbeitende}

\begin{table}[h]
    \centering
    \begin{tabular}{|l|p{15cm}|}
        \hline
        \textbf{Id} & \textbf{Anforderung}                                                                                                                                                                                      \\
        \hline
        U01         & Als Praxismitarbeiter/in möchte ich alle Funktionen aus der existierenden Applikation weiterhin verwenden können, damit mir diese weiterhin die Arbeit erleichtern. \footnote[2]{}                        \\
        \hline

        U02         & Als Praxismitarbeiter/in möchte ich, dass wichtige eingehende Benachrichtigungen vorgelesen werden, damit den Inhalt der Benachrichtigung kenne, ohne meine Aufmerksamkeit auf den Bildschirm zu richten. \\
        \hline
        U03         & Als Praxismitarbeiter/in möchte ich, das Vorlesen von Benachrichtigungen deaktivieren können, damit ich bei der Arbeit nicht unnötig gestört werde.                                                       \\
        \hline
        U04         & Als Praxismitarbeiter/in möchte ich, per Button eine Sprachverbindung zu einem anderen Praxiszimmer aufbauen können, damit ich mich mit einer anderen Person absprechen kann.                             \\
        \hline
        U05 & Als Praxismitarbeiter/in möchte ich, per Button eine Sprachverbindung zu mehreren anderen Praxiszimmern aufbauen können damit, ich mich mit mehreren anderen Personen absprechen kann. \\
        \hline
        U06 & Als Praxismitarbeiter/in möchte ich über geöffnete Sprachverbindungen in Echtzeit kommunizieren können damit es die Funktion einer Gegensprechanlage wirklich erfüllt.                 \\
        \hline
        U07 & Als Praxismitarbeiter/in möchte ich nur Buttons für Sprachverbindungen sehen, die für mich relevant sind.                                                                              \\
        \hline
        U08 & Als Praxismitarbeiter/in möchte ich benachrichtigt werden, wenn ein anderes Zimmer eine Sprachverbindung öffnet, damit ich auf die Anfrage Antworten kann.                             \\
        \hline
        U09 & Als Praxismitarbeiter/in möchte ich vergangene und verpasste Sprachverbindungen nachvollziehen können, damit ich mich zurückmelden kann.                                               \\
        \hline
        U10 & Als Praxismitarbeiter/in möchte ich, dass eingehende Sprachverbindungen aus anderen Praxiszimmern automatisch geöffnet werden damit ich meine Hände für besseres brauchen kann.        \\
        \hline
        U11 & Als Praxismitarbeiter/in möchte ich, direkte Sprachverbindungen aus anderen Praxiszimmern trennen können damit ich ein Gespräch beenden kann.                                          \\
        \hline
        U12 & Als Praxismitarbeiter/in möchte ich, aus Sprachverbindungen zu mehreren Praxiszimmern (Gruppenunterhaltungen) austreten können, damit ich nicht unnötig bei der Arbeit gestört werde.  \\
        \hline
    \end{tabular}\label{tab:userstories0}
\end{table}

\clearpage

\subsubsection{Praxisadministrator}

\begin{table}[h]
    \centering
    \begin{tabular}{|l|p{15cm}|}
        \hline
        \textbf{Id} & \textbf{Anforderung}                                                                                                                                                                                                    \\
        \hline
        U13         & Als Praxisadministrator möchte ich konfigurieren können, welche Benachrichtigungen dem Praxismitarbeitenden vorgelesen werden, damit nur relevante Benachrichtigungen vorgelesen werden.                                \\
        \hline
        U14         & Als Praxisadministrator möchte ich konfigurieren können, aus welchen Zimmern Sprachverbindungen zu welchen anderen Zimmern aufgebaut werden können, damit die Mitarbeitendend das System effizient bedienen können.     \\
        \hline
        U15         & Als Praxisadministrator möchte ich Benachrichtigungen, Clients und Benutzer wie zuvor konfigurieren können, damit ich das System weiterhin auf meine Praxis zuschneiden und bestehende Konfigurationen übernehmen kann. \\
        \hline
    \end{tabular}\label{tab:userstories2}
\end{table}

\subsubsection{Auftraggeber}

\begin{table}[h]
    \centering
    \begin{tabular}{|l|p{15cm}|}
        \hline
        \textbf{Id} & \textbf{Anforderung}                                                                                                                                                                          \\
        \hline
        U16         & Als Auftraggeber möchte ich die bestehende Betriebsinfrastruktur übernehmen, um von der bereits geleisteten Arbeit profitieren zu können.                                                     \\
        \hline
        U17         & Als Auftraggeber möchte ich, dass die bestehende Komponenten des Systems wo immer möglich weiter verwendet werden, um von der bereits geleisteten Arbeit profitieren zu können.               \\
        \hline
        U18         & Als Auftraggeber möchte ich, der bestehende Mobile Client als native iOS Applikation ungeschrieben wird, um Wartbarkeit und Gerätekompatibilität zu gewährleisten.                            \\
        \hline
        U19         & Als Auftraggeber möchte ich, dass wo möglich der Betrieb von Serverseitigen Dienstleistungen über AWS betrieben wird, damit ich von bestehender Infrastruktur und Erfahrung profitieren kann. \\
        \hline
    \end{tabular}\label{tab:userstories3}
\end{table}

\subsection{Features}

Aus den User Stories ergeben sich drei Features, welche mit dem Projekt "P2P Sprachübertragung in Praxisrufsystemen" umgesetzt werden müssen.

\begin{table}[h]
    \centering
    \begin{tabular}{|l|p{15cm}|}
        \hline
        \textbf{Id} & \textbf{Feature}                        \\
        \hline
        F01         & Migration des bestehenden Mobile Client \\
        \hline
        F02         & Sprachsynthese                          \\
        \hline
        F03         & Gegensprechanlage                       \\
        \hline
    \end{tabular}\label{tab:features}
\end{table}

Diese Features dienen zur Aufteilung der Umsetzungsphase.
F01 Migration des bestehenden Mobile Client bildet die Grundalge für die anderen Features und wird deshalb als erstes Umgesetzt.
F02 Sprachsynthese ist eng mit der bestehenden Funktionalität verbunden.
Es soll deshalb möglichst zeitnah nach F01 umgesetzt werden.
F03 Gegensprechanlage ist das Kernstück der neuen Funktionalität.
Dieses Feature kann erst umgesetzt werden, wenn die Grundlagenarbeit von F01 umgesetzt ist.


\clearpage
