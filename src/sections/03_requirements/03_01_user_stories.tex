\subsection{User Stories}

Im folgenden Kapitel werden die Anforderungen aus Sicht der Stakeholder dokumentiert.
Im Vorgänger-projekt wurden die drei Stakeholdergruppen Praxismitarbeitende, Praxisverantwortliche und Auftraggeber identifiziert.
Die Gruppe Praxismitarbeitende verwendet das Praxisrufsystem, um in der Praxis zu kommunizieren.
Praxisverantwortliche sind dafür zuständig, die Konfiguration des Praxisrufsystems zu erstellen und zu verwalten.
Als dritte Stakeholdergruppe hat der Auftraggeber ein Interesse daran, dass gewisse Rahmenbedingungen gesetzt und eingehalten werden~\cite{ip5}.
Die Stakeholdergruppen des Vorgängerprojektes werden für dieses Projekt übernommen.
Im Folgenden werden die erarbeiteten Anforderungen pro Stakeholdergruppe tabellarisch aufgelistet.

\subsubsection{Praxismitarbeitende}

\begin{table}[h]
    \centering
    \begin{tabular}{|l|p{15cm}|}
        \hline
        \textbf{Id} & \textbf{Anforderung}                                                                                                                                                                                      \\
        \hline
        U01 & Als Praxismitarbeiter/in möchte ich alle bestehenden Funktionen von Praxisruf weiterhin verwenden können, damit mir diese weiterhin die Arbeit erleichtern. \\
        \hline
        U02 & Als Praxismitarbeiter/in möchte ich, dass eingehende Benachrichtigungen vorgelesen werden, damit ich den Inhalt der Benachrichtigung kenne, ohne meine Aufmerksamkeit auf den Bildschirm zu richten. \\
        \hline
        U03 & Als Praxismitarbeiter/in möchte ich, das Vorlesen von Benachrichtigungen deaktivieren können, damit ich bei der Arbeit nicht unnötig gestört werde.                                                       \\
        \hline
        U04 & Als Praxismitarbeiter/in möchte ich, per Button eine Sprachverbindung zu einem anderen Praxiszimmer aufbauen können, damit ich mich mit einer anderen Person absprechen kann.                             \\
        \hline
        U05 & Als Praxismitarbeiter/in möchte ich, per Button eine Sprachverbindung zu mehreren anderen Praxiszimmern aufbauen können damit, ich mich mit mehreren anderen Personen absprechen kann. \\
        \hline
        U06 & Als Praxismitarbeiter/in möchte ich über geöffnete Sprachverbindungen in Echtzeit kommunizieren können, damit Praxisruf die Funktion einer Gegensprechanlage erfüllt.                 \\
        \hline
        U07 & Als Praxismitarbeiter/in möchte ich nur Buttons für Sprachverbindungen sehen, die für mich relevant sind, damit ich das System einfacher bedienen kann.                                                                          \\
        \hline
        U08 & Als Praxismitarbeiter/in möchte ich benachrichtigt werden, wenn ein anderes Zimmer eine Sprachverbindung öffnet, damit ich auf die Anfrage antworten kann.                             \\
        \hline
        U09 & Als Praxismitarbeiter/in möchte ich den Verlauf empfangener und verpasster Sprachverbindungen nachvollziehen können, damit ich mich zurückmelden kann.                                               \\
        \hline
        U10 & Als Praxismitarbeiter/in möchte ich, dass eingehende Sprachverbindungen aus anderen Praxiszimmern automatisch geöffnet werden, damit das Gespräch möglichst schnell beginnen kann.        \\
        \hline
        U11 & Als Praxismitarbeiter/in möchte ich, direkte Sprachverbindungen aus anderen Praxiszimmern trennen können, damit ich ein Gespräch beenden kann.                                          \\
        \hline
        U12 & Als Praxismitarbeiter/in möchte ich, aus Sprachverbindungen zu mehreren Praxiszimmern (Gruppenunterhaltungen) austreten können, damit ich nicht unnötig bei der Arbeit gestört werde.  \\
        \hline
    \end{tabular}\label{tab:userstories0}
\end{table}

\clearpage

\subsubsection{Praxisverantwortliche}

\begin{table}[h]
    \centering
    \begin{tabular}{|l|p{15cm}|}
        \hline
        \textbf{Id} & \textbf{Anforderung}                                                                                                                                                                                                    \\
        \hline
        U13         & Als Praxisverantwortliche/-r möchte ich konfigurieren können, welche Benachrichtigungen vorgelesen werden, damit nur relevante Benachrichtigungen vorgelesen werden.                       \\
        \hline
        U14         & Als Praxisverantwortliche/-r möchte ich konfigurieren können, aus welchen Zimmern Sprachverbindungen zu welchen anderen Zimmern aufgebaut werden können, damit Praxismitarbeitende das System effizient bedienen können.     \\
        \hline
        U15         & Als Praxisverantwortliche/-r möchte ich Benachrichtigungen, Geräte und Benutzer wie zuvor konfigurieren können, damit ich das System weiterhin für meine Praxis konfigurieren kann. \\
        \hline
    \end{tabular}\label{tab:userstories2}
\end{table}

\subsubsection{Auftraggeber}

\begin{table}[h]
    \centering
    \begin{tabular}{|l|p{15cm}|}
        \hline
        \textbf{Id} & \textbf{Anforderung}                                                                                                                                                                          \\
        \hline
        U16         & Als Auftraggeber möchte ich bestehende Betriebsinfrastruktur übernehmen, um von der bereits geleisteten Arbeit profitieren zu können.                                                     \\
        \hline
        U17         & Als Auftraggeber möchte ich, dass die bestehenden Komponenten des Systems wo immer möglich weiter verwendet werden, um von der bereits geleisteten Arbeit profitieren zu können.               \\
        \hline
        U18         & Als Auftraggeber möchte ich, dass der bestehende Mobile Client als native iOS Applikation neu entwickelt wird, um Weiterentwicklung, Betrieb und Gerätekompatibilität langfristig zu gewährleisten.                            \\
        \hline
        U19         & Als Auftraggeber möchte ich, dass für Betrieb eigener Services und Bezug externer Dienstleistungen Amazon Webservices verwendet wird, damit ich von bestehender Infrastruktur und Erfahrung profitieren kann. \\
        \hline
    \end{tabular}\label{tab:userstories3}
\end{table}

