\subsection{Meilensteine}

In der Anfangsphase des Projektes wurden folgende Meilensteine definiert:

\begin{table}[h]
    \centering
    \begin{tabular}{|l|p{15cm}|}
        \hline
        \textbf{Id} & \textbf{Beschreibung}                                                                                                                                                                                         \\
        \hline

        M01         & \textbf{Initiale Anforderungsanalyse}

        Die Anforderungen an das Projekt aus der Aufgabenstellungen sind in User Stories dokumentiert.\\
        \hline

        M02         & \textbf{Einarbeit und Setup IOS Umgebung}

        Projektteilnehmende sind mit groben Konzepten der IOS Entwicklung vertraut.
        Die Entwicklungsumgebung ist bereit für die Umsetzung. \\
        \hline

        M03         & \textbf{Evaluation Technologien}

        Die Evaluation der Technologien für  und Gegensprechanalge (VOIP Kommuniation) ist abgeschlossen. \\
        \hline

        M04         & \textbf{Konzepte}

        Die Konzepte für Systemarchitektur, Aufbau und Architektur der mobilen Applikation sowie Anpassungen
        an bestehenden Kompenenten im System sind abgeschlossen. \\
        \hline

        M05         & \textbf{Migration betehender Funktionalität}

        Die Funktionen die im Mobile Client der Projektarbeit IP5 Cloudbasiertes Praxisrufsystem umgesetzt wurden,
        stehen in der neu entwickelten nativen IOS Applikation zur Verfügung. \\
        \hline

        M06         & \textbf{Umsetzung }

        Alle Anforderungen zu der  Funktion sind in der neu entwickelten nativen IOS Applikation umgesetzt. \\
        \hline

        M07         & \textbf{Umsetzung Gegensprechanlage 1:1}

        Alle Anforderungen für die 1:1 Kommunikation über die Funktion Gegensprechanlage sind in der neu entwickelten nativen IOS Applikation umgesetzt. \\
        \hline

        M08         & \textbf{Umsetzung Gegensprechanlage 1:n}
        Alle Anforderungen für die 1:1 Kommunikation über die Funktion Gegensprechanlage sind in der neu entwickelten nativen IOS Applikation umgesetzt. \\
        \hline

        M09         & \textbf{Polishing und Erweiterungen}

        Die Applikation wurde eingehend getestet.
        Bekannte Fehler sind behoben oder dokumentiert.
        Gewünschte Anpassungen und Erweiterungen sind umgesetzt oder dokumentiert.
        Bei allen nicht umgesetzten Anpassungen ist beschrieben, wieso diese nicht umgesetzt werden konnten.
        \\
        \hline

        M10         & \textbf{Abnahme}

        Die Abnahmetests wurden zusammen mit dem Kunden ausgeführt. \\
        \hline

        M11         & \textbf{Projektbericht}

        Der Projektbericht ist vollständig.\\
        \hline

        M12         & \textbf{Abgabe}

        Projektbericht und Quellcode sind fertiggestellt.
        Das Projekt ist abgegeben.\\
        \hline



    \end{tabular}\label{tab:milestones}
\end{table}

\subsubsection{Abweichungen}

Die in Kapitel 2.2 definierten Meilensteine wurden erreicht und das Projekt wurde erfolgreich abgeschlossen.
Der zu Beginn definierte Projektplan konnte grösstenteils eingehalten werden.
Setup- und Konzept-Phase wurden im geplanten Zeitraum abgeschlossen.
Während der Umsetzung ist es hingegen zu mehreren Verzögerungen gekommen.
Die Umsetzung und Testen der Gegensprechanlage hat deutlich mehr Zeit beansprucht, als eingeplant wurde.
Insbesondere die Integration und effiziente Verwendung von WebRTC im nativen iOS Client war anspruchsvoller als erwartet.
Dies ist einerseits auf die mangelhafte Dokumentation der verwendeten Bibliotheken zurückzuführen.
Andererseits wurde schlicht zu wenig Puffer für die Implementation eingeplant.
Die Verzögerungen während der Umsetzung hat dazu geführt, dass weniger Zeit als erhofft für Polishing und Erweiterungen aufgewendet werden konnten.


\clearpage
