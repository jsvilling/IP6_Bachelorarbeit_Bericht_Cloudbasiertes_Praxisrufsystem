\subsection{Text To Speech}

Praxisruf soll es neu Unterstützen, dass empfangene Benachrichtigungen vorgelesen werden.
Um dies zu ermöglichen muss eine Technologie integriert werden, die es erlaubt den Textinhalt einer Benachrichtigung
in Audio zu konvertieren welches abgespielt werden kann.

Grundsätzlich gibt es zwei Varianten wie dies erreicht werden kann.
Die erste Option ist es, die Konvertierung auf dem Mobile Client selbst vorzunehmen.
In diesem Fall ist die Konvertierung teil der Mobile App.
Das IOS Core Framework bietet Packages welches dies erlauben.

Die zweite Option ist es, die Konvertierung an einen Cloud Service zu delegieren.
Dazu muss ein externer Text To Speech Provider angebunden werden, der vom Praxisruf angesprochen werden kann.


\subsubsection{Nativ}

Vorteile \\

Einfach einzubinden \\
Garantierter nativer support \\

Nachteile \\

Starke bindung an Apple. \\
Wenn die Funktion künftig nicht mehr zur Verfügung steht, geht sie verloren \\
Mehr Verantwortung im Mobile Client selbst (weniger klar getrennte Verantwortung) \\
Funktion für Android Client kann damit nicht umgesetzt werden \\


Anbieter \\
Speech Synthesis von IOS\cite{ios_speech_synthesis}

\subsubsection{Cloud}

Vorteile \\

Unabhängig von Apple \\
Einfacher austauschbar (bei entsprechender implementation der Anbindung) \\
Klar getrennte Verantwortung \\
Funktion für Android Clients kann gleich Umgesetzt werden.

Nachteile \\
Zusätzliche Anbindung von Service \\
Mehrere Umsetzmöglichkeiten (Client spricht direkt mit T2S Cloud oder Cloud Service Spricht mit T2s Cloud) \\
Zusätzlicher Point of Failure / Komlexität \\



Anbieter \\

AWS Polly \cite{aws_polly} \\
Unterstützt alles \\
Rest Infrastruktur ist bereits bei Amazon \\
Kunde wünscht einheitlich bei AWS wo möglich \\


\subsubsection{Entscheidung}

Anbindung per Cloud wird genommen.
Die Implementierung kann dadurch komplizierter werden.
Es ist aber sehr wichtig unabhängig zu bleiben.
Apple Version bindet zu stark an Apple.
Kann nicht für künftigen Android Client wiederverwendet werden.
Zukunftssicherer weil der Konkrete T2S Service ausgewechselt werden kann.
Es bestehen SDKs für Java und für Client.
Damit ist die Anbindung direkt aus Client oder aus Cloud Service möglich.
Welche Option implementiert wird, ist im Konzeptteil zu beschreiben.


\clearpage
