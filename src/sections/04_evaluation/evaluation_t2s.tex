\subsection{Text To Speech}

Praxisruf soll es neu Unterstützen, dass empfangene Benachrichtigungen vorgelesen werden.
Um dies zu ermöglichen muss eine Technologie integriert werden, die es erlaubt den Textinhalt einer Benachrichtigung
in Audio zu konvertieren welches abgespielt werden kann.

Grundsätzlich gibt es zwei Varianten wie dies erreicht werden kann.
Die erste Option ist es, die Konvertierung auf dem Mobile Client selbst vorzunehmen.
In diesem Fall ist die Konvertierung teil der Mobile App.
Das IOS Core Framework bietet Packages welches dies erlauben.

Die zweite Option ist es, die Konvertierung an einen Cloud Service zu delegieren.
Dazu muss ein externer Text To Speech Provider angebunden werden, der vom Praxisruf angesprochen werden kann.

\subsubsection{Nativ}

Die Standardbibliothek von für iOS unterstützt das Konvertieren von Text zu Sprache.\cite{ios_speech_synthesis}
Dementsprechend ist es mit Boardmitteln von iOS möglich, Sprach Synthetisierung umzusetzten.
Diese Variante bietet den Vorteil, dass die Synthese ohne das Einbinden von weiteren Frameworks umgesetzt werden kann.
Es ist weiter garantiert, dass die Funktion mit iOS funktionieren wird.
Da die Funktion von Apple selbst zur Verfügung gestellt wird.

Diese Variante hat allerdings den Nachteil, dass eine starke Bindung zu Apple stattfindet.
Sollte Apple sich je entscheiden, diese Funktion nicht mehr zur Verfügung zu stellen, muss die ganze Funktionalität
auf einen anderen Provider umgeschrieben werden.
Weiter hat die Variante den Nachteil, dass sie mehr Funktionalität in den Mobile Client auslagert.
Die Verantwortung des Mobile Clients wird weniger klar getrennt.
Er wäre nicht nur für die Interaktion mit dem Benutzer verantwortlich sonder muss die fachliche Anforderung der Sprachsynthese übernehmen.
Letztlich hat diese Variante den Nachteil, dass dieselbe Funktionaltät für einen Android Client komplett neu entwickelt werden müsste.


\subsubsection{Cloud}

Vorteile \\

Unabhängig von Apple \\
Einfacher austauschbar (bei entsprechender implementation der Anbindung) \\
Klar getrennte Verantwortung \\
Funktion für Android Clients kann gleich Umgesetzt werden.

Nachteile \\
Zusätzliche Anbindung von Service \\
Mehrere Umsetzmöglichkeiten (Client spricht direkt mit T2S Cloud oder Cloud Service Spricht mit T2s Cloud) \\
Zusätzlicher Point of Failure / Komlexität \\



Anbieter \\

AWS Polly \cite{aws_polly} \\
Unterstützt alles \\
Rest Infrastruktur ist bereits bei Amazon \\
Kunde wünscht einheitlich bei AWS wo möglich \\


\subsubsection{Entscheidung}

Anbindung per Cloud wird genommen.
Die Implementierung kann dadurch komplizierter werden.
Es ist aber sehr wichtig unabhängig zu bleiben.
Apple Version bindet zu stark an Apple.
Kann nicht für künftigen Android Client wiederverwendet werden.
Zukunftssicherer weil der Konkrete T2S Service ausgewechselt werden kann.
Es bestehen SDKs für Java und für Client.
Damit ist die Anbindung direkt aus Client oder aus Cloud Service möglich.
Welche Option implementiert wird, ist im Konzeptteil zu beschreiben.


\clearpage
