\subsection{Sprachübertragung}

Praxisruf soll um die Funktion einer Gegensprechanalge erweitert werden.
Um dies zu ermöglichen muss das System Sprachübertragung zwischen Mobile Clients in Echtzeit unterstützten.
In diesem Kapitel werden die Technologien ermittelt, mit denen dies Umgesetzt werden kann.

\subsubsection{WebRTC}

"Mit WebRTC können Sie Ihrer Anwendung Echtzeit-Kommunikationsfunktionen hinzufügen, die auf einem offenen Standard basieren.
Es unterstützt Video-, Sprach- und generische Daten, die zwischen Peers gesendet werden, sodass Entwickler leistungsstarke Sprach- und Videokommunikationslösungen erstellen können.
Die Technologie ist in allen modernen Browsern sowie auf nativen Clients für alle wichtigen Plattformen verfügbar.
Die Technologien hinter WebRTC sind als offener Webstandard implementiert und in allen gängigen Browsern als reguläre JavaScript-APIs verfügbar.
Für native Clients wie Android- und iOS-Anwendungen steht eine Bibliothek mit derselben Funktionalität zur Verfügung."\cite{webrtc}


Varianten \\
Externer Anbieter (twilio) \\

Vorteile: \\

Einfachere Integration auf Client Seite durch Vendor SDK \\
Auf Cloud Service Seite nur Configuration Domain betroffen, keine andere Erweiterung nötig \\
Alles andere kann auf Client Seite erledigt werden.


Nachteile: \\
Kompliziertere Integration mit Client -> Client muss korrekte Verbindungen anhand Button Konfiguration aufmachen \\
Verantwortung des Clients wächst. Aktuell hat er nur Verantwortung zum Empfangen, die Arbeit wird immer von andern gemacht \\


Self Hosted (WebRtc Server als Teil von cloud service implementieren) \\

Vorteile: \\

Vermittlung und Signaling kann vom Cloud Service übernommen werden. \\
Grössere Flexibilität, da Vermittlung in Cloud Serivce \\
Evtl. Synergie mit Rules Engine in Configuration Domain \\

Nachteile: \\

Kompliziertere Einbindung auf Client Seite.


\subsubsection{Meeting Solution}

AWS Chime, Teams oder Ähnliches als Basis \\
Nicht wirklich p2p? \\
Starke Vendor Bindung \\



\subsection{Full on VOIP mit Telefonnummern}

Not feasible \\
Nicht was wir brauchen, da nur vordefinierte calls zwischen clients gemäss konfiguration möglich sein sollen \\
Es ist nicht gefordert und nicht gewünscht dass "irgendjemand" per telefonnummer, den client erreichen kann. \\


\subsubsection{Entscheidung}

WebRTC self hosted. \\

Synergie Rules Engine \\
Dispatching in Verantwortung Cloud Service \\
SDK gut genug \\
Nicht vendor abhängig \\


\clearpage
