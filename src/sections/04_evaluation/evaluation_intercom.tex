\subsection{Sprachübertragung}

Praxisruf soll um die Funktion einer Gegensprechanalge erweitert werden.
Um dies zu ermöglichen muss das System Sprachübertragung zwischen Mobile Clients in Echtzeit unterstützten.
In diesem Kapitel werden die Technologien ermittelt, mit denen dies Umgesetzt werden kann.

\subsubsection*{WebRTC}

"Mit WebRTC können Sie Ihrer Anwendung Echtzeit-Kommunikationsfunktionen hinzufügen, die auf einem offenen Standard basieren.
Es unterstützt Video-, Sprach- und generische Daten, die zwischen Peers gesendet werden, sodass Entwickler leistungsstarke Sprach- und Videokommunikationslösungen erstellen können.
Die Technologie ist in allen modernen Browsern sowie auf nativen Clients für alle wichtigen Plattformen verfügbar.
Die Technologien hinter WebRTC sind als offener Webstandard implementiert und in allen gängigen Browsern als reguläre JavaScript-APIs verfügbar.
Für native Clients wie Android- und iOS-Anwendungen steht eine Bibliothek mit derselben Funktionalität zur Verfügung."\cite{webrtc}


\subsubsection*{Other Option}

\clearpage
