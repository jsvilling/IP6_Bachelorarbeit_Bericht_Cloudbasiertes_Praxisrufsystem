
\subsection{Mobile Client}

Mit dem Projekt ''IP5 Cloudbasiertes Praxisrufsystem''\cite{ip5} wurde bereits eine mobile Applikation für Praxisruf umgesetzt.
Mit dieser Applikation können bereits heute Benachrichtungen über Praxisruf versendet und empfangen werden.
Die bestehende Applikation wurde mit Nativescript als Multi-Platform Applikation gebaut.
Um die Wartbarkeit und Hardware- sowie Betriebssystemkomaptibilität zu gewährleisten wurde im Fazit des Vorgängerprojekts empfohlen,
die Applikation neu als native Applikation für iOS und Android zu schreiben.\cite{ip5}

Mit diesem Projekt soll die Applikation dementsprechend neu als native iOS Applikation umgesetzt werden.
Dabei ist es wichtig, dass sämtliche bestehende Funktionalität auch im neu entwickelten nativen Mobile Client zur Verfügung steht.
Um weiterhin Benachrichtungen senden und empfangen zu können, muss die gewählte Technologie es ermöglichen Firebase Cloud Messaging anzubinden
und Push Benachrichtigungen im Vorder- sowie im Hintergrund empfängen können.
Weiter muss die Technologie es ermöglichen, Hintergrundtasks zu erstellen und Audiosignale abzuspielen.
Dadurch wird es möglich, regelmässig ein Signal abzuspielen um die Praxismitarbeitenden an verpasste Benachrichtigungen zu erinnern.

\subsubsection*{Programmiersprache}

Für die Entwicklung von nativen iOS Applikationen ist die Programmiersprache Swift als Standard gesetzt.\cite{ios_swift}


\subsubsection*{Frameworks}

Für die Umsetzung von iOS Applikationen stellt Apple die zwei Frameworks UIKit\cite{ios_uikit} und SwiftUI\cite{ios_swift_ui} zur Verfügung.
UIKit ist das ältere der beiden Frameworks und ist seit iOS 2.0 verfügbar.
Dementsprechend ist das Framework ausgereifter und bietet viele Funktionen zur Integration einer Applikation mit iOS.
Es hat allerdings den Nachteil das es schwerer zu erlernen und langsamer zu schreiben ist. (Citation Needed)

SwiftUI ist deutlich neuer als UIKit und steht seit iOS 13.0 zur Verfügung.
Es hat eine tiefere Einstiegshürde als UIKit und ist grundsätzlich einfacher zu schreiben und warten (Citation Needed).
In den Worten von Apple selbst: "SwiftUI helps you build great-looking apps across all Apple platforms with the power of Swift — and as little code as possible."\cite{ios_swift_ui}

SwiftUI bietet zudem ausgezeichnete Integration des Entwicklungsworkflows in die XCode Entwicklungsumgebung.
Es bietet schnelle live previews der Komponenten die geschrieben werden.
Dies vereinfacht Design und Umsetzung der Ansichten.

\clearpage


\subsubsection*{Unterstützung Features}

Die Ansichten aus dem bestehenden Mobile Client können mit SwiftUI und UIKit umgesetzt werden.
SwiftUI bietet weiter viele Standardkomponenten wie Listenansichten, Formfelder und andere UIKomponenten,
die es einfacher machen eine Benutzeroberfläche zu erstellen die den Look und Feel einer nativen iOS Applikation hat.

Die funktionalen Anforderungen zum Versenden und Empfangen von Benachrichtigungen sowie
dem Abspielen von regelmässigen Erinnerungstönen können unabhängig vom gewählten UI Framework umgesetzt werden.

Für die Integration von Firebase Cloud Messaging stellt Firebase eine iOS Library zur Verfügung.\cite{firebase_github_ios}
Die Registrierung bei Messaging Service\footnote{Vgl. IP5 Kapitel X} sowie das Versenden der Benachrichtigung kann direkt in über Services,
welche die Library zur Verfügung stellt gemacht werden.
Das Empfangen von Benachrichtigungen, benötigt Integration mit dem iOS Betriebssystem (citation needed).
Mit sogenannten AppDelegates\cite{ios_app_delegate} ist es möglich sich in den Lifecycle des Betriebssystems einzuhängen
und auf entsprechende Events zu hören.
Um Hintergrundbenachrichtigungen empfangen zu können, muss die Methode ``HERE_GOES_THE_METHOD`` von UIApplicationDelegate verwendet werden.
AppDelegates sind ein Konzept welches aus UIKit stammt.
SwiftUI verwendet grundsätzlich keine AppDelegates. (Citation Needed)
Es ist allerdings möglich, AppDelegates auch mit SwiftUI zu verwenden.
Dazu müssen diese explizit eingebunden werden. (Citation Needed.)

Erinnerungen können mit Boardmitteln aus den Libraries die Apple zur Verfügung stellt umgesetzt werden.
Einerseits können mit Timer\cite{ios_timer} regelmässig wiederholbare Tasks erfasst werden, die ausgeführt werden, wenn die App geladen ist.
Weiter können über die Klasse BGTaskScheduler\cite{ios_bgtaskscheduler} Tasks erfasst werden, die im Hintergrund ausgeführt werden.

\subsubsection*{Entscheid}

Der native iOS Mobile Client für Praxisruf wird mit Swift und SwiftUI umgesetzt.
Als Zielplatform wird IOS15 verwendet, damit möglichst alle Funktionen aus SwiftUI zur Verfügung stehen.
SwiftUI ist der neue Standard oder zumindest die Richtung in die Apple pushed.
SwiftUI ist leichtgewichtiger und flexibler als UIKit.
Es bietet ausgezeichnete Integration mit der Entwicklungsumgebung XCode und Entwicklungstools zur Umsetzung
von Benutzeroberflächen mit nativem Look und Feel.
Es ist davon auszugehen, dass die Entwicklung dadurch schneller und der resultierende Code schlanker und wartbarer ist.

SwiftUI ist neuer als UIKit und hat dementsprechend noch nicht denselben Umfang.
SwiftUI ist aber mit UIKit kompatibel.
Das heisst für Funktionen, die nur mit UIKIt umgesetzt werden können, kann UIKit verwendet werden.
Diese Teile der Applikation können in Zukunft, wenn diese Funktionen mit SwiftUI umgesetzt werden migriert werden.

\clearpage
