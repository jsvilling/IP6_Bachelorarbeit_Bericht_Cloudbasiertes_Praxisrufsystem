\subsection{Sprachsynthese}

Praxisruf soll es neu Unterstützen, dass empfangene Benachrichtigungen vorgelesen werden.\footnote{F02 Sprachsynthese}
Um dies zu ermöglichen muss eine Technologie integriert werden, die es erlaubt den Textinhalt einer Benachrichtigung in eine Sprachdatei zu konvertieren.
Diese Sprachdatei muss Audio beinhalten, welches vom Mobile Client abgespielt werden kann.

Diese Integration kann mit den Standardbibliotheken für iOS oder durch die Anbindung eines externen Providers umgesetzt werden.
Die Anbindung eines externen Provider kann entweder direkt im Mobile Client implementiert werden.
Alternativ das Backend von Praxisruf\footnote{Cloud Service} an den Provider angebunden werden und dem Mobile Client eine Schnittstelle bieten, um diese Daten abzufragen.

\subsubsection*{Apple Speech Synthesis}

Die Standardbibliothek von für iOS unterstützt das Konvertieren von Text zu Sprache.\cite{ios_speech_synthesis}
Sprachsynthese könnte dadurch ohne die Anbindung eines externen Providers umgesetzt werden.
Durch die Verwendung Funktionalität die Apple selbst zur Verfügung stellt, ist zudem die Kompatibilität mit nativen iOS Clients garantiert.
Gleichzeitig entsteht auch eine starke Bindung an Apple als Sprachsyntheseprovider.
Sollte die Funktion in künftigen Versionen nicht mehr unterstützt werden, müsste die ganze Integration von Sprachsynthese neu implementiert werden.
Weiter bringt diese Variante den Nachteil, dass der Mobile Client komplexer wird, weil er mit einer zusätzlichen Instanz kommunizieren muss.
Letztlich hat diese Variante den Nachteil, dass dieselbe Funktionaltät für einen Android Client komplett neu entwickelt werden müsste.

\subsubsection*{Amazon Polly}

Vom Auftraggeber ist explizit gewünscht, dass Infrastruktur und Services wo möglich über Amazon Webservices bestellt werden.\footnote{Siehe Kapitel 3 - F19}
Mit dem Service ''Polly'' bietet Amazon Webservices einen Service, welcher Text in Sprachdaten verwandeln kann.\cite{aws_polly}
Amazon Webservices stellt dazu Libraries für iOS\cite{aws_polly_ios} als auch für Java zur Verfügung\cite{aws_polly_java}.
Polly kann damit sowohl direkt in den nativen Mobile Client als auch in den Cloud Service integriert werden.

Polly von AWS Webservices bietet eine Library für die Intagration in iOS. \cite{aws_polly_ios}
Mit dieser Library ist es möglich, die Anbindung des externen Sprachsyntheseproviders direkt im Mobile Client zu implementieren.
Damit wird eine Lösung eingesetzt, die analog auch für zukünftige Android Clients eingesetzt werden kann.
Die starke Bindung zu Apple wird durch diese Lösung ebenfalls aufgehoben.
Sie wird allerdings durch eine starke bindung zu Amazon Webservices ersetzt.
Ein Wechsel des Providers bleibt auch in dieser Variante aufwändig.
Weiter bringt auch diese Variante den Nachteil, dass der Mobile Client komplexer wird, weil er mit einer zusätzlichen Instanz kommunizieren muss.

\clearpage

Alternativ ist es möglich, die Anbindung des Sprachsyntheseproviders im Cloud Service vorzunehmen.
Dies hat einerseits den Vorteil, dass alle Clients eine einheitliche Schnittstelle haben können.
Sowohl iOS als auch Android und WebClients können die Audiodaten über genau dieselbe Schnittstelle beziehen.
Dies ermöglicht es auch den Provider in Zukunft auszutauschen ohne in den Clients etwas verändern zu müssen.
Die Option hat zusaätzlich den Vorteil, dass Optimierungen, die nicht Client spezifisch sind getroffen werden können.
So wäre es möglich einen externen File Storage anzubinden auf, welchem Audiodaten gespeichert werden.
Dadurch müsste der Sprachsynthese Provider weniger oft angesprochen werden.
Die Variante hat den Nachteil, dass der Cloud Service komplexer wird.
Durch die Integration von Sprachsynthese muss aber mindestens eine Komponente im System ausgebaut werden.
Wie stark die Komplexität des System wächsts, kann jedoch minimiert werden, indem die neue Funktionalität gekapselt wird.
Sie kann so umgesetzt werden, dass sie unabhängig von restlichen System bleibt und alle nötigen Daten über die Schnittstelle der anderen Module bezieht.\footnote{Siehe Kapitel 5.1}

\subsubsection*{Entscheidung}

Die Sprachsynthese wird durch die Anbindung des externen Providers AWS Polly umgesetzt.
Der Cloud Service übernimmt die Kommunikation mit AWS Polly und bietet eine Schnittstelle, über welche Clients Audiodaten beziehen können.

Durch diesen Ansatz kann die Abhängigkeit zu einem spezifischen Provider minimiert werden und die Anbindung für alle Client Plattformen gleich gelöst werden.
Dies macht diese Variante zukunftssicher und einfacher betreibbar.
Der Einfluss von zusätzlicher Komplexität, die dieser Ansatz mit sich bringt, soll durch eine entsprechende Kapselung in der Systemarchitektur minimiert werden.

\clearpage
