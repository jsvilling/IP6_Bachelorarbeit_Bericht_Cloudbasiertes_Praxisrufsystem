\subsection{Sprachsynthese}

Praxisruf soll es neu Unterstützen, dass empfangene Benachrichtigungen vorgelesen werden.
Um dies zu ermöglichen muss eine Technologie integriert werden, die es erlaubt den Textinhalt einer Benachrichtigung
in Audio zu konvertieren welches abgespielt werden kann.

Grundsätzlich gibt es zwei Varianten wie dies erreicht werden kann.
Die erste Option ist es, die Konvertierung auf dem Mobile Client selbst vorzunehmen.
In diesem Fall ist die Konvertierung teil der Mobile App.
Das IOS Core Framework bietet Packages welches dies erlauben.

Die zweite Option ist es, die Konvertierung an einen Cloud Service zu delegieren.
Dazu muss ein externer Text To Speech Provider angebunden werden, der vom Praxisruf angesprochen werden kann.

\subsubsection*{Nativ in Mobile Client}

Die Standardbibliothek von für iOS unterstützt das Konvertieren von Text zu Sprache.\cite{ios_speech_synthesis}
Dementsprechend ist es mit Boardmitteln von iOS möglich, Sprach Synthetisierung umzusetzten.
Diese Variante bietet den Vorteil, dass die Synthese ohne das Einbinden von weiteren Frameworks umgesetzt werden kann.
Es ist weiter garantiert, dass die Funktion mit iOS funktionieren wird.
Da die Funktion von Apple selbst zur Verfügung gestellt wird.

Diese Variante hat allerdings den Nachteil, dass eine starke Bindung zu Apple stattfindet.
Sollte Apple sich je entscheiden, diese Funktion nicht mehr zur Verfügung zu stellen, muss die ganze Funktionalität
auf einen anderen Provider umgeschrieben werden.
Weiter hat die Variante den Nachteil, dass sie mehr Funktionalität in den Mobile Client auslagert.
Die Verantwortung des Mobile Clients wird weniger klar getrennt.
Er wäre nicht nur für die Interaktion mit dem Benutzer verantwortlich sonder muss die fachliche Anforderung der Sprachsynthese übernehmen.
Letztlich hat diese Variante den Nachteil, dass dieselbe Funktionaltät für einen Android Client komplett neu entwickelt werden müsste.

\subsubsection*{Externer Provider in Mobile Client}

Als zweite Option ist es möglich, die Sprachsynthese an einen externen Provider zu delegieren.
Mit AWS Polly\cite{aws_polly} ist es möglich, den Provider direkt aus einer iOS Applikation anzusprechen.
Diese Variante hat den Vorteil, dass keine Bindung mehr zu Apple vorhanden ist.
Für die Integration mit Android besteht hier ein wenig mehr Synergie.
Es kann für iOS derselbe Provider verwendet werden.
Da AWS Polly auch für Android einen SDK bietet.
Diese Variante hat aber trotzdem noch die Nachteile, dass die Verantwortung des Mobile Clients grösser wird
und dass unterschiedliche SDKs für die Android und iOS Plattformen verwendet werden.

\subsubsection*{Externer Provider über Cloud Service}

Als Dritte Variante ist es möglich, die Sprachsynthese im Cloud Service vorzunehmen.
Da AWS Polly auch einen SDK für Java bietet, ist es möglich, die Synthese dort einzubinden.
Dies hat einerseits den Vorteil, dass alle Clients eine einheitliche Schnittstelle haben können.
Sowohl iOS als auch Android und WebClients können die Audiodaten über genau dieselbe Schnittstelle beziehen.
Dies ermöglicht es auch den Provider in Zukunft auszutauschen ohne in den Clients etwas verändern zu müssen.
Die Option hat zusaätzlich den Vorteil, dass Optimierungen die nicht Client spezifisch sind getroffen werden können.
So wäre es z.B. möglich einen externen File Storage anzubinden auf dem Audiodaten gespeichert werden.
Dadurch muss der Sprachsynthese Provider weniger oft angesprochen werden.
Die Variante hat den Nachteil, dass der Cloud Service komplexer wird.
Durch das Hinzufügen von neuer Funktionaltiät ist das auf Systemebene aber so oder so gegeben.
Der Einfluss davon kan weiter minimiert werden, indem die neue Funktionalität gekapselt wird.
Sie kann so umgesetzt werden, dass sie unabhängig von restlichen system ist und alle nötigen Daten über die
Schnittstelle der anderen Module bezieht. \footnote{Vgl. Kapitel Systemarchitektur}

\subsubsection*{Entscheidung}

Die Sprachsynthese wird durch die Anbindung des externen Providers AWS Polly umgesetzt.
Der Cloud Service übernimmt die Kommunikation mit AWS Polly und bietet eine Schnittstelle über die der Mobile Client Audiodaten beziehen knan.

Durch diesen Ansatz kann die Abhängigkeit zu einem spezifischen Provider minimiert werden und
die Umsetzung für alle Platformen gleich gelöst werden.
Dies macht diese Variante zukunftssicher und einfach wartbar.
Der Einfluss von zusätzlicher Komplexität, die dieser Ansatz mit sich bringt,
soll durch eine entsprechende Kapselung in der Systemarchitektur minimiert werden.

\clearpage
