\section{Benutzerhandbuch}

\subsubsection*{Mobile Client}

\textbf{Anmeldung und Konfiguration}

\begin{enumerate}
    \item Mobile Client Applikation öffnen
    \item Anmeldedaten eingeben und bestätigen
    \item Gewünschte Konfiguration auswählen
\end{enumerate}

\textbf{Benachrichtigungen}

\begin{enumerate}
    \item In Navigationsleiste (unten) ``Home'' auswählen
    \item Im Bereich ''Benachrichtigung'' Button mit dem gewünschten Titel antippen
    \item Die Benachrichtigung wird versendet
\end{enumerate}

\textbf{Sprachverbindungen}

\begin{enumerate}
    \item In Navigationsleiste (unten) ``Home'' auswählen
    \item Im Bereich ''Gegensprechanlage'' Button mit dem gewünschten Titel antippen
    \item Die Sprachverbindung wird aufgebaut.
    \item In der geöffneten Ansicht sehen Sie alle Teilnehmer und deren Status.
    \item Die grauen Buttons in der Ansicht geben die Möglichkeit Mikrofon und Lautsprecher stummzuschalten.
    \item Mit dem rote Button in der Ansicht kann die Verbindung getrennt werden.
\end{enumerate}

\textbf{Inbox}

\begin{enumerate}
    \item In Navigationsleiste (unten) ``Inbox'' auswählen
    \item In dieser Ansicht sehen Sie alle empfangenen noch nicht quittierten Benachrichtigungen
    \item Zum Quittieren einer Benachrichtigung den ``OK'' Button der Benachrichtigung antippen
\end{enumerate}

\textbf{Lokale Einstellungen ändern}
\begin{enumerate}
    \item In Navigationsleiste (unten) ``Einstellungen'' auswählen
    \item Über die Schaltfläche ''Benachrichtigungen vorlesen'' kann das Vorlesen für alle Benachrichtigungen deaktiviert werden.
    \item Über die Schaltfläche ''Anrufe empfangen'' können eingehende Anrufe automatisch abgelehnt werden.
    \item Logout bestätigen
\end{enumerate}

\textbf{Abmeldung}

\begin{enumerate}
    \item In Navigationsleiste (unten) ``Einstellungen'' auswählen
    \item Button ''Abmelden'' antippen.
\end{enumerate}

\textbf{Berechtigungen nach Installation}

Die Praxisruf-App benötigt mehrere lokale Berechtigungen.
Um sicherzustellen, dass alle Funktionen zur Verfügung stehen, müssen diese vom Benutzer bestätigt werden.
Zu diesem Zweck müssen alle Funktionen direkt nach Installation einmal verwendet werden.
Gehen Sie dazu wie folgt vor:

\begin{enumerate}
    \item Öffnen Sie die Applikation und melden Sie sich an.
    \item Bestätigen Sie, dass Praxisruf Benachrichtigungen anzeigen darf.
    \item Starten Sie anschliessend einen Anruf zu einem anderen Gerät.
    \item Bestätigen Sie ebenfalls, dass Praxisruf Geräte im Lokalen Netzwerk entdecken darf.
    Dies ist notwendig, um die Verbindung zu anderen Geräten herzustellen.
    \item Bestätigen Sie, dass Praxisruf auf Mikrofon und Lautsprecher zugreifen dar.
    Dies ist notwendig um Gespräche über die aufgebauten Verbindungen zu führen.
\end{enumerate}

\textbf{Netzwerk}

Um sicherzustellen, dass Daten ausgetauscht werden können, müssen Sie auf folgendes achten:

\begin{enumerate}
    \item Benachrichtigungen können nur ausgetauscht werden, wenn die Geräte eine Verbindung ins Internet haben.
    \item Anrufe können nur zwischen Geräten die im selben lokalen Netzwerk sind geführt werden.
\end{enumerate}

\subsubsection*{Admin UI}

\textbf{Anmeldung}

\begin{enumerate}
    \item Nach Installation des Systems Anmeldedaten und URL des Admin UI beim Betreiber einholen
    \item Admin UI im Browser öffnen
    \item Anmeldedaten eingeben und bestätigen
\end{enumerate}

\textbf{Konfiguration verwalten}
\begin{enumerate}
    \item Melden Sie sich im Admin UI an.
    \item Wählen Sie auf der Linken Seite die Kategorie, die Sie verwalten möchten-
    \item Auf dieser Seite können Sie nun Einträge zur gewählten Kategorie verwalten:
    \begin{itemize}
        \item Klicken Sie oben rechts auf die Schaltfläche ``Create'' um einen neuen Eintrag zu erstellen.
        \item Klicken Sie auf einen Eintrag in der Liste um ihn zu bearbeiten.
        \item Klicken Sie die Checkbox auf der Linken Seite eines Eintrages und dann auf die Schaltfläche ``löschen'' um einen Eintrag zu löschen.
    \end{itemize}
\end{enumerate}

\textbf{Praxisruf konfigurieren}
\begin{enumerate}
    \item Melden Sie sich im Admin UI an.
    \item Erfassen Sie in der Kategorie ``User'' mindestens einen Benutzer.
    \item Erfassen Sie in der Kategorie ``Client'' pro Endgerät in Ihrem System einen Eintrag und weisen Sie einem Benutzer zu.
    \item Erfassen Sie in der Kategorie ``Notification Types'' alle Benachrichtigungen die Sie zur Verfügung stellen möchten.
    \item Erfassen Sie in der Kategorie ``CallTypes'' alle Sprachverbindungen die Sie zur Verfügung stellen möchten.
    \item Erfassen Sie in der Kategorie ``Configurations'' pro Gerät einen Eintrag und weisen Sie ihn dem entsprechenden Client zu.
    \item Unter Notification Types können die Benachrichtigungen auswählen, die auf diesem Gerät zum Versenden verfügbar sind.
    \item Unter Rule Parameters können Sie Regeln definieren, welche Benachrichtigungen diesem Client zugestellt werden.

    \begin{itemize}
        \item Mit dem Rule Type ``Sender'', werden alle Benachrichtigungen vom angegebenen Sender Empfangen.
        \item Mit dem Rule Type ``NotificationType'', werden alle Benachrichtigungen mit dem angegebenen Typ empfangen.
    \end{itemize}
\end{enumerate}

\clearpage
