\section{Installationsanleitung}

In diesem Kapitel wird beschrieben wie Praxisruf mit Amazon Webservices betrieben werden kann.
Die hier aufgeführten Anleitungen sind aus dem Projektbericht ''IP5 Cloudbasiertes Praxisrufsystem''~\cite{ip5} übernommen und wo nötig ergänzt.
Sämtliche Anleitungen behandeln ausschliesslich eine Neuinstallation des Systems und beinhalten keine Migrationsanleitng auf Basis des Vorgängerprojekts.

\subsubsection*{Firebase Messaging}

Im Folgenden wird beschrieben wie Firebase Cloud Messaging konfiguriert werden muss, um Benachrichtigungen in Praxisruf zu ermöglichen.
Die Konfiguration von Firebase Cloud Messaging muss vor Installation von Cloudservice und Mobile Client vorgenommen werden.
Diese Anleitung hat sich im Vergleich zum Vorgängerprojekt nicht verändert und ist direkt aus dem Anhang D des Projektberichts ''IP5 Cloudbasiertes Praxisrufsystem''~\cite{ip5} übernommen.

\begin{enumerate}
    \item Öffnen Sie die \href{https://console.firebase.google.com/}{\textit{Firebase Console}} und erstellen Sie ein neues Projekt.
    \item Navigieren Sie in das Menü ``Project Settings''.
    \item Navigieren Sie zum Tab ``Service Accounts''.
    \item Klicken Sie den Button ``Generate new private Key'' und bestätigen Sie die Eingabe.
    \item Speichern Sie die generierte Datei an einem Ort Ihrer Wahl.
    \item Kopieren Sie den Dateiinhalt und Benutzen Sie ein Werkzeug Ihrer Wahl\footnote{z.B. \url{https://www.base64decode.org/}} um den Inhalt mit Base64 zu kodieren.
    \item Speichern Sie den Base64 kodierten Dateiinhalt an einem Ort Ihrer Wahl. Sie werden diesen Wert für die Konfiguration des Cloud Services benötigen.
    \item Öffnen Sie erneut die \href{https://console.firebase.google.com/}{\textit{Firebase Console}}
    \item Navigieren Sie in das Menü ``Project Settings''.
    \item Erstellen Sie für die gewünschte Plattform (iOS, Android oder Web app) eine Anwendung via ``Add App''.
    \item Verwenden Sie für den Android package name bzw.\ die iOS bundle ID den package\_name des Mobile Clients. (ch.fhnw.ip5.praxisintercom.client)
    \item Vergeben Sie einen beliebigen App-Nickname.
    \item Laden Sie nach dem Registrieren die Konfigurationsdatei GoogleService-Info.plist (iOS) oder google-services.json (Android) herunter.
    \item Legen Sie die Konfigurationsdatei in einem Verzeichnis Ihrer Wahl ab.
\end{enumerate}

Mehr Informationen zu Firebase und der Integration von Firebase Projekten finden Sie in der \href{https://firebase.google.com/docs/projects/learn-more#setting_up_a_firebase_project_and_registering_apps}{offiziellen Dokumentation.}\cite{understand-firebase}

\clearpage

\subsubsection*{Cloudservice}

\textit{Installation mit Amazon Webservices}

Im Folgenden wird beschrieben wie die Cloud Service Applikation mit AWS betrieben werden kann.
Diese Anleitung wurde im Vergleich zum Vorgängerprojekt erweitert um die Konfiguration von Gegensprechanlage und Sprachsynthese zu erweitert.
Dazu wurde die Anleitung aus dem Anhang D des Projektberichts ''IP5 Cloudbasiertes Praxisrufsystem''~\cite{ip5} erweitert.

\begin{enumerate}
    \item Stellen Sie sicher, dass der Quellcode der Cloud Service Applikation in einem Git Repository bei einem der Anbieter Github, Gitlab, BitBucket oder AWS CodeCommit zur Verfügung steht.
    \item Erstellen Sie einen mit AWS ein Elastic Beanstalk Environment.
    \begin{enumerate}
        \item Folgen Sie dazu der \href{https://docs.aws.amazon.com/elasticbeanstalk/latest/dg/GettingStarted.CreateApp.html}{\textit{offiziellen Anleitung}}\cite{aws-elastic}
        \item Wählen Sie unter Plattform ``Java'' und die dazugehörigen Standardeinstellungen.
        \item Wählen Sie unter Application Code ``Sample Application''.
    \end{enumerate}
    \item Erstellen Sie mit AWS RDS eine Datenbank für die Cloud Service Applikation
    \begin{enumerate}
        \item Öffnen Sie die Beanstalk Console.
        \item Folgen Sie der \href{https://docs.aws.amazon.com/elasticbeanstalk/latest/dg/using-features.managing.db.html}{\textit{offiziellen Anleitung}}\cite{aws-elastic-rds} um eine Relationale Datenbank an das Beanstalk Environment anzubinden.
        \item Wählen Sie als Datenbank Engine ``postgres'' in der Version 13.3.
        \item Wenn sie oben genannter Anleitung folgen, ist keine weitere Konfiguration für die Datenbankanbindung im Cloud Service nötig.
        Sollten Sie wählen, die Datenbank auf eine andere Art zu Betreiben müssen in Schritt 4 die Umgebungsvariablen RDS\_HOSTNAME, RDS\_PORT, RDS\_DB\_NAME, RDS\_USERNAME und RDS\_PASSWORD mit den entsprechenden Werten konfiguriert werden.
    \end{enumerate}
    \item Um die Anbindung an Amazon Polly nutzen zu können, muss ein Benutzer für den Zugriff definiert und berechtigt werden.
    \begin{enumerate}
        \item Erstellen Sie einen IAM Benutzer für den Zugriff auf Amazon Polly. Folgen Sie dazu der \href{https://aws.amazon.com/premiumsupport/knowledge-center/create-access-key/}{\textit{offiziellen Anleitung}}.
        \item Stellen Sie sicher, dass der Benutzer die Rolle \textit{AmazonPollyFullAccess} zugewiesen hat.
        \item Erstellen Sie einen Access Key für den erstellten Benutzer. Folgen Sie dazu der \href{https://aws.amazon.com/premiumsupport/knowledge-center/create-access-key/}{\textit{offiziellen Anleitung}}.
        \item Speichern Sie den generierten Access Key ID und Access Secret Key an einem sicheren Ort. Sie werden für die weitere Konfiguration benötigt.
    \end{enumerate}

    \item Definieren Sie die nötigen Umgebungsvariablen für die Cloud Service Applikation:
    \begin{enumerate}
        \item Folgen Sie der \href{https://docs.aws.amazon.com/elasticbeanstalk/latest/dg/environments-cfg-softwaresettings.html}{\textit{offiziellen Anleitung}}\cite{aws-elastic-env} um die nötigen Umgebungsvariablen zu setzen:
        \item Name: FCM\_CREDENTIALS, Wert: Firebase Credentials mit Base 64 Encoded\footnote{Siehe Installationsanleitung Firebase Messaging}
        \item Name: SPRING\_PROFILES\_ACTIVE, Wert: aws.
        \item Name: ADMIN\_ORIGIN, Wert: Admin UI Domain
        \item Name: JWT\_SECRET\_KEY, Wert: Zufälliger 64 Bit String\footnote{\url{https://www.grc.com/passwords.htm}}
        \item Name: AWS\_ACCESS\_KEY, Wert: Access Key ID, welcher in Schritt 4 erstellt wurde.
        \item Name: AWS\_SECRET\_KEY, Wert: Access Secret Key, welcher in Schritt 4 erstellt wurde.
        \item Name: NOTIFICATION\_TYPE\_FOR\_UNAVAILABLE, \\ Wert: '63d530ab-48af-4597-a9fd-2fb4c9700c55'. Oder die ID des NotificationTypes, welcher für verpasste Anrufe verwendet werden soll.
    \end{enumerate}
    \item Konfigurieren Sie AWS CodeBuild um die Cloud Service Applikation mit AWS bauen zu können.
    \begin{enumerate}
        \item Öffnen Sie die AWS Console und wählen Sie unter Services ``Code Pipeline'' aus.
        \item Wählen Sie die Option ``Create build project''.
        \item Geben Sie unter Project Configuration einen passenden Namen ein.
        \item Wählen Sie unter Source den gewünschten Anbieter und geben Sie die Repository URl sowie die gewünschte Version (Branch Name) ein.
        \item Wählen Sie unter Buildspec ``Use a buildspec file''. Die projektspezifische Build Konfiguration ist in der Cloud Service Applikation bereits enthalten.
        \item Bestätigen Sie die Eingaben.
    \end{enumerate}
    \item Konfigurieren Sie AWS CodePipeline um die Cloud Service Applikation zu installieren:
    \begin{enumerate}
        \item Öffnen Sie die AWS Console und wählen Sie unter Services ``Code Pipeline'' aus.
        \item Wählen Sie die Option ``Create New Pipeline''.
        \item Geben Sie in Schritt 1 einen Namen für die Pipeline an und wählen Sie ``Next''.
        \item Wählen Sie in Schritt 2 gewünschten Anbieter und folgen sie dem Wizard um das Git Repository des Cloud Services anzubinden.
        \item Wählen Sie in Schritt 3 AWS CodeBuild und das CodeBuild Projekt welches Sie in Schritt 3 erstellt haben.
        \item Wählen Sie in Schritt 4 AWS Elastic Beanstalk und die in Schritt 2 definierten Instanzen.
        \item Bestätigen Sie die Eingaben.
    \end{enumerate}
    \item Stellen Sie sicher, dass die installierte Cloud Service Applikation über HTTPS erreichbar ist. Dies ist für die Kommunikation mit Mobile Client Instanzen zwingend notwendig.
    \begin{enumerate}
        \item Folgen Sie dazu der offiziellen Anleitung\href{https://aws.amazon.com/premiumsupport/knowledge-center/elastic-beanstalk-https-configuration/}{\textit{offiziellen Anleitung}}\cite{aws-elastic-https}.
        \item Am Cloud Service sind dabei keine Änderungen nötig.
    \end{enumerate}
    \item Stellen Sie sicher, dass die in Schritt 7 konfigurierte Pipeline erfolgreich ausgeführt wurde.
    Durch das erste Starten der Applikation wird das Datenbankschema automatisch initialisiert.
    \item Als letzter Schritt müssen zwei Scripts manuell auf der Datenbank ausgeführt werden.
    Gehen Sie dazu wie folgt vor:
    \begin{enumerate}
        \item Richten Sie den Datenbankzugriff auf die in Schritt 3 erstellte Datenbank gemäss der \href{https://docs.aws.amazon.com/AmazonRDS/latest/UserGuide/USER_ConnectToPostgreSQLInstance.html}{\textit{offiziellen Anleitung}}\cite{aws-elastic-rds-access}. ein.
        \item Wählen Sie ein Passwort
        \item Erstellen Sie ein Hash des gewählten Passwort mit dem "bcrypt"-Algorithmus.\footnote{\url{https://bcrypt-generator.com/}}
        \item Passen Sie die Parameter in folgenden Script an und führen es auf der Datenbank aus:
        \lstinputlisting[caption=createadmin.sql,language=sql,label={lst:createadmin.sql}]{listings/createadmin.sql}
        \item Erstellen Sie die Konfiguration für Benachrichtigung bei verpassten Anrufen in der Gegensprechanlage, indem Sie folgendes Script ausführen:
        \lstinputlisting[caption=missedcallnotification.sql,language=sql,label={lst:missedcallnotification.sql}]{listings/missedcallnotification.sql}
    \end{enumerate}

\end{enumerate}

\clearpage

\textit{Lokale Installation}

Die folgende Anleitung beschreibt, wie der Cloudservice lokal gestartet werden kann.
Diese Art den Cloudservice zu installieren sollte nur zu Testzwecken verwendet werden.

\begin{enumerate}
    \item Stellen Sie sicher, dass auf Ihrem System Java in der Version 13 oder neuer installiert ist.
    \item Installieren Sie PostgreSQL auf Ihrem System und erstellen Sie eine Datenbank für den Cloudservice.
    Verwenden Sie die folgenden Informationen für die Konfiguration: 
    \begin{enumerate}
        \item Port: 5433
        \item Benutzer: admin
        \item Passwort: secret
        \item Name der Datenbank: intercom\_configuration\_db\_local
    \end{enumerate}
    \item Führen Sie die Skripts aus Schritt 10 aus dem vorhergehenden Abschnitt auf der erstellten Datenbank aus.
    \item Stellen Sie sicher, das die Umgebungsvariablen aus Schritt 5 der Anleitung für Installation mit AWS auf ihrem System gesetzt sind.
    \item Kopieren Sie die .jar Datei des Cloudservices auf ihr System
    \item Öffnen Sie ein Terminal im Verzeichnis in welchen die .jar Datei abgeleg ist.
    \item Führen Sie den Befehl ''java -jar -Dspring.profiles.active=local ./$<$filename$>$.jar'' aus.
    \item Der Cloudservice wird gestartet und ist danach unter der localhost:5000 erreichbar.
    \item Die API Dokumentation des Cloudservice ist unter localhost:5000/swagger-ui.html zu finden.
\end{enumerate}


\textit{Cloudservice Release bauen}

Mit dem Sourcecode des Cloudservice wurde eine ausführbare .jar-Datei basierend auf dem aktuellen Stand des Cloudservice veröffentlicht.
Sie finden dieses im Verzeichnis release im Rootverzeichnis des Cloudservice-Projektes.
Dieses Image kann, wie im vorhergehenden Abschnitt beschrieben installiert werden.

Um einen neuen Release zu bauen, gehen Sie wie folgt for:

\begin{enumerate}
    \item Öffnen Sie ein Terminal im Rootverzeichnis des Cloudservice-Projektes.
    \item Setzen Sie den Befhel ''./gradlew build'' ab.
    \item Nach Abschluss des Befehls finden Sie die ausführbare .jar Datei im Verzeichnis ./praxisruf-app/build/libs
    (Relativ zum Rootverzeichnis des Cloudservice-Projektes).
\end{enumerate}

\clearpage

\subsubsection*{Admin UI}

\textit{Installation mit Amazon Webservices}

Im Folgenden wird beschrieben wie die Admin UI Applikation mit AWS betrieben werden kann.
Diese Anleitung hat sich im Vergleich zum Vorgängerprojekt nicht verändert und ist direkt aus dem Anhang D des Projektberichts ''IP5 Cloudbasiertes Praxisrufsystem''~\cite{ip5} übernommen.

\begin{enumerate}
    \item AWS Amplify Service aufsetzen:
    \begin{enumerate}
        \item Amazon Webservice unterstützt die Anbindung von Github, Gitlab, BitBucket und AWS CodeCommit. Stellen Sie sicher, dass der Quellcode der Admin UI Applikation in einem Git Repository zur Verfügung steht.
        \item Folgen Sie den Schritten in der \href{https://docs.aws.amazon.com/amplify/latest/userguide/getting-started.html}{\textit{offiziellen Anleitung}}.\cite{aws-amplify}
    \end{enumerate}
    \item Verbindung zum Cloud Service konfigurieren:
    \begin{enumerate}
        \item Öffnen sie die AWS Amplify Konsole für die in Schritt 1 erstellte Applikation.
        \item Wählen Sie den Menüpunkt ``Environment Variables''
        \item Erstellen Sie eine neue Variable mit dem Namen REACT\_APP\_BACKEND\_BASE\_URI. Setzten Sie als Wert dafür die Domain, unter welcher der Cloud Service erreichbar ist.
    \end{enumerate}
    \item Konfigurieren Sie eine Domain für die Admin UI Applikation.
    \begin{enumerate}
        \item Folgen Sie dazu den Schritten in der \href{https://docs.aws.amazon.com/amplify/latest/userguide/custom-domains.html}{\textit{offiziellen Anleitung}}.\cite{aws-amplify-domain} folgen.
    \end{enumerate}
\end{enumerate}

\textit{Lokale Installation}

Das Admin UI kann lokal mit einem Webserver betrieben werden.
Diese Art das Admin UI zu betreiben sollte nur zu Testzwecken verwendet werden.

\begin{enumerate}
    \item Stellen Sie sicher, dass npm\footnote{\href{https://www.npmjs.com/}{https://www.npmjs.com/}}  und Node.js\footnote{\href{https://nodejs.org/en/}{https://nodejs.org/en/}} auf Ihrem System installiert sind.
    \item Stellen Sie sicher, dass der Webserver ''serve''\footnote{\href{https://www.npmjs.com/package/serve}{https://www.npmjs.com/package/serve}} auf Ihrem System installiert ist.
    \item Kopieren Sie die Dateien des Releases in ein Verzeichnis auf ihrem System.
    \item Öffnen Sie ein Terminal in disem Verzeichnis.
    \item Führen Sie den Befehl ''serve -s .'' aus.
    \item Das Admin UI ist danach unter localhost:3000 erreichbar.
    Wenn Sie den Release Build, der mit der Abgabe des Projektes veröffentlicht wurde verwenden, muss der Cloudservice unter localhost:5000 erreichbar sein.
\end{enumerate}


\textit{Admin UI Release bauen}

Mit dem Sourcecode des Admin UIs wurde ein Release Build basierend auf dem aktuellsten Stand des Projektes geliefert.
Dieser kann, wie im vorhergehenden Abschnitt beschrieben installiert werden.

Um einen neuen Release Build auszuführen, gehen Sie wie folgt vor:

\begin{enumerate}
    \item Stellen Sie sicher, dass npm und Node.js auf Ihrem System installiert sind.
    \item Stellen Sie sicher, dass folgende Umgebungsvariable gesetzt ist:
    \begin{enumerate}
        \item Name: REACT\_APP\_BACKEND\_BASE\_URI   Wert: Basis URL des Cloudservice.   
    \end{enumerate}
    \item Öffnen Sie ein Terminal im Rootverzeichnis des Admin UI Projektes.
    \item Führen Sie den Befehl ''npm install''  aus.
    \item Führen Sie den Befehl ''npm run build'' aus.
    \item Die Releasedateien das Admin UIs finden sich danach im Verzeichnis ./build unter dem Rootverzeichnis des Projektes. 
\end{enumerate}


\clearpage

\subsubsection*{Mobile Client}

Dieses Kapitel beschreibt wie die native iOS Applikation auf Geräten installiert werden kann.

Mit dem Quellquode des Mobile Clients ist ein deploybares Image veröffentlicht.
Dieses beinhaltet die Credentials und Konfigurationen für die Infrastruktur die während diese Projektarbeit aufgebaut wurden.
Das Image ist im Verzeichnis release unter dem Rootverzeichnis des Mobile Client-Projektes abgelegt.
Solange diese Infrastruktur besteht und verwendet wird, kann das veröffentlichte Image verwendet werden, um Praxisruf zu betreiben.

Das veröffentlichte Image und Images die neu gebaut werden, können wie folgt installiert werden:

\textit{Installation eines bestehenden Image}

\begin{enumerate}
    \item Öffnen Sie das Projekt \textit{praxisruf-ios-mobile-client} in die XCode Entwicklungsumgebung.
    \item Öffnen Sie die Deviceliste innerhalb von Xcode. (cmd + shift + 2)
    \item Installieren Sie das .ipa File des Images via Drag and Drop auf dem IPad.
\end{enumerate}

\textit{Mobile Client Image bauen}

Die Integration von Firebase Cloud Messaging in den Mobile Client muss zur Buildzeit passieren.
Dies bedeutet, dass die Applikation neu gebaut werden muss.
Dazu gehen Sie wie folgt vor:

\begin{enumerate}
    \item Öffnen Sie das Projekt \textit{praxisruf-ios-mobile-client} in die XCode Entwicklungsumgebung.
    \item Öffnen Sie die Projekteinstellungen (Doppelklick auf das Topverzeichnis \textit{praxisruf-ios-mobile-client} in XCode)
    \item Navigieren Sie zu Build Settings $>$ User Defined
    \item Setzten Sie den Wert für BASE\_URL\_HTTPS auf die URL unter welcher die HTTP API des Cloudservice erreichbar ist. Dies entspricht standardmässig $<$serverUrl$>$/api
    \item Setzten Sie den Wert für BASE\_URL\_WSS unter welcher die Websocket API des Cloudservice erreichbar ist. Dies entspricht standardmässig $<$serverUrl$>$
    \item Wenn Sie die Instanz des Cloudservice unter ''www.praxisruf.ch'' verwenden, müssen Sie die Werte nicht anpassen.
    \item Speichern Sie die Konfigurationsdatei aus Schritt 13 der Firebase Cloud Messaging Installationsanleitung im XCode Projekt unter praxisruf-ios-mobile-client/Resources/GoogleService-Info.plist ab.
    \item Führen Sie den Build der Applikation aus. (shift + command + R).
    \item Wählen Sie in der Top Menüleiste von XCode Product $>$ Archive.
    \item Warten Sie bis die Aktion beendet hat und der Dialog \textit{Archives} sich öffnet.
    \item Wählen Sie im Dialog \textit{Distribute App}
    \item Wählen Sie die Option\textit{Ad Hoc} und fahren mit \textit{next} fort.
    \item Aktivieren Sie die Option \textit{Rebuild from Bitcode} deaktivieren Sie alle anderen Optionen und fahren mit \textit{next} fort.
    \item Wählen Sie \textit{Automatically manage signing} und fahren mit \textit{next} fort.
    \item Bestätigten Sie mit \textit{Export}
    \item Geben Sie das Verzeichnis, an in welches die Applikation exportiert werden soll.
    \item Die in diesem Verzeichnis abgelegte .ipa Datei kann nun, wie in \textit{Installation eines bestehenden Image} beschrieben, auf Geräten installiert werden.
\end{enumerate}


\clearpage
