\section{Features und Testszenarien}

\subsubsection*{F01 - Benachrichtigungen Versenden}

Die Folgenden Szenarien sind im Originaltext aus dem Anhang E des Berichts des Vorgängerprojekts übernommen.\cite{ip5}

\begin{tabbing}
    Left \= Middle \= Right \= Right \kill
    Scenario S01: \> \> \> Benachrichtigung versenden\\ \\
    Given:  \> \> \> Benutzer ist vollständig angemeldet\\
    And: \>    \> \> Mindestens ein Empfänger ist konfiguriert\\
    When:   \> \> \> Praxismitarbeiter tippt auf einen Benachrichtigungs-Button\\
    Then:   \> \> \> Benachrichtigung wird an den zentralen Cloud Service gesendet\\
    And: \>    \> \> Benachrichtigung wird an alle Mobile Clients versendet\\
    \> \>  \> die sich für diese Benachrichtigung subscribed haben weitergeleitet\\
    And:   \> \> \> Praxismitarbeiter erhält optische Rückmeldung, dass Benachrichtigung versendet wurde\\
    \\
    Scenario S02: \>  \> \> Keine Empfänger konfiguriert\\ \\
    Given:  \> \> \> Benutzer ist vollständig angemeldet\\
    And:  \> \>   \> Kein Empfänger ist konfiguriert\\
    When:  \> \>  \> Praxismitarbeiter tippt auf einen Benachrichtigungs-Button\\
    Then:  \> \>  \> Benachrichtigung wird an den zentralen Cloud Service gesendet\\
    And:  \> \>   \> Benachrichtigung wird nicht weitergeleitet\\
    \\
    Scenario S03: \>  \> \> Empfangen\\ \\
    Given: \>  \> \>  Eine Benachrichtigung wurde von Mobile Client versendet\\
    When: \>  \> \>   Cloud Service Notification an Empfänger Mobile Client weiterleitet\\
    Then: \>  \> \>   Wird die Benachrichtigung vom Empfänger Mobile Client empfangen\\
    And: \>  \> \>    In einer Übersicht für empfangene Benachrichtigung angezeigt.\\
    \\
    Scenario S04: \> \> \>  Fehler Rückmeldung\\ \\
    Given: \> \> \>   Eine Benachrichtigung wurde von Mobile Client versendet\\
    When: \> \> \>    Weiterleitung von Cloud Service Notification an Empfänger schlägt auf Service Seite fehl\\
    Then: \> \> \>    Der Praxismitarbeiter wird über den Fehler informiert\\
    And: \> \> \>     Der Praxismitarbeiter hat die Möglichkeit die fehlgeschlagenen Benachrichtigungen zu wiederholen\\
    \\
    Scenario S05: \> \> \>  Confirm Retry\\ \\
    Given: \> \> \>   Benachrichtigung ist fehlgeschlagen\\
    And: \> \> \>     Dialog zum Wiederholen wird angezeigt\\
    When: \> \> \>    Praxismitarbeiter bestätigt, dass wiederholt werden soll\\
    Then: \> \> \>    Der Cloudservice versucht erneut, die fehlgeschlagenen zuzustellen\\
    \\
    Scenario S06: \> \> \>  Cancel Retry\\ \\
    Given: \> \> \>   Benachrichtigung ist fehlgeschlagen\\
    And: \> \> \>     Dialog zum Wiederholen wird angezeigt\\
    When: \> \> \>    Praxismitarbeiter klick, dass nicht wiederholt werden soll\\
    Then: \> \> \>    Werden die fehlgeschlagenen nicht wiederholt\\
    And: \> \> \>     Zurück zur Notificationsansicht\\
    \\
    Scenario S07: \> \> \>  Foreground\\ \\
    Given: \> \> \>   Mobile Client ist geöffnet\\
    When: \> \> \>    Eine Benachrichtigung wird vom Mobile Client empfangen\\
    Then: \> \> \>    Ein Audio Signal erklingt\\
    \\
    Scenario S08: \> \> \>  Background\\ \\
    Given: \> \> \>   Mobile Client läuft im Hintergrund\\
    When: \> \> \>    Eine Benachrichtigung wird vom Mobile Client empfangen\\
    Then: \> \> \>    Ein Audio Signal erklingt\\
    And: \> \> \>     Eine Push-Benachrichtigung wird angezeigt\\
    \\
    Scenario S09: \> \> \>  Nicht Quittiert\\ \\
    Given: \> \> \>   Mobile Client ist geöffnet\\
    And: \> \> \>     Eine Benachrichtigung wurde empfangen\\
    When: \> \> \>    Benachrichtigung wird nicht quittiert\\
    Then: \> \> \>    Ein Audio Signal erklingt\\
    And: \> \> \>     Das Audio Signal wiederholt sich alle 30 Sekunden, bis die Benachrichtigung quittiert wurde.\\
    \\
    Scenario S10: \> \> \>  Startbildschirm, wenn nicht angemeldet\\ \\
    Given: \> \> \>   Mobile Client is geöffnet\\
    When: \> \> \>  Benutzer ist nicht angemeldet\\
    Then: \> \> \>  Benutzer wird zum Login aufgefordert\\
    \\
    Scenario S11: \> \> \>  Startbildschirm, wenn angemeldet\\ \\
    Given: \> \> \>   Mobile Client is geöffnet\\
    When: \> \> \>  Benutzer ist angemeldet\\
    Then: \> \> \>  Konfiguration, die der Benutzer zuletzt gewählt hat, wird angezeigt\\
    And: \> \> \>    Benachrichtigungs-Buttons gemäss Konfiguration werden angezeigt.\\
    \\
    Scenario S12: \> \> \>  Anmelden korrekt\\ \\
    Given: \> \> \>  Benutzer ist nicht angemeldet\\
    And: \> \> \>    Login Screen wird angezeigt\\
    And: \> \> \>     Für den Benutzer sind gültige Konfigurationen erfasst\\
    When: \> \> \>   Benutzer meldet sich mit korrekten Daten an\\
    Then: \> \> \>   Benutzer wird auf nächste Seite geleitet und kann dort die Konfiguration auswählen, die er Benutzen möchte.\\
    \\
    Scenario S13: \> \> \>  Anmelden falsch\\ \\
    Given: \> \> \>  Benutzer ist nicht angemeldet\\
    And: \> \> \>    Login Screen wird angezeigt\\
    When: \> \> \>   Benutzer meldet sich mit falschen Daten an\\
    Then: \> \> \>   Fehlermeldung\\
    And: \> \> \>  Benutzer wird nicht weitergeleitet\\
    \\
    Scenario S14: \> \> \>  Konfiguration Wählen\\ \\
    Given: \> \> \>  Benutzer hat sich korrekt angemeldet\\
    And: \> \> \>    Konfiguration Auswählen Screen wird angezeigt\\
    When: \> \> \>   Der Benutzer wählt die gewünschte Konfiguration\\
    Then: \> \> \>   Der Benutzer wird weitergeleitet\\
    And: \> \> \>    Die gewählte Konfiguration wird geladen\\
    And: \> \> \>    Benachrichtigungs Buttons gemäss Konfiguration werden angezeigt.\\
    \\
    Scenario S15: \>  \> \> Logout\\ \\
    Given: \> \> \>  Benutzer ist angemeldet\\
    When: \> \> \>   Benutzer klickt logout\\
    Then: \> \> \>  Benutzer wird zur Login Seite weitergeleitet\\
    \\
    Scenario S16: \> \> \>  Login\\ \\
    Given: \> \> \>  Benutzer ist nicht angemeldet\\
    And: \> \> \>    Admin UI Login Screen wird angezeigt\\
    When: \> \> \>   Admin meldet sich mit korrekten Daten an\\
    Then: \> \> \>   Admin wird auf Übersichtsseite weitergeleitet\\
    \\
    Scenario S17: \> \> \>  Anmelden falsch\\ \\
    Given: \> \> \>  Benutzer ist nicht angemeldet\\
    And: \> \> \>    Admin UI Login Screen wird angezeigt\\
    When: \> \> \>   Admin meldet sich mit falschen Daten an\\
    Then: \> \> \>   Fehlermeldung wird angezeigt\\
    And: \> \> \>   Admin wird nicht weitergeleitet.\\
    \\
    Scenario S18: \> \> \>  Konfiguration verwalten \\ \\
    Given: \> \> \>  Admin ist angemeldet\\
    When: \> \> \>  Admin UI wird aufgerufen\\
    Then: \> \> \>  Alle existierenden Konfigurationen werden angezeigt\\
    And: \> \> \>  Neue Konfigurationen können erstellt werden\\
    And: \> \> \>  Bestehende Konfigurationen können verändert werden\\
    And: \> \> \>  Bestehende Konfigurationen können gelöscht werden\\
\end{tabbing}

\subsubsection*{F02 - Sprachsynthese}
\begin{tabbing}
    Left \= Middle \= Right \= Right \kill
    Scenario S19: \> \> \> Benachrichtigung vorlesen - Sprachsynthese aktiviert und Benachrichtigung relevant \\ \\
    Given:  \> \> \> ...\\
    When:   \> \> \> ...\\
    Then:   \> \> \> ...\\
    \\
    Scenario S20: \> \> \> Benachrichtigung nicht vorlesen - Sprachsynthese aktiviert und Benachrichtigung nicht relevant \\ \\
    Given:  \> \> \> ...\\
    When:   \> \> \> ...\\
    Then:   \> \> \> ...\\
    \\
    Scenario S21: \> \> \> Lokale Einstellung - Sprachsynthese deaktiviert und Benachrichtigung relevant \\ \\
    Given:  \> \> \> ...\\
    When:   \> \> \> ...\\
    Then:   \> \> \> ...\\
    \\
    Scenario S22: \> \> \> Lokale Einstellung - Sprachsynthese deaktiviert und Benachrichtigung nicht relevant \\ \\
    Given:  \> \> \> ...\\
    When:   \> \> \> ...\\
    Then:   \> \> \> ...\\
    \\
    Scenario S23: \> \> \> Benachrichtigung verwalten - Relevanz Sprachsynthese kann im Admin UI aktiviert / deaktiviert werden \\ \\
    Given:  \> \> \> ...\\
    When:   \> \> \> ...\\
    Then:   \> \> \> ...\\
    \\
    Scenario S24: \> \> \> Benachrichtigung empfangen - Änderung an Typ in Admin UI wird sofort angewendet \\ \\
    Given:  \> \> \> ...\\
    When:   \> \> \> ...\\
    Then:   \> \> \> ...\\
    \\
\end{tabbing}
\subsubsection*{F03 - Gegensprechanlage}
\begin{tabbing}
    Left \= Middle \= Right \= Right \kill
    Scenario S25: \> \> \> Gegensprechanlage Buttons nach Anmeldung anzeigen \\ \\
    Given:  \> \> \> ...\\
    When:   \> \> \> ...\\
    Then:   \> \> \> ...\\
    \\
    Scenario S26: \> \> \> Verbindungsaufbau 1:1 - Gegenüber ist Verfügbar \\ \\
    Given:  \> \> \> ...\\
    When:   \> \> \> ...\\
    Then:   \> \> \> ...\\
    \\
    Scenario S27: \> \> \> Verbindungsaufbau 1:1 - Gegenüber ist nicht Verfügbar \\ \\
    Given:  \> \> \> ...\\
    When:   \> \> \> ...\\
    Then:   \> \> \> ...\\
    \\
    Scenario S28: \> \> \> Verbindungsaufbau 1:1 - Gegenüber hat Gegensprechanlage deaktivier \\ \\
    Given:  \> \> \> ...\\
    When:   \> \> \> ...\\
    Then:   \> \> \> ...\\
    \\
    Scenario S29: \> \> \> Verbindungsaufbau 1:1 - Benachrichtigungston \\ \\
    Given:  \> \> \> ...\\
    When:   \> \> \> ...\\
    Then:   \> \> \> ...\\
    \\
    Scenario S30: \> \> \> Verbindungsaufbau 1:1 - Automatische Annahme \\ \\
    Given:  \> \> \> ...\\
    When:   \> \> \> ...\\
    Then:   \> \> \> ...\\
    \\
    Scenario S31: \> \> \> Unterhaltung 1:1 - Unterhaltung in Echtzeit möglich \\ \\
    Given:  \> \> \> ...\\
    When:   \> \> \> ...\\
    Then:   \> \> \> ...\\
    \\
    Scenario S32: \> \> \> Verbindungsaufbau 1:n - Alle Gegenüber sind Verfügbar \\ \\
    Given:  \> \> \> ...\\
    When:   \> \> \> ...\\
    Then:   \> \> \> ...\\
    \\
    Scenario S33: \> \> \> Verbindungsaufbau 1:n - 1 von n Gegenüber ist nicht Verfügbar \\ \\
    Given:  \> \> \> ...\\
    When:   \> \> \> ...\\
    Then:   \> \> \> ...\\
    \\
    Scenario S34: \> \> \> Verbindungsaufbau 1:n - 1 von n Gegenüber hat Gegensprechanlage deaktiviert \\ \\
    Given:  \> \> \> ...\\
    When:   \> \> \> ...\\
    Then:   \> \> \> ...\\
    \\
    Scenario S35: \> \> \> Verbindungsaufbau 1:n - 1 von n Benachrichtigungston \\ \\
    Given:  \> \> \> ...\\
    When:   \> \> \> ...\\
    Then:   \> \> \> ...\\
    \\
    Scenario S36: \> \> \> Verbindungsaufbau 1:n - Automatische Annahme in allen Empfängern \\ \\
    Given:  \> \> \> ...\\
    When:   \> \> \> ...\\
    Then:   \> \> \> ...\\
    \\
    Scenario S37: \> \> \> Unterhaltung 1:n - Unterhaltung in Echtzeit möglich \\ \\
    Given:  \> \> \> ...\\
    When:   \> \> \> ...\\
    Then:   \> \> \> ...\\
    \\
    Scenario S38: \> \> \> Inbox - Vergangene Sprachverbindungen \\ \\
    Given:  \> \> \> ...\\
    When:   \> \> \> ...\\
    Then:   \> \> \> ...\\
    \\
    Scenario S39: \> \> \> Inbox - Verpasste Sprachverbindungen \\ \\
    Given:  \> \> \> ...\\
    When:   \> \> \> ...\\
    Then:   \> \> \> ...\\
    \\
    Scenario S40: \> \> \> Inbox - Abgelehnte Unterhaltungen \\ \\
    Given:  \> \> \> ...\\
    When:   \> \> \> ...\\
    Then:   \> \> \> ...\\
    \\
    Scenario S41: \> \> \> Verbindung trennen 1:1 - Durch Empfänger \\ \\
    Given:  \> \> \> ...\\
    When:   \> \> \> ...\\
    Then:   \> \> \> ...\\
    \\
    Scenario S42: \> \> \> Verbindung trennen 1:1 - Durch Initiator \\ \\
    Given:  \> \> \> ...\\
    When:   \> \> \> ...\\
    Then:   \> \> \> ...\\
    \\
    Scenario S43: \> \> \> Verbindung trennen 1:n - Durch Empfänger \\ \\
    Given:  \> \> \> ...\\
    When:   \> \> \> ...\\
    Then:   \> \> \> ...\\
    \\
    Scenario S44: \> \> \> Verbindung trennen 1:1 - Durch Initiator \\ \\
    Given:  \> \> \> ...\\
    When:   \> \> \> ...\\
    Then:   \> \> \> ...\\
    \\
    Scenario 45: \> \> \> Konfiguration über Admin UI \\ \\
    Given:  \> \> \> ...\\
    When:   \> \> \> ...\\
    Then:   \> \> \> ...\\
    \\
\end{tabbing}
\clearpage
