\section{Features und Testszenarien}

\subsubsection*{F01 - Migration des Mobile Clients}

Die Szenarien S01 bis S18 Anhang E des Berichts des Vorgängerprojekts dokumentiert.\cite{ip5}
Diese Szenarien werden für dieses Projekt übernommen und dienen als Testszenarien für F01 Migration des Mobile Clients.

\subsubsection*{F02 - Sprachsynthese}

Im folgenden sind die Scenarien S19 bis S24 definiert.
Diese dienen zur Verifizierung der Umsetzung der Sprachsynthese für Benachrichtigungen.

\begin{tabbing}
    Left \= Middle \= Right \= Right \kill
    Scenario S19: \> \> \> Benachrichtigung vorlesen - Sprachsynthese aktiviert und Benachrichtigung relevant \\ \\
    Given:  \> \> \> Sprachsynthese für eine Benachrichtigung X ist aktiviert \\
    And:  \> \> \> Sprachsynthese in den App Einstellungen ist aktiviert \\
    When:   \> \> \> Eine Benachrichtigung X wird empfangen \\
    Then:   \> \> \> Der Inhalt der Benachrichtigung wird vorgelesen. \\
    \\
    Scenario S20: \> \> \> Benachrichtigung nicht vorlesen - Sprachsynthese aktiviert und Benachrichtigung nicht \\
    \> \> \> relevant \\ \\
    Given:  \> \> \> Sprachsynthese für eine Benachrichtigung X ist deaktiviert \\
    And:  \> \> \> Sprachsynthese in den App Einstellungen ist aktiviert \\
    When:   \> \> \> Eine Benachrichtigung X wird empfangen \\
    Then:   \> \> \> Der Inhalt der Benachrichtigung wird nicht vorgelesen. \\
    \\
    Scenario S21: \> \> \> Lokale Einstellung - Sprachsynthese deaktiviert und Benachrichtigung relevant \\ \\
    Given:  \> \> \> Sprachsynthese für eine Benachrichtigung X ist aktiviert \\
    And:  \> \> \> Sprachsynthese in den App Einstellungen ist deaktiviert \\
    When:   \> \> \> Eine Benachrichtigung X wird empfangen \\
    Then:   \> \> \> Der Inhalt der Benachrichtigung wird nicht vorgelesen. \\
    \\
    Scenario S22: \> \> \> Lokale Einstellung - Sprachsynthese deaktiviert und Benachrichtigung nicht relevant \\ \\
    Given:  \> \> \> Sprachsynthese für eine Benachrichtigung X ist deaktiviert \\
    And:  \> \> \> Sprachsynthese in den App Einstellungen ist deaktiviert \\
    When:   \> \> \> Eine Benachrichtigung X wird empfangen \\
    Then:   \> \> \> Der Inhalt der Benachrichtigung wird vorgelesen. \\
    \\
    Scenario S23: \> \> \> Benachrichtigung verwalten - Relevanz Sprachsynthese kann im Admin UI aktiviert \\
    \> \> \>  / deaktiviert werden \\ \\
    Given:  \> \> \> Sprachsynthese für eine Benachrichtigung X ist aktiviert/deaktiviert \\
    When:   \> \> \> Sprachsynthese für die Benachrichtigung X wird über das Admin UI aktiviert/deaktiviert. \\
    Then:   \> \> \> Konfiguration von Benachrichtigung X wird bei Empfang korrekt angewendet. \\
    \clearpage
\end{tabbing}
\begin{tabbing}
    Left \= Middle \= Right \= Right \kill
    Scenario S24: \> \> \> Benachrichtigung empfangen - Änderung an Typ in Admin UI wird sofort angewendet \\ \\
    Given:  \> \> \> Sprachsynthese für eine Benachrichtigung X ist aktiviert \\
    When:   \> \> \> Inhalt der Benachrichtigung wird im Admin UI angepasst\\
    Then:   \> \> \> Bei Empfang der Benachrichtigung, wird der angepasste Inhalt der Benachrichtigung  \\
    \> \> \> vorgelesen.\\
\end{tabbing}
\subsubsection*{F03 - Gegensprechanlage}
\begin{tabbing}
    Left \= Middle \= Right \= Right \kill
    Scenario S25: \> \> \> Gegensprechanlage Buttons nach Anmeldung anzeigen \\ \\
    Given:  \> \> \>  Konfiguration X ist im Clouservice erfasst\\
    And:  \> \> \>  Für Konfiguration X wurden Buttons für Sprachverbindungen definiert.  \\
    When:   \> \> \> Benutzer meldet sich an und wählt Konfiguration X\\
    Then:   \> \> \> Die konfigurierten Buttons werden geladen und angezeigt\\
    \\
    Scenario S26: \> \> \> Verbindungsaufbau - Gegenüber ist Verfügbar \\ \\
    Given:  \> \> \> Ziel der Sprachverbindung ist beim Signaling Server registriert. \\
    And:  \> \> \> Ziel der Sprachverbindung hat das Empfangen von Anrufen in der lokalen Konfiguration \\
    \> \> \> aktiviert. \\
    When:   \> \> \> Button für Sprachverbindung wird aktiviert \\
    Then:   \> \> \> Die Sprachverbindung wird aufgebaut und die Ansicht für Aktive Anrufe auf beiden \\
    \> \> \> Clients angezeigt.\\
    \\
    Scenario S27: \> \> \> Verbindungsaufbau - Gegenüber ist nicht Verfügbar \\ \\
    Given:  \> \> \> Ziel der Sprachverbindung ist nicht beim Signaling Server registriert\\
    When:   \> \> \> Button für Sprachverbindung wird aktiviert\\
    Then:   \> \> \> Ziel der Sprachverbindung erhält eine Benachrichtigung für verpassten Anruf\\
    Then:   \> \> \> Ziel der Sprachverbindung sieht einen Eintrag für den verpassten Anruf in der Inbox.\\
    Then:   \> \> \> Dem Initiator der Verbindung wird angezeigt, dass das Ziel nicht erreicht wurde. \\
    \\
    Scenario S28: \> \> \> Verbindungsaufbau - Gegenüber hat Gegensprechanlage deaktiviert \\ \\
    Given:  \> \> \> Ziel der Sprachverbindung ist beim Signaling Server registriert\\
    When:   \> \> \> Button für Sprachverbindung wird aktiviert\\
    Then:   \> \> \> Ziel der Sprachverbindung erhält eine Benachrichtigung für verpassten Anruf\\
    And:   \> \> \> Ziel der Sprachverbindung sieht einen Eintrag für den verpassten Anruf in der Inbox.\\
    And:   \> \> \> Dem Initiator der Verbindung wird angezeigt, dass das Ziel nicht erreicht wurde. \\
    \\
    Scenario S29: \> \> \> Verbindungsaufbau - Benachrichtigungston \\ \\
    Given:  \> \> \> Empfang von Sprachverbindungen ist aktiviert\\
    When:   \> \> \> Eine Sprachverbindung wird empfangen\\
    Then:   \> \> \> Ein Benachrichtigungston ertönt, bevor die Verbindung angenommen wird.\\
    \\
    Scenario S30: \> \> \> Verbindungsaufbau - Automatische Annahme \\ \\
    Given:  \> \> \> Empfang von Sprachverbindungen ist aktiiert\\
    When:   \> \> \> Eine Sprachverbindung wird empfangen.\\
    Then:   \> \> \> Die Verbindung wird automatisch und ohne weitere Benutzereingaben geöffnet. \\
    \\
    Scenario S31: \> \> \> Unterhaltung 1:1 - Unterhaltung in Echtzeit möglich \\ \\
    Given:  \> \> \> Sprachverbindung zwischen zwei Teilnehmern wurde initialisiert\\
    When:   \> \> \> Die Sprachverbindung aufgebaut ist. \\
    Then:   \> \> \> Können beide Teilnehmer in Echtzeit kommunizieren.\\
    \\
    Scenario S32: \> \> \> Unterhaltung 1:n - Unterhaltung in Echtzeit möglich \\ \\
    Given:  \> \> \> Ale Ziele einer Sprachverbindung sind beim Signaling Service registriert\\
    And:  \> \> \> Ale Ziele einer Sprachverbindung haben den Empfang von Anrufen aktiviert.\\
    When:   \> \> \> Eine Sprachverbindung zu mehreren Teilnehmern aufgebaut wird\\
    Then:   \> \> \> Können der Initiator mit allen Teilnehmern in Echtzeit kommunizieren.\\
    Then:   \> \> \> Kenn jedes Ziel mit dem Initiator Echtzeit kommunizieren.\\
    Then:   \> \> \> Können die Ziele nicht direkt kommunizieren.\\
    \\
    Scenario S33: \> \> \> Verbindungsaufbau 1:n  \\ \\
    Given:  \> \> \> Ale Ziele einer Sprachverbindung sind beim Signaling Service registriert\\
    And:  \> \> \> Ale Ziele einer Sprachverbindung haben den Empfang von Anrufen aktiviert.\\
    When:   \> \> \> Eine Sprachverbindung zu mehreren Teilnehmern aufgebaut wird\\
    Then:   \> \> \> Der Initiator sieht die Verbindung und den Status aller Ziele \\
    Then:   \> \> \> Jedes Ziel verhält sich als ob es Teil einer Einzelunterhaltung ist (S26-S30).\\
    \\
    Scenario S34: \> \> \> Inbox - Vergangene Sprachverbindungen \\ \\
    Given:  \> \> \> Eine Sprachverbindung wurde empfangen. \\
    When:   \> \> \> Die Sprachverbindung wird beendet.  \\
    Then:   \> \> \> Es ist ein Eintrag für die Verbindung in der Inbox zu sehen. \\
    \\
    Scenario S35: \> \> \> Inbox - Verpasste Sprachverbindungen \\ \\
    Given:  \> \> \> Das Ziel einer Sprachverbindung ist nicht erreichbar. \\
    When:   \> \> \> Es wird versucht eine Sprachverbindung zum Ziel aufzubauen.\\
    Then:   \> \> \> Es ist ein Eintrag für die verpasste Verbindung in der Inbox zu sehen.\\
    \\
    Scenario S36: \> \> \> Inbox - Abgelehnte Unterhaltungen \\ \\
    Given:  \> \> \> Das Ziel einer Sprachverbindung ist nicht erreichbar. \\
    And:  \> \> \> Das Ziel einer hat den Empfang von Anrufen lokal deaktiviert. \\
    When:   \> \> \> Es wird versucht eine Sprachverbindung zum Ziel aufzubauen.\\
    Then:   \> \> \> Es ist ein Eintrag für die verpasste Verbindung in der Inbox zu sehen.\\
    \\
\end{tabbing}
\begin{tabbing}
    Left \= Middle \= Right \= Right \kill
    Scenario S37: \> \> \> Verbindung trennen durch Empfänger \\ \\
    Given:  \> \> \> Eine Sprachverbindung zwischen beliebig vielen Teilnehmern wurde aufgebaut. \\
    When:   \> \> \> Das Ziel der Verbindung beendet die Verbindung. \\
    Then:   \> \> \> Die Verbindung wird auf beiden Seiten beendet.\\
    \\
    Scenario S38: \> \> \> Verbindung durch Initiator \\ \\
    Given:  \> \> \> Eine Sprachverbindung zwischen zwei Teilnehmern wurde aufgebaut. \\
    When:   \> \> \> Der Initiator der Verbindung beendet die Verbindung. \\
    Then:   \> \> \> Die Verbindung wird auf beiden Seiten beendet.\\
    \\
    Scenario S39: \> \> \> Austreten aus Gruppenunterhaltung \\ \\
    Given:  \> \> \> Eine Sprachverbindung zwischen mehr als zwei Teilnehmern wurde aufgebaut. \\
    When:   \> \> \> Eines der Ziele beendet die Verbindung. \\
    Then:   \> \> \> Die Verbindung wird für dieses Ziel beendet.\\
    And:   \> \> \> Die Verbindungen aller weiteren Teilnehmer bleiben offen.\\
    \\
    Scenario S40: \> \> \> Konfiguration über Admin UI \\ \\
    Given: \> \> \>  Admin ist angemeldet\\
    When: \> \> \>  Admin UI wird aufgerufen\\
    Then: \> \> \>  Alle konfigurierten Buttons für Sprachverbindungen werden angezeigt.\\
    And: \> \> \>  Neue Buttons für Sprachverbindungen können erstellt werden\\
    And: \> \> \>  Bestehende Buttons für Sprachverbindungen können verändert werden\\
    And: \> \> \>  Bestehende Buttons für Sprachverbindungen können gelöscht werden\\
    \\
\end{tabbing}
\clearpage
