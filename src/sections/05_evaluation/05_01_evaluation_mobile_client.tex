
\subsection{Native iOS Applikation}

Mit dem Projekt ''IP5 Cloudbasiertes Praxisrufsystem'' wurde eine mobile Applikation entwickelt, die als Endpunkt in einem Praxisrufsystem verwendet werden kann.
Diese unterstützt das Senden und Empfangen vorkonfigurierter Benachrichtungen.
Die Applikation wurde auf Basis des Frameworks Nativescript als Multi-Platform Applikation entwickelt~\cite{ip5}.
Dadurch ist es möglich, dieselbe Codebasis für die Entwicklung von Android und iOS Applikationen zu verwenden~\cite{nativescript}.
Im Fazit des Vorgängerprojekts wurde empfohlen, diese Applikation durch native Applikationen pro Platform zu ersetzten.
Dadurch könne effiziente Weiterentwicklung, sowie Hardware- und Betriebssystemkompatibilität langfristig gewährleistet werden~\cite{ip5}.

Mit diesem Projekt wird die Applikation deshalb neu als native iOS Applikation umgesetzt.
Dabei ist es wichtig, dass sämtliche bestehende Funktionalität auch im neu entwickelten nativen Mobile Client zur Verfügung steht.
Um weiterhin Benachrichtigungen empfangen zu können, muss die gewählte Technologie es ermöglichen, Firebase Cloud Messaging anzubinden.
Um Benachrichtigungen versenden und Konfigurationen laden zu können, muss es möglich sein, die Applikation an die API des Cloudservices anzubinden.
Weiter muss die Technologie es ermöglichen, regelmässige Aufgaben auszuführen.
Dadurch kann überprüft werden, ob der Benutzer ungelesene Benachrichtigungen hat und wenn nötig eine Erinnerung dafür angezeigt werden.

\subsubsection{Programmiersprache}

Für die Entwicklung von nativen iOS Applikationen ist die Programmiersprache Swift Industriestandard~\cite{ios_swift}.
Der native iOS Client für Praxisruf wird deshalb mit Swift implementiert.

\subsubsection{Frameworks}

Für die Umsetzung von iOS Applikationen stellt Apple die zwei Frameworks UIKit~\cite{ios_uikit} und SwiftUI~\cite{ios_swift_ui} zur Verfügung.
UIKit ist das ältere der beiden Frameworks und ist seit iOS 2.0 verfügbar.
Dementsprechend ist das Framework ausgereift und bietet viele Funktionen zur Integration einer Applikation mit iOS~\cite{ios_uikit}.

SwiftUI ist deutlich neuer und steht seit iOS 13.0 zur Verfügung~\cite{ios_swift_ui}.
Apple wirbt auf der offiziellen Dokumentationsseite für SwiftUI und schreibt: ''SwiftUI helps you build great-looking apps across all Apple platforms with the power of Swift — and as little code as possible.''~\cite{ios_swift_ui}.
SwiftUI fokussiert sich auf eine deklarative Syntax und ist dadurch leichtgewichtiger als UIKit.
Es bietet zudem ausgezeichnete Integration in die XCode Entwicklungsumgebung und viele Standardkomponenten wie Listenansichten, Formfelder und andere UIKomponenten~\cite{ios_swift_ui}.
Dadurch wird es einfacher eine Benutzeroberfläche mit nativem Look und Feel einer iOS Applikation umzusetzten.

Da SwiftUI deutlich neuer ist als UIKit, ist es möglich das es noch nicht alle Funktionen und Betriebssystem Integrationen unterstützt die in UIKit möglich sind.
Dieses Problem wird dadurch relativiert, dass UIKit Funktionen nahtlos in SwiftUI integriert werden können~\cite{ios_swift_ui_uikit}.
Es ist also grundsätzlich möglich alles, was mit UIKit umgesetzt werden kann, auch mit SwiftUI umzusetzen.

\clearpage

\subsubsection{Anbindung Firebase Cloud Messaging}

Die neue iOS Applikation, muss es weiterhin ermöglichen Benachrichtigungen über Firebase Cloud Messaging zu empfangen.
Firebase stellt dazu eine native iOS Bibliothek zur Verfügung~\cite{firebase_github_ios}.
Die Integration von Firebase Cloud Messaging kann mit dieser Library implementiert werden.
Dies beinhaltet die Registrierung bei Firebase Cloud Messaging, sowie das Empfangen der Benachrichtigungen über Firebase~\cite{firebase_ios}.
Damit Push-Benachrichtigungen über das Betriebssystem angezeigt werden können, müssen empfangene Benachrichtigungen an das Betriebssystem übergeben werden.
Mit sogenannten AppDelegates ist es möglich, mit dem iOS-Betriebssystems zu interagieren.
Dies beinhaltet das Anzeigen von Vorder- und Hintergrundbenachrichtigungen über das Benachrichtigungszenter von iOS anzuzeigen~\cite{ios_app_delegate}.

Die Firebase Cloud Messaging Bibliothek kann sowohl mit SwiftUI als auch mit UIKit verwendet werden.
Die Integration muss dabei aber auf jeden Fall über einen AppDelegate umgesetzt werden~\cite{firebase_ios}.
AppDelegates sind ein Konzept welches aus UIKit stammt und in reinen SwiftUI Applikationen nicht verwendet werden~\cite{ios_app_delegate}.
UIKit Funktionen können allerdings nahtlos mit SwiftUI integriert werden.
So ist es auch möglich AppDelegates in SwiftUI Anwendungen zu verwenden~\cite{ios_swift_ui_uikit}.
Die Firebase Cloud Messaging Bibliothek für iOS ermöglicht es damit, Benachrichtigungen von Praxisruf sowohl mit UIKit als auch mit SwiftUI umzusetzen.

\subsubsection{Anbindung Cloudservice API}

Die iOS Applikation muss die API des Cloudservice ansprechen können, um Konfigurationen zu laden und Benachrichtigungen zu versenden.
Die bestehende API des Cloudservice ist als HTTP-Schnittstelle umgesetzt~\cite{ip5}.
Die Integration von Http Anfragen in nativen iOS Applikationen ist über die URLRequest-Komponente aus der iOS Standardbibliothek möglich.
Diese erlaubt das Erstellen von beliebigen Http-Requests~\cite{ios_urlrequest}.
Dies bedeutet, dass die Anbindung an die API des Cloudservice kann mit Mitteln der iOS Standardbibliothek umgesetzt werden.

\subsubsection{Benachrichtigungen prüfen}

Die mobile Applikation aus dem Vorgängerprojekt sammelt empfangene Benachrichtigungen in einer Inbox.
Praxismitarbeitende müssen Benachrichtigungen in dieser Inbox als gelesen markieren.
Die App prüft regelmässig, ob ungelesene Benachrichtigungen vorhanden sind.
Ist dies der Fall, wird ein Benachrichtigungston zur Erinnerung abgespielt.
Diese Funktion muss auch von der neuen iOS Applikation unterstützt werden.

Die iOS Standardbibliothek bietet Mittel, mit welchen regelmässige Aufgaben angestossen werden können.
Einerseits können über eine Timer-Komponente in regelmässigen abständen Events veröffentlicht werden.
Auf einer SwiftUI Ansicht (View) können Komponenten (Listener) registriert werden, die beim Empfang eines Timer-Events aufgerufen werden.
Dies bringt die Einschränkung mit sich, dass die Prüfung nur ausgeführt ist, wenn die App im Vordergrund läuft und die View geladen wurde~\cite{ios_timer}.
Da der bestehende Mobile Client dieselbe Einschränkung hat, kann mit einem Timer dasselbe Verhalten, wie in der bestehenden Applikation, umgesetzt werden.
Um die Prüfung auch im Hintergrund auszuführen, kann die Komponente BGTaskScheduler verwendet werden.
Über diese können Aufgaben erfasst werden, die im Hintergrund ausgeführt werden~\cite{ios_bgtaskscheduler}.

Die Erinnerungsfunktion kann mit Mitteln aus der iOS Standardbibliothek umgesetzt werden.
Dabei ist es möglich die Prüfung analog zur bestehenden Applikation umzusetzen und auszuführen, solange die Applikation angezeigt wird.
Weiter kann die Prüfung in Zukunft erweitert werden, um auch nach dem Beenden der Applikation an ungelesene Benachrichtigungen zu erinnern.

\clearpage

\subsubsection{Entscheid}

Die native iOS Applikation für Praxisruf wird mit Swift und SwiftUI umgesetzt.
Für die Anbindung an Firebase Cloud Messaging wird die nativen iOS Bibliothek von Firebase umgesetzt.
Die Anbindung an die API des Cloudservice und die Umsetzung der Erinnerungsfunktion werden mit Mitteln der iOS Standardbibliothek umgesetzt.

Die deklarative Syntax von SwiftUI ermöglicht es, einfacher übersichtliche und lesbare Komponenten zu implementieren.
Integration in die XCode Entwicklungsumgebung und verfügbare Standardkomponenten vereinfachen diese Entwicklung weiter.
Es ist möglich, dass einige Funktionen noch nicht mit SwiftUI umgesetzt werden können, weil dafür benötigte Funktionen noch nicht unterstützt werden.
Da UIKit Funktionen nahtlos in SwiftUI integriert werden können, ist es möglich betroffene Teile der Applikation mit UIKit zu implementieren.
Diese Teile können, sobald die entsprechenden Funktionen in SwiftUI verfügbar sind, migriert werden.
Um die Verwendung von UIKit Funktionen auf ein Minimum zu reduzieren wird als Zielplattform iOS15 verwendet.
Damit wird die zur Zeit der Umsetzung neuste iOS Version unterstützt.
Dies ermöglicht es, alle verfügbaren SwiftUI Features zu verwenden und minimiert die Wahrscheinlichkeit, auf UIKIt zurückgreifen zu müssen.

\clearpage
