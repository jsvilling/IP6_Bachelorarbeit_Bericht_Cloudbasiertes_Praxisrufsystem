
\subsection{Mobile Client}

Mit dem Projekt ''IP5 Cloudbasiertes Praxisrufsystem'' eine mobile Applikation entwickelt, die als Endpunkt in einem Praxisrufsystem verwendet werden kann.
Diese unterstützt das Senden und Empfangen vorkonfigurierter Benachrichtungen~\cite{ip5}.
Die Applikation wurde auf Basis des Frameworks Nativescript~\cite{nativescript} als Multi-Platform Applikation entwickelt.
Dadurch ist es möglich dieselbe Codebasis für die Entwicklung von Android und iOS Applikationen zu verwenden.
Im Fazit des Vorgängerprojekts wurde empfohlen, diese Applikation durch native Applikationen pro Platform zu ersetzten.
Dadurch könne effiziente Weiterentwicklung, sowie Hardware- und Betriebssystemkompatibilität langfristig gewährleistet werden~\cite{ip5}.

Mit diesem Projekt wird die Applikation deshalb neu als native iOS Applikation umgesetzt.
Dabei ist es wichtig, dass sämtliche bestehende Funktionalität auch im neu entwickelten nativen Mobile Client zur Verfügung steht.
Um weiterhin Benachrichtungen senden und empfangen zu können, muss die gewählte Technologie es ermöglichen Firebase Cloud Messaging anzubinden
und Push Benachrichtigungen im Vorder- sowie im Hintergrund empfängen können.
Weiter muss die Technologie es ermöglichen, regelmässige Aufaben auszuführen.
Dadurch kann überprüft werden, ob der Benutzer ungelesene Benachrichtigungen hat und wenn nötig einen Erinnerung dafür angezeigt werden.

\subsubsection{Programmiersprache}

Für die Entwicklung von nativen iOS Applikationen ist die Programmiersprache Swift Industriestandard~\cite{ios_swift}.
Der native iOS Client für Praxisruf wird deshalb mit Swift implementiert.

\subsubsection{Frameworks}

Für die Umsetzung von iOS Applikationen stellt Apple die zwei Frameworks UIKit~\cite{ios_uikit} und SwiftUI~\cite{ios_swift_ui} zur Verfügung.
UIKit ist das ältere der beiden Frameworks und ist seit iOS 2.0 verfügbar.
Dementsprechend ist das Framework ausgereift und bietet viele Funktionen zur Integration einer Applikation mit iOS~\cite{ios_uikit}.

SwiftUI ist deutlich neuer und steht seit iOS 13.0 zur Verfügung~\cite{ios_swift_ui}.
Apple wirbt auf der offiziellen Dokumentationsseite für SwiftUI und schreibt: "SwiftUI helps you build great-looking apps across all Apple platforms with the power of Swift — and as little code as possible."~\cite{ios_swift_ui}
SwiftUI fokussiert sich auf eine declarative Syntax und ist dadurch leichtgewichtiger als UIKit.
Es bietet zudem ausgezeichnete Integration in die XCode Entwicklungsumgebung und viele Standardkomponenten wie Listenansichten, Formfelder und andere UIKomponenten.~\cite{ios_swift_ui}.
Dadurch wird es einfacher eine Benutzeroberfläche mit nativem Look und Feel einer iOS Applikation umzusetzten.

Da SwiftUI deutlich neuer ist als UIKit, ist es möglich das es noch nicht alle Funktionen und Betriebssystem Integrationen unterstützt die in UIKIt möglich sind.
Dieses Problem wird dadurch aufgehoben, dass UIKit Funktionen nahtlos in SwiftUI integriert werden können.\cite{ios_swift_ui_uikit}
Es ist also grundsätzlich möglich, alles was mit UIKit umgesetzt werden kann auch mit SwiftUI umuzusetzen.

\clearpage

\subsubsection{Unterstützung Benachrichtigungen}

Die neue iOS Applikation, muss es weiterhin ermöglichen Benachrichtigungen über Firebase Cloud Messaging zu empfangen und versenden.
Firebase stellt dazu eine native iOS Library zur Verfügung~\cite{firebase_github_ios}.
Die Integration von Firebase Cloud Messaging kann mit dieser Library implementiert werden.
Dies beinhaltet die Registrierung bei Firebase Cloud Messaging, sowie das Empfangen der Benachrichtigungen über Firebase~\cite{firebase_ios}.
Damit Push-Benachrichtigungen über das Betriebssystem angezeigt werden können, müssen empfangene Benachrichtigungen an das Betriebssystem übergeben werden.
Mit sogenannten AppDelegates ist es möglich sich in den Lifecycle des Betriebssystems einzuhängen~\cite{ios_app_delegate}.
Dadurch ist es auch möglich, Vorder- und Hintergrundbenachrichtigungen über das Benachrichtigungszenter von iOS anzuzeigen~\cite{firebase_ios}.

Die Firebase Cloud Messaging Library kann sowohl mit SwiftUI als auch mit UIKit verwendet werden.
AppDelegates sind ein Konzept welches aus UIKit stammt\cite{ios_app_delegate}.
SwiftUI Applikationen können ohne AppDelegates implementiert werden.
UIKit Funktionen können allerdings nahtlos mit SwiftUI integriert werden.\cite{ios_swift_ui_uikit}
Die Firebase Cloud Messaging Library für iOS ermöglicht es also, Benachrichtigungen von Praxisruf sowohl mit UIKit als auch mit SwiftUI umzusetzten.

Um Konfigurationen zu laden und Benachrichtigungen zu versenden, muss die REST Api des Cloudservice angesprochen werden können.
Dies ist über die URLRequest-Komponente aus der iOS Standardbibliothek möglich~\cite{ios_urlrequest}.

\subsubsection{Benachrichtigungen prüfen}

Die mobile Applikation aus dem Vorgängerprojekt erinnert in regelmässigen Abständen, wenn ungelesene Benachrichtigungen vorhanden sind.
Diese Funktion muss auch von der neuen iOS Applikation unterstützt werden.
Dazu muss regelmässig geprüft werden, ob es ungelesene Benachrichtigungen gibt und gegebenenfalls ein Erinnerungston abgespielt werden.
Die standard iOS Bibliothek bietet Mittel, mit welchen regelmässige Aufgaben angestossen werden können.
Einerseits können über eine ''Timer''-Komponente in Regelmässigen abständen Events veröffentlicht werden.
Auf einer SwiftUI View kann ein beliebiger Listener registriert werden, der beim Empfang eines Events des Timers aufgerufen wird.
Dies bringt die Einschränkung mit sich, dass die Prüfung nur ausgeführt ist, wenn die App im Vordergrund läuft und die View geladen wurde~\cite{ios_timer}.
Da der bestehende Mobile Client dieselbe Einschränkung hat, könnte mit einem Timer trotzdem genau dasselbe Verhalten wie in der bestehenden Applikation umgesetzt werden.
Die Standardbibliothek bietet allerdings auch Mittel, um Aufgaben im Hintergrund zu verarbeiten.
Über die Klasse BGTaskScheduler können Aufgaben erfasst werden, die im Hintergrund ausgeführt werden~\cite{ios_bgtaskscheduler}.

Die Erinnerungsfunktion kann mit Mitteln aus der iOS Standardbibliothek umgesetzt werden.

\subsubsection{Entscheid}

Der native iOS Mobile Client für Praxisruf wird mit Swift und SwiftUI umgesetzt.
Der deklarative Syntax, ermöglicht es einfacher übersichtliche und lesbare Komponenten zu implementieren.
Ausgezeichnete Integration in die XCode Entwicklungsumgebung und die zur Verfügung gestellten Standard Komponenten reduzieren den Entwicklungsaufwand weiter.
Die Neuheit von SwiftUI bedeutet, dass möglicherweise gewisse Funktionen noch nicht mit purem SwiftUI umgesetzt werden können.
Um diese Wahrscheinlichkeit zu minimieren, wird iOS15 als Zielplattform gewählt.
Mit der Verwendung der neusten iOS Version als Zielplattform kann auch die neuste Version von SwiftUI und damit alle zur Verfügung stehenden Funktionen aus SwiftUI verwendet werden.
Sollten trotzdem Funktionen benötigt werden, die SwiftUI nicht unterstützt, können diese dank der nahtlosen Integration mit UIKit implementiert und in Zukunft migriert werden.

\clearpage
