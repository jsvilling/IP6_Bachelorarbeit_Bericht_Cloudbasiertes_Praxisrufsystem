\subsection{Gegensprechanlage}

Die zentrale Aufgabenstellung dieser Arbeit dreht sich darum, Peer-To-Peer Sprachübertragung in ein cloudbasiertes Praxisrufsystem zu integrieren.
In diesem Kapitel wird entschieden, welche Technologie zur Umsetzung dieser Aufgabe verwendet wird.
Es werden dabei zwei Ansätze beschrieben, mit welchen diese Anforderung umgesetzt werden kann.
Der erste Ansatz beinhaltet die Anbindung eines externen Anbieters für Kommunikationslösungen.
Der zweite Ansatz verzichtet auf die Integration eines Anbieters und setzt Sprachverbindungen als Peer-To-Peer Verbindungen basierend auf WebRTC um.

\subsubsection{Amazon Chime}

Eine Möglichkeit um Sprachverbindungen in Praxisruf zu implementieren, ist die Integration einer bestehenden Business-Kommunikationslösung wie Webex oder Microsoft Teams.
Auch für die Integration von Sprachverbindungen gilt, dass für Infrastruktur und Dienstleistungen wo möglich Amazon Webservices verwendet wird\footnote{Siehe Anforderung U19 in Kapitel 3.1.3}.
Mit Amazon Chime bietet Amazon einen Dienst für Telefonie, Chats und Videokonferenzen~\cite{aws_chime}.
Chime bietet einerseits Web- und Mobile Applikationen für Anrufe und Meetings.
Andererseits ist es mit dem Amazon Chime SDK für iOS möglich, Chime in eigene native iOS Applikationen zu integrieren~\cite{aws_chime_sdk}.
Chime ermöglicht es Konferenzunterhaltungen mit bis zu 250 Teilnehmern zu starten~\cite{aws_faq}.
Damit bietet der Anbieter alle benötigte Funktionen um Einzel- und Gruppenunterhaltungen in einem Praxisrufsystem umzusetzen.

Integration eines Providers wie Amazon Chime hat den Vorteil, dass die Telefonie in einen etablierten Provider ausgelagert werden kann.
Durch Verträge mit dem Provider können Verfügbarkeitsgarantien und Supportleistungen vereinbart werden.
Dies erhöht die Stabilität des Systems und ermöglicht effizientes Reagieren im Fehlerfall.
Weiter fallen für Praxisruf selbst wenig bis keine Aufwände für den Betrieb der nötigen Infrastruktur.

Amazon Chime unterstützt deutlich mehr Funktionen als in einem Praxisrufsystem benötigt werden.
Ein Rufsystem muss als Gegensprechanlage verwendet werden können und kurze Gespräche zwischen Teilnehmern erlauben.
Dies bedeutet einerseits, dass Chime garantiert alle Funktionen bietet welche benötigt werden.
Es bedeutet aber andererseits auch, dass nur ein kleines Subset der vorhandenen Möglichkeiten ausgenutzt wird.
Ein Nachteil daran ist, dass eine starke Bindung an Amazon als Telefonieanbieter und die Infrastruktur, die Amazon Chime zum Verbindungsaufbau anbietet, stattfindet.
Gleichzeitig kann von vielen Vorteilen von Amazon Chime nicht profitiert werden, weil die entsprechenden Funktionen für ein Praxisrufsystem nicht relevant sind.

\subsubsection{WebRTC}

WebRTC (Web Real-Time Communication) ermöglicht Echtzeitkommunikation basierend auf einem offenen Standard.
Das Open-Source-Projekt wird unter anderem von Apple, Google, Microsoft und Mozilla unterstützt.
Es erlaubt den Austausch von Sprach-, Video- und allgemeinen Daten zwischen Clients.
Wie auf der offiziellen Webseite von WebRTC beschrieben stehen die Technologien
''[\ldots] in allen gängigen Browsern als reguläre JavaScript-APIs verfügbar.
Für native Clients wie Android- und iOS-Anwendungen steht eine Bibliothek zur Verfügung, die dieselben Funktionen bietet.''~\cite{webrtc}.

WebRTC baut zur Kommunikation direkte Peer-To Peer-Verbindungen zwischen den Kommunikationspartnern auf.
Die WebRTC Libraries und APIs bieten Komponenten für die Integration von Peer-To-Peer Verbindung, Hardwarezugriff auf Microphon und Kamera.
WebRTC spezifiziert allerdings nicht, wie für den Verbindungsaufbau notwendige Informationen zwischen Teilnehmern ausgetauscht werden müssen.
Um den Austausch diese Informationen zu ermöglichen ist eine Signaling Instanz notwendig.
Die Signaling Instanz muss es ermöglichen, Informationen zwischen Teilnehmern zu vermitteln.
WebRTC spezifiziert nicht, wie diese Signaling Instanz aussehen muss.
Neben der Signaling Instanz ist keine weitere Infrastruktur notwendig.
Sämtliche Daten werden direkt über die Peer-To-Peer Verbindungen zwischen Clients ausgetauscht.

Dies hat den Vorteil, dass die Abhängigkeit zu externen Umsystemen minimiert werden kann.
Die Signaling Instanz kann auf die eigenen Anforderungen zugeschnitten werden.
Die Technologie über welche Signale ausgetauscht werden, kann frei gewählt werden.
Signaling für Sprachverbindungen im Praxisrufsystem kann damit in den Cloudservice integriert werden.
So kann notwendige Infrastruktur, die als Teil des Praxisrufsystems betrieben werden muss, schlank gehalten werden.
Die Integration der Signalvermittlung in den Cloudservice bietet weiter den Vorteil, dass bestehende Funktionen des Systems angesprochen werden können.
So können z.B. \ Benachrichtigung bei verpassten Anrufen über die bereits umgesetzte Benachrichtigungsfunktion versendet werden.

Die Verwendung von WebRTC hat den Nachteil, dass kein fachlicher Support bei Verbindungsproblemen zur Verfügung steht.
Signaling Instanz und Implementation der Sprachverbindungen stehen in der alleinigen Verantwortung des Betreibers des Praxisrufsystems.
Abgesehen von Fehlern in der Implementation können Verbindungsprobleme an zwei Stellen auftreten.
Einerseits ist es möglich, dass die Signaling Instanz ausfällt und nicht erreichbar ist.
Dieses Problem kann durch den Betrieb des Cloudservice adressiert werden.
Die Signaling Instanz kann in den Cloudservice integriert werden.
Dieser wird wiederum bei einem Cloud-Provider betrieben.
Für diesen Betrieb können Verträge definiert werden, die Verfügbarkeit und Support im Fehlerfall bieten.
Andererseits können lokale Netzwerkprobleme auftreten, welche Endgeräte unerreichbar machen.
Dieses Problem besteht unabhängig davon, wie Sprachübertragung implementiert wird.
Im Fall von Praxisruf kann das Problem dadurch adressiert werden, dass nicht erreichbare Endgeräte durch Benachrichtigungen über verpasste Verbindungen informiert werden.

Die Gegensprechanlage in einem Praxisrufsystem muss Verbindung mit mehreren Teilnehmern gleichzeitig erlauben.
Verbindungen müssen deshalb als zwischen zwei Teilnehmern als 1:1 Verbindungen, aber auch zwischen mehreren Teilnehmern als 1:n Verbindungen umgesetzt werden.
WebRTC sieht ausschliesslich direkte Peer-To-Peer Verbindungen vor.
Es ist allerdings möglich, mehrere direkte Peer-To-Peer Verbindungen gleichzeitig zu verwenden~\cite{webrtc_mesh}.
Weiter ist es möglich, die Kommunikation über eine zentrale Instanz zu bündeln.
Eine solche nennt sich Multipoint Conferencing Unit (MCU).
Bei der Verwendung einer MCU werden Verbindungen nicht direkt zwischen Endgeräten hergestellt.
Stattdessen stellt jedes Endgerät eine Verbindung mit der MCU her.
Die MCU vermittelt die Daten zwischen allen Teilnehmern.
Die Verwendung einer MCU verkompliziert das System und die benötigte Infrastruktur massgeblich~\cite{webrtc_mcu}.

\subsubsection{Entscheidung}

Sprachverbindungen für die Funktion Gegensprechanlage werden mit WebRTC umgesetzt.
Der Cloudservice wird um ein Modul erweitert, welches als Signaling Instanz dient.
Die WebRTC iOS Bibliothek wird verwendet, um Peer-To-Peer Sprachübertragung in einer nativen iOS App zu implementieren.

Durch die Verwendung von WebRTC, können Sprachverbindungen aufgebaut werden, ohne ein weiteres Drittsystem anzubinden.
Die benötigte Infrastruktur kann dadurch auf ein Minimum reduziert werden.
Eine eigene Implementierung der Signaling Instanz ermöglicht es weiter, bestehende Funktionen des Praxisrufsystems effizient zu verwenden.
Die Verfügbaren Libraries für iOs, Android und Javascript bedeuten, dass künftige Web- und Android Clients in das System integriert werden können.
Die Verfügbarkeit der Signaling Instanz wird den Betrieb des Cloudservices sichergestellt.

\clearpage
