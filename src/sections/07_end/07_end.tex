\section{Schluss}

Im Rahmen dieser Projektarbeit wurde Sprachübertragung und Sprachsynthese in ein cloudbasiertes Praxisrufsystem integriert.
Die umgesetzte Lösung basiert auf dem Praxisrufsystem das im Rahmen des Projektes ''Cloudbasiertes Praxisrufsystem'' umgesetzt wurde.\cite{ip5}
Das erweiterte System besteht aus einer Mobilen Applikation, einem Cloudservice und einer Web-Applikation.
Die Mobile Applikation wurde neu als native Applikation für iOS entwickelt.
Sie ersetzt die Shared Platform Applikation, welche im Rahmen des Vorgängerprojektes entwickelt wurde.
Dabei wurden sämtliche Funktionen und Anbindungen an Umsysteme auch in der neuen Applikation implementiert.
Mit diesem Projekt neu konzipiert und umgesetzt wurden die Sprachsynthese für empfangene Benachrichtigung mit ''AWS Polly '' sowie die Integration einer Gegensprechanalge über Peer To Peer Verbindungen.

Das umgesetzte Praxisrufsystem kann zum Austausch von Informationen in einem Praxisumfeld verwendet werden.
Als Endgeräte dienen dabei iOS Tablets.
Über die Mobile Applikation des Systems ist es möglich Sprachverbindungen zu einem oder mehreren anderen Clients aufzubauen.
Eingehende Sprachverbindungen werden automatisch angenommen.
Das System unterstützt weiter das Versenden und Empfangen von Benachrichtigungen.
Dies wird einerseits verwendet, um nicht erreichbare Empfänger über verpasste Sprachverbindungen zu informieren.
Weiter bietet die Applikation Praxismitarbeitenden die Möglichkeit vorkonfigurierte Benachrichtigungen an andere Clients zu versenden.
Der Inhalt von Benachrichtigungen kann dabei beim Empfang automatisch vorgelesen werden.
Sowohl empfangene Benachrichtigungen als auch verpasste und vergangene Anrufe, werden gesammelt und in einer Inbox angezeigt.
Empfangene Benachrichtigungen und verpasste Anrufe müssen von Praxismitarbeitenden quittiert werden.
Sind unquittierte Elemente in der Inbox, ertönt in regelmässigen Abständen ein Erinnerungston.

Das umgesetzte System deckt die wesentlichen Anforderungen eines cloudbasierten Praxisrufsystems ab.

Mandatenfähigkeit fehlt.
OAuth/OpenID Connect fehlt.
Stabilität und Resilience ist verbesserungswürdig.
Keine Erfahrung mit Mobile oder iOS.
Covid sucks.
Keine Erfahrung mit P2P oder Sprachübertragung.



\clearpage
