\section{Konzept}

Dieses Kapitel beschreibt das Konzept, nach welchem das cloudbasierte Praxisrufsystem aus dem Vorgängerprojekt erweitert wurde.
Die Konzepte wurden vor Beginn der Umsetzung definiert und während der Umsetzung laufend überarbeitet.
Die folgende Dokumentation beschreiben die neuste Version des Konzepts und stellen damit das umgesetzte System dar.
Dabei wird nicht hervorgehoben, welche Teile zu Beginn definiert und welche Teile später überarbeitet oder ergänzt wurden.

Das Konzept gibt zuerst einen Überblick über das System als Ganzes.
Es werden die einzelnen Teile des Systems beschrieben und es werden Schritte definiert um die Weiterentwicklung des Systems zu vereinfachen.
Nach dem Systemkonzept werden Konzepte für die Umsetzung der drei Features ''Migration des bestehenden Mobile Client'', ''Sprachsynthese'' und ''Gegensprechanlage'' beschrieben.
Das Kapitel schliesst mit einem Überblick über die Domänen- und Servicemodelle sowie der Schnittstellen welche die API des Systems bietet.

\subsection{Systemarchitektur}

Dieses Kapitel beschreibt die Komponenten des cloudbasierten Praxisrufsystems und wie diese erweitert werden.
Abbildung 7.1 gibt einen Überblick über die Systemkomponenten und welche Module diese beinhalten.
Dabei sind Module, die im Rahmen dieses Projektes entwickelt oder neu integriert werden, grün umrandet.
Die Pfeile zwischen Modulen zeigen gerichtete Abfragen und Aufrufe zwischen den Komponenten.
Alle Abfragen zwischen Services sowie zwischen Modulen des Cloudservices finden über die API Schnittstellen der jeweiligen Module statt.

\begin{figure}[h]
    \centering
    \begin{minipage}[b]{0.8\textwidth}
        \fbox{\includegraphics[width=\textwidth]{graphics/diagramms/Component_System_V03}}
        \caption{Systemkomponenten}
    \end{minipage}
\end{figure}

\subsubsection{Systemkomponenten}

Dieses Kapitel beschreibt die in Abbildung 7.1 abgebildeten Systemkomponenten.
Dabei wird für jede Komponente beschrieben, welche Aufgaben ihr zufallen und wie sie im Rahmen dieses Projektes erweitert wird.

\textbf{Cloudservice}

Der Cloudservice bildet die zentrale Serverkomponente von Praxisruf.
Zu Beginn dieses Projektes umfasst der Cloudservice die beiden Domänen Notification und Configuration.
Dabei ist die Domäne Notification für das Versenden von Benachrichtigungen und die Domäne Configuration für die Verwaltung und Auswertung der Konfigurationen verantwortlich.
Die relevanten Empfänger für eine Benachrichtigung werden in der Configuration Domain ermittelt.
Um auf diese Informationen zugreifen zu können, muss aus der Domäne Notification eine Abfrage an die Domäne Configuration gesendet werden.
Die Configuration-Domäne bietet dazu eine HTTP Schnittstelle~\cite{ip5}.

Die Trennung der Domänen wurde im Vorgängerprojekt lediglich auf Package-Ebene realisiert.
Mit diesem Projekt soll die Trennung einen Schritt weiter gehen.
Der Cloudservice wird in Module aufgeteilt.
Diese Module werden weiterhin in einer einzelnen Applikation zusammengefasst, können aber zu einem späteren Zeitpunkt in einzelne Microservices betrieben werden.

Nachdem die Auftrennung der bestehenden Domänen in Module stattgefunden hat, wird der Cloudservice erweitert.
Es werden die zwei Module Signaling und Speech Synthesis umgesetzt.
Das neue Modul Signaling ist für die Signalvermittlung zwischen Mobile Clients verantwortlich.
Es übernimmt die Aufgabe der Signaling Instanz für den Aufbau von Sprachverbindungen mit WebRTC.
Das Modul Signaling hat eine funktionale Abhängigkeit zum Modul Notification.
Es verwendet die API des Notification Moduls, um Empfänger mit Benachrichtigungen über verpasste Verbindungen zu informieren.

Das neue Modul Speech Synthesis dient als Schnittstelle zu einem externen Provider für Sprachsynthese.
Dies ermöglicht es, die Sprachsynthese als Teil der API des Cloudservice anzubieten.
Dadurch können Clients aller Plattformen und auch Systeme, die künftig angebunden werden, auf die Sprachsynthese zugreifen.
Weil alle Clients die Daten aus derselben Schnittstelle beziehen, ist garantiert, dass die Konfiguration und Funktionsweise dieselbe für alle Clients ist.

\textbf{Mobile Client}

Der Mobile Client ist eine mobile Appliaktion, über welche Praxisruf bedient werden kann.
Der mit dem Vorgängerprojekt umgesetzte Mobile Client erlaubt es Benachrichtigungen zu versenden und empfangen~\cite{ip5}.
Dieser Mobile Client wird durch eine native iOS Applikation ersetzt.
Dabei wird sämtliche Funktionen des Mobile Clients aus dem Vorgängerprojekt in die native App migriert.
Weiter werden die Funktionen Gegensprechanlage und Sprachsynthese für empfangene Benachrichtigungen umgesetzt.

\textbf{Admin UI}

Das Admin UI ist eine Webapplikation, über welche die Konfiguration des Praxisrufsystems verwaltet werden kann.
Die Konfiguration des Systems wird für Gegensprechanlage und Sprachsynthese erweitert.
Für die Gegensprechanlage müssen Buttons konfiguriert werden können.
Diese beinhalten Anzeigetext und Teilnehmer einer Sprachverbindung.
Weiter muss pro Benachrichtigung konfigurierbar sein, ob ihr Inhalt bei Empfang vorgelesen werden sollen.
Das Admin UI muss erweitert werden, um die Verwaltung der erweiterten Konfiguration zu ermöglichen.

\textbf{Messaging Service}

Der Messaging Service ist für die Zustellung von Push Benachrichtigungen an Mobile Clients verantwortlich.
Der Cloudservice muss an den Messaging Service angebunden sein, um Benachrichtigungen anhand der Konfiguration zu versenden~\cite{ip5}.
Die Anbindung des Cloudservices an den Messaging Service ist mit dem Vorgängerprojekt bereits umgesetzt und muss für dieses Projekt nicht angepasst werden.
Die neu entwickelte native iOS App muss hingegen an den Messaging Service angebunden werden, um Benachrichtigungen zu empfangen.
Als Messaging Service wird Firebase Cloud Messaging verwendet.

\clearpage

\textbf{Speech Synthesis Service}

Um Sprachsynthese zu ermöglichen, wird ein externer Service angebunden.
Dieser übernimmt die Konvertierung von Text aus Benachrichtigungen zu Sprachdaten.
Die Anbindung an den Speech Synthesis Service wird ausschliesslich im Cloudservice implementiert.
Sämtliche andere Komponenten die Sprachdaten benötigen, fragen diese beim Cloudservice ab.
Die REST API des Cloudservice wird um entsprechende Endpunkte erweitert.
Als Speech Synthesis Service wird Amazon Polly verwendet.

\subsubsection{Modularisierung Cloudservice}

Der mit dem Vorgängerprojekt umgesetzte Cloudservice ist als monolithische Applikation implementiert.
Er trennt intern die beiden Domänen Notification und Configuration.
Die Domäne Configuration ist für die Verwaltung und Auswertung der Konfiguration des Systems und die Domäne Notification für das Versenden von Benachrichtigungen verantwortlich.
Abhängigkeiten zwischen den beiden Domänen ist über eine REST-Schnittstelle abstrahiert.

Die Trennung der Domänen, erlaubt es die Anwendung zukünftig in mehrere Microservices aufzuteilen.
Diese könnten unabhängig betrieben und erweitert werden.
Weiter wird es dadurch möglich, einzelnen Teilen der Applikation mehr Ressourcen zuzuteilen.
Die Trennung der Domänen in eigene Microservices wurde im Rahmen des Vorgängerprojektes noch nicht vorgenommen.
Die beiden Domänen wurden lediglich durch die Package Struktur innerhalb eines einzelnen monolithischen Services getrennt.

Mit diesem Projekt wir die Trennung der Domänen innerhalb des Cloudservice verstärkt.
Die Applikation wird in strikt getrennte Module aufgeteilt.
Dabei wird pro Domäne ein Modul erstellt.
Dieses kapselt sämtliche Domänenobjekte, Services und Schnittstellen der jeweiligen Domäne.
Dadurch ist garantiert, dass die Domänen sauber voneinander getrennt sind.
Sämtliche Kommunikation zwischen den Modulen muss über die API der jeweiligen Module geschehen.

Es werden die vier Domänen-Module Configuration, Notification, Spech Synthesis und Signaling definiert.
Das Modul Configuration beinhaltet alle Domänenobjekte, Services und Schnittstellen für die Verwaltung, Auswertung und Abfrage der Systemkonfiguration.
Das Modul Notification beinhaltet alle Domänenobjekte, Services und Schnittstellen für das Versenden von Benachrichtiungen.
Das Modul Speech Synthesis beinhaltet die Anbindung an den Speech Synthesis Service.
Es stellt eine Schnittstelle zur Verfügung, über die Sprachdaten angefragt werden können.
Das Modul Signaling dient als Signaling Instanz für den Aufbau von Sprachverbindungen mit WebRTC\@.
Es beinhaltet Domänenobjekte, Services und Schnittstellen für die Signalvermittlung zwischen Mobile Clients.

Weiter werden die zwei Module Commons und App definiert.
Komponenten, welche in mehr als einer Domäne verwendet werden, werden in ein zusätzliches Commons Modul verlegt.
Dazu gehören Data Transfer Objects für Schnittstellen zwischen den Modulen, geteilte Clients um Abfragen auf andere Module abzusetzen sowie Komponenten für Security und Fehlerhandling.

Der Cloudservice wird weiterhin als monolithische Applikation betrieben.
Die Modularisierung garantiert dabei eine strikte Trennung der Domänen.
In Zukunft können einzelne Module aus dem Cloudservice ausgelöst und als eigenständige Applikationen betrieben werden.

\clearpage

\subsubsection{Domänenmodell Cloudservice}

Das Domänenmodell Cloudservices wird für die Integration von Sprachsynthese und Gegensprechanlage erweitert.
Abbildung 7.2 zeigt das vollständige Domänenmodell der Domänen Configuration und Notification.
Dabei sind Felder und Entitäten die in diesem Projekt hinzugefügt werden grün markiert.
Die neuen Domänen Speech Synthesis und Signaling führen keine persistierten Daten.
Die Services und Komponenten dieser Domänen sind in den Kapiteln 7.3 und 7.4 beschrieben.

\begin{figure}[h]
    \centering
    \begin{minipage}[b]{0.9\textwidth}
        \fbox{\includegraphics[width=\textwidth]{graphics/diagramms/erd_v02}}
        \caption{Entitiy Relationship Diagramm - Cloudservice}
    \end{minipage}
\end{figure}

\clearpage

\subsection{Nativer Mobile Client}

Dieses Kapitel beschreibt das Konzept für einen nativen iOS Client zur Bedienung des cloudbasierten Praxisrufsystems.
Es wird Entwurf und Funktionsweise der notwendigen Ansichten der Benutzeroberfläche beschrieben.
Weiter wird definiert, wie die aus dem Vorgängerprojekt zu migrierenden Funktionen in eine native iOS Applikation integriert werden können.
Dies beinhaltet insbesondere Anbindung der API des Cloudservice sowie Anbindung von Firebase Cloud Messaging.
Die Integration von Gegensprechanlage und Sprachsynthese wird nur im Rahmen des Entwurfs der Benutzeroberfläche erwähnt.
Eine detaillierte Beschreibung dieser Konzepte folgt in den Kapiteln 7.3 und 7.4.


\subsubsection{Benutzeroberfläche}

Die Ansichten zur Anmeldung und Zimmerauswahl werden analog zum bestehenden Mobile Client umgesetzt.
Die Loginseite beinhaltet einen kurzen Willkommenstext und ein Logo für Praxisruf.
Darunter findet sich ein einfaches Formular zur Eingabe von Benutzername und Passwort, sowie ein Button zur Bestätigung.

\begin{figure}[h]
    \centering
    \begin{minipage}[b]{0.4\textwidth}
        \fbox{\includegraphics[width=\textwidth]{/home/joshua/FHNW/dev/IP6/IP6_Bachelorarbeit_Bericht_Cloudbasiertes_Praxisrufsystem/src/graphics/mockups/mockup_login}}
        \caption{Mockup Login}
    \end{minipage}
    \hfill
    \begin{minipage}[b]{0.4\textwidth}
        \fbox{\includegraphics[width=\textwidth]{/home/joshua/FHNW/dev/IP6/IP6_Bachelorarbeit_Bericht_Cloudbasiertes_Praxisrufsystem/src/graphics/mockups/mockup_clientselect}}
        \caption{Mockup Zimmerwahl}
    \end{minipage}\label{fig:Mockups-Login-ClientSelection}
\end{figure}

Nach dem Eingeben der Anmeldedaten, erhält der Benutzer die Möglichkeit die gewünschte Konfiguration auszuwählen.
Die Ansicht besteht aus einem Seitentitel und einer Liste zur Auswahl der gewünschten Konfiguration.
In der Auswahl sind alle Zimmer zu sehen, welche dem Benutzer zur Verfügung stehen.
Diese Konfigurationen müssen vor der Anmeldung im Admin UI erfasst und dem Benutzer zugewiesen werden.
In der Kopfzeile sind die Schaltflächen ''Zurück'' und ''Fertig'' zu sehen.
Die Schaltfläche ''Zurück'', bricht die Anmeldung ab und führt zurück zur Eingabe der Logindaten.
Die Schaltfläche ''Fertig'' bestätigt die Auswahl und leitet zur Hauptansicht weiter.
Wird bestätigt, ohne dass ein Zimmer angewählt ist, wird dem Benutzer eine Fehlermeldung angezeigt und nicht zur Hauptansicht weitergeleitet.

Die Hauptansicht der Applikation gliedert sich in die Bereiche Home, Inbox und Einstellungen.
Zwischen den drei Bereichen kann über eine Leiste am unteren Ende des Bildschirms navigiert werden.
Die Ansicht Home zeigt dem Benutzer die Buttons, über welche er Benachrichtigungen versenden und Anrufe in der Gegensprechanlage starten kann.
Wird ein Anruf gestartet, wird die Ansicht für aktive Anrufe angezeigt.
Diese zeigt dem Benutzer den Titel des gestarteten Anrufs, sowie eine Liste aller Teilnehmer zusammen mit dem Verbindungsstatus jedes Teilnehmers.
Der Titel des Anrufes entspricht dem Anzeigetext des entsprechenden Buttons für ausgehende Anrufe und dem Namen des Anrufers für empfangene Anrufe.
Neben den Anrufinformationen zeigt die Ansicht für aktive Anrufe drei Buttons.
Über diese können Mikrofon und Lautsprecher des eigenen Gerätes stumm geschaltet werden.
Weiter kann über der Anruf über einen roten Button am rechten Rand beendet werden.
Nach einem beendeten Anruf wird automatisch zu der zuvor angezeigten Ansicht navigiert.

\begin{figure}[h]
    \centering
    \begin{minipage}[b]{0.4\textwidth}
        \fbox{\includegraphics[width=\textwidth]{/home/joshua/FHNW/dev/IP6/IP6_Bachelorarbeit_Bericht_Cloudbasiertes_Praxisrufsystem/src/graphics/mockups/mockup_intercom}}
        \caption{Mockup Home}
    \end{minipage}
    \hfill
    \begin{minipage}[b]{0.4\textwidth}
        \fbox{\includegraphics[width=\textwidth]{/home/joshua/FHNW/dev/IP6/IP6_Bachelorarbeit_Bericht_Cloudbasiertes_Praxisrufsystem/src/graphics/mockups/mockup_call}}
        \caption{Mockup Aktiver Anruf}
    \end{minipage}\label{fig:Mockups-Home-ActiveCall}
\end{figure}

Der Bereich Inbox zeigt eine Liste der empfangenen Benachrichtigungen sowie der empfangenen und verpassten Anrufe.
Für Benachrichtigungen wird der Titel der Benachrichtigung gefolgt vom Namen des Versenders in Klammern sowie der Inhalt der Benachrichtigungen angezeigt.
Für Anrufe wird der Name des Anrufers angezeigt. 
Zudem wird bei Anrufen beschrieben ob, es sich um einen empfangenen, verpassten oder abgelehnten Anruf handelt.
Einträge für Benachrichtigungen sowie verpasste und abgelehnte Anrufe müssen durch eine Wischgeste quittiert werden.
Die Funktionsweise der Quittierung wird aus dem bestehenden Mobile Client übernommen.
Quittierte Meldungen werden aus der Inbox entfernt und nicht mehr angezeigt.
Ein Quittieren von Meldungen und Anrufen passiert ausschliesslich lokal in der iOS Applikation.
Der Empfänger wird nicht über die Quittierung informiert~\cite{ip5}.

Es muss sichergestellt werden, dass verpasste Benachrichtigungen und Anrufe gesehen werden.
Dazu wird im Abstand von 60 Sekunden geprüft, ob unquittierte Benachrichtigungen oder Anrufe vorhanden sind.
Ist dies der Fall, wird ein Erinnerungston abgespielt und eine Benachrichtigung angezeigt.
Dieser Mechanismus wird in Kapitel 7.2.4 weiter beschrieben.

\begin{figure}[h]
    \centering
    \begin{minipage}[b]{0.4\textwidth}
        \fbox{\includegraphics[width=\textwidth]{/home/joshua/FHNW/dev/IP6/IP6_Bachelorarbeit_Bericht_Cloudbasiertes_Praxisrufsystem/src/graphics/mockups/mockup_inbox}}
        \caption{Mockup Inbox}
    \end{minipage}
    \hfill
    \begin{minipage}[b]{0.4\textwidth}
        \fbox{\includegraphics[width=\textwidth]{/home/joshua/FHNW/dev/IP6/IP6_Bachelorarbeit_Bericht_Cloudbasiertes_Praxisrufsystem/src/graphics/mockups/mockup_settings}}
        \caption{Mockup Einstellungen}
    \end{minipage}\label{fig:Mockups-Inbox-Settings}
\end{figure}

Abbildung 7.7 zeigt den Bereich Einstellungen.
Der Bereich Einstellungen zeigt den aktuellen Benutzernamen und die gewählte Konfiguration.
Über die Schaltfläche Abmelden, können sich Praxismitarbeitende aus der Applikation abmelden.
Die Schaltfläche Benachrichtigungen vorlesen ist standardmässig aktiviert.
Wird die Option deaktiviert, werden Benachrichtigungen nie vorgelesen.
Die Schaltfläche Anrufe empfangen ist ebenfalls standardmässig aktiviert.
Wird diese Option deaktiviert, werden alle empfangenen Anrufe automatisch abgelehnt und stattdessen eine Benachrichtigung angezeigt.
Ausgehende Anrufe können auch getätigt werden, wenn diese Option aktiviert ist.

\subsubsection{Anbindung CloudService}

Der Mobile Client muss an die API des Cloudservices angebunden werden.
Es wird eine Anbindung an die Domäne Configuration zur Anmeldung und Auswahl des gewünschten Zimmers, an die Domäne Notification zum Versenden von Benachrichtigungen und an die Domäne Speech Synthesis für den Bezug von Sprachdaten benötigt.
Die Schnittstellen dieser Domänen stehen als HTTP Endpunkte zur verfügung.
In diesem Unterkapitel wird beschrieben, wie REST Aufrufe in den nativen iOS Client integriert werden.
Das Abrufen von Sprachdaten und die Anbindung an die Signaling Instanz werden in den Kapiteln 7.3 und 7.4 beschrieben.

Die Basisbibliothek für iOS Entwicklung bietet die Klasse URLSession, über welche Netzwerkaufrufe getätigt werden können.
Über URLSession.shared steht eine Standard-Instanz zur Verfügung, über welche Netzwerkanfragen verarbeitet werden können~\cite{ios_urlsession}.
Die Klasse UrlRequest ermöglicht es, Http-Request für eine URL mit Header und Body zu erstellen~\cite{ios_urlrequest}.
Um die Integration dieser Klassen in die Applikation zu vereinfachen, wird ein zentraler Service mit dem Namen PraxisrufApi erstellt.
Dieser kapselt das Erstellen, Befüllen und Absetzten der nötigen UrlRequest Instanzen.
Er bietet öffentliche für die Http Verben Get, Post und Delete an über welche entsprechende Http-Requests abgesetzt werden können.
Das Klassendiagramm in Abbildung 7.8 zeigt den Aufbau des Service PraxisrufApi\@.

Die Basis URL für API Anfragen wird in der Konfiguration der iOS Applikation definiert.
Der PraxisrufApi Service lädt diese und verwendet sie als Basis für alle Abfragen die abgesetzt werden.
Alle öffentlichen Methoden von PraxisrufApi nehmen einen Parameter ''subUrl'' als String entgegen.
Dieser String wird der Basis Url angehängt.
Die Methoden Post und Put von PraxisrufApi nehmen zudem einen optionalen Parameter von Typ Data entgegen.
Dieser definiert den Inhalt des Request Bodies.
Mit diesen Informationen kann der Http Request erstellt und versendet werden.

\begin{figure}[h]
    \centering
    \begin{minipage}[b]{0.8\textwidth}
        \fbox{\includegraphics[width=\textwidth]{/home/joshua/FHNW/dev/IP6/IP6_Bachelorarbeit_Bericht_Cloudbasiertes_Praxisrufsystem/src/graphics/diagramms/Class_PraxisrufApi_V01}}
        \caption{Klassendiagramm PraxisrufApi}
    \end{minipage}
\end{figure}

Sämtliche öffentlichen Methoden des PraxisrufApi Service nehmen einen Parameter mit dem Namen completion entgegen.
Dabei handelt es sich um ein Callback, welches beim Erfolg oder Fehlschlagen der Http Anfragen ausgeführt wird.
Als Input dieses Callbacks wird immer der Typ Result$<$T, PraxisrufApiError$>$ verwendet.
Bei Result handelt es sich um einen Wrapper Typen welcher entweder das Resultat einer Abfrage oder ein Fehlerobjekt beinhaltet~\cite{ios_result}.
Im Fehlerfall wird das Result mit einem PraxisrufApiError Objekt erstellt.
Im Erfolgsfall wird der Wrapper mit einem Object vom Typ T erstellt.
Der Typ T ist generisch und wird vom Aufrufer der PraxisrufApi definiert.
Der Typ kann grundsätzlich beliebig sein, er muss aber das Protokoll Decodable aus der iOS Standardbibliothek unterstützen.
Decodable Instanzen können von einer JSON Spring Repräsentation in ein Swift Object konvertiert werden~\cite{ios_decodable}.
So kann das Resultat im Erfolgsfall aus der Response generiert und an, dass Callback übergeben werden.
Dadurch kann die Konvertierung generisch in der Basisklasse behandelt werden.

Anhand des Inhalts der Result-Instanz kann geprüft werden, ob die Anfrage erfolgreich war.
Das Resultat kann entsprechend verarbeitet werden.
Diese Prüfung und Verarbeitung findet innerhalb des completion-Callbacks statt.
Dieser Ansatz ermöglicht es im PraxisrufApi-Service ausschliesslich den API Call abzuhandeln.
Der Api Service muss keinen State führen und kann generisch für alle Anwendungen wiederverwendet werden.

Requests die über den Api Service erstellt werden, werden automatisch authentisiert.
Dazu lädt der Service die hinterlegten Credentials aus dem KeyStore von iOS und generiert einen entsprechenden Authorization Header.
Ist kein Token vorhanden, wird keine Anfrage abgesetzt.
Es wird direkt das completion-Callback mit einem Fehler-Resultat aufgerufen.

Mit dieser Lösung steht ein Service zur verfügung, über welchen Http Abfragen einfach integriert werden können.
Dank der generischen Methoden im Basis Service können neue Calls einfach hinzugefügt werden, ohne das Boilerplate-Code wiederholt werden muss.
Durch die completion Methoden kann zudem jede Abfrage den Bedürfnissen des Aufrufers entsprechend verarbeitet werden.

Um die Verwendung der API von Praxisruf weiter zu vereinfachen, wird der PraxisrufApi Service um sprechende Methoden für die nötigen Abfragen erweitert.
Dazu wird pro Domäne eine Extension-Klasse definiert erstellt.
Diese fügt Methoden mit sprechenden Namen für die angesprochene Funktionalität hinzu und kapseln die Verwendung der get, post und delete Methoden.

Der PraxsrufApi-Service ermöglicht es Abfragen an die Cloudservice API abzusetzen.
Die Resultate dieser Abfragen müssen in der Benutzeroberfläche angezeigt werden können.
Es wird pro Domäne ein weiterer Service geschrieben, welche den Aufruf des API Services kapselt.
Dieser Service bietet Methoden, über welche der PraxisrufApi Service angesprochen werden kann.
View-Komponenten können diese Services nutzen, um durch Benutzereingaben ausgelöste Anfragen an die Cloudservice Api zu senden.
Resultate und Fehler aus Anfragen an den Cloudservice werden in diesen Services als Instanzvariablen gehalten.
Die View-Komponenten können lesend auf diese Variablen zugreifen, um die entsprechenden Resultate oder Fehler anzuzeigen.

\subsubsection{Anbindung Messaging Service}

Um Benachrichtigungen empfangen zu können muss der Messaging Service Firebase Cloud Messaging an die native iOS Applikation angebunden werden.
Firebase bietet eine Bibliothek mit welcher Firebase Cloud Messaging in iOS Clients integriert werden kann~\cite{firebase_ios}.
Diese Integration kann allerdings nicht mit dem Mitteln von SwiftUI implementiert werden.
Dies liegt daran, dass für das Empfangen von Benachrichtigungen und das Anzeigen von Push-Benachrichtigungen Integration mit dem Betriebssystem notwendig ist.
Diese Integration kann bis heute nur über AppDelegates umgesetzt werden.
SwiftUI Applikationen können oft ohne AppDelegates implementiert werden.
Sobald aber Integration mit dem Betriebssystem notwendig ist, müssen AppDelegates verwendet werden.
Dazu können AppDelegates bei der Initialisierung der Applikation registriert werden.

Zur Anbindung von Firebase Cloud Messaging wird dementsprechend ein AppDelegate implementiert.
Die Logik des AppDelegates wird dabei auf das minimal Nötige reduziert.
Der AppDelegate selbst ist für die direkte Kommunikation mit Messaging Service und Betriebssystem verantwortlich.
Alle weitere Logik wird nicht im AppDelegate selbst ausgeführt, sondern an die Applikation delegiert.
Dies ermöglicht es die Anbindung des Messaging Service im AppDelegate zu kapseln.
Wird in Zukunft auf einen anderen Messaging Service gewchselt muss damit ausschliesslich die Logik im AppDelegate angepasst werden.
Weiter sorgt diese Trennung dafür, dass die Fachlogik vollständig mit SwiftUI implementiert werden kann.
Der AppDelegate beinhaltet lediglich die Teile, welche aus technischen Gründen nicht mit SwiftUI umgesetzt werden können.

Mit dem AppDelegate werden im Detail folgende Funktionen umgesetzt.
Beim Start der Applikation muss sich der Mobile Client beim Messaging Service registrieren.
Nach der Registrierung wird für den Mobile Client ein Token generiert, welches den Client eindeutig beim Messaging Service identifiziert.
Der AppDelegate muss, darauf reagieren und das erneuerte Token an die Applikation übergeben.
Der AppDelegate muss weiter die Möglichkeit bieten, den Client beim Messaging Service abzumelden.
Für die Verarbeitung von Benachrichtigungen muss der AppDelegate Benachrichtigungen im Vordergrund empfangen und als Push-Benachrichtigungen anzeigen können.
Die Informationen aus der empfangenen Benachrichtigung müssen anschliessend an die Applikation übergeben werden.
Der AppDelegate muss weiter Benachrichtigungen im Hintergrund empfangen und als Push-Benachrichtigung anzeigen können.
Sobald die Applikation wieder in den Vordergrund tritt, müssen die Daten dieser Benachrichtigung an die Applikation zur weiteren Verarbeitung übergeben werden.

\subsubsection{Scheduled Reminder für Inbox}

Der bestehende Mobile Client prüft in regelmässigen Abständen, ob ungelesene Benachrichtigungen in der Inbox vorhanden sind.
Wenn ungelesene Benachrichtigungen gefunden werden, wird ein Benachrichtigungston abgespielt, um den Praxismitarbeitende darauf aufmerksam zu machen.
Diese Prüfung findet bisher nur statt, wenn die Applikation in Vordergrund aktiv ist und nicht, wenn sie minimiert ist.
Diese Funktion wird mit dem nativen iOS Applikation übernommen.
Dazu werden zwei Komponenten umgesetzt.
Erstens wird InboxReminderService erstellt, welcher eine Liste der aktuellen Benachrichtigungen führt.
Zweitens benötigt es eine Komponente, welche diesen InboxService regelmässig überprüft und bei Bedarf den Erinnerungston abspielt.
Die Standardbibliothek für iOS bietet eine Timer Klasse.
Über diese ist es möglich auf einer View in regelmässigen Abständen Events auszulösen.\cite{ios_timer}
Ein solcher Timer wird auf der Hauptansicht für angemeldete Benutzer registriert.
Der Timer löst alle 60 Sekunden die Prüfung des InboxReminderService aus.

Die Prüfung von Benachrichtigungen im Hintergrund wird im Rahmen dieses Projektes nicht umgesetzt.
Für künftige Erweiterungen ist es möglich diese Funktion zu implementieren.
Um dies zu ermöglichen, müssen Benachrichtigungen auf dem Gerät persistiert werden.
So stehen die Daten auch zur Verfügung, wenn die Applikation nicht gestartet ist.
Weiter muss ein Hintergrundtask implementiert und registriert werden~\cite{ios_bgtaskscheduler}, welcher die persistierten Daten lädt und die darauf die Prüfung des InboxReminderService ausführt.

\subsubsection{Security}

Es muss eine sichere Übertragung von Daten gewährleistet sein.
Alle Daten zwischen den Services müssen über verschlüsselte Verbindungen ausgetauscht werden.
HTTP Anfragen an den CloudService erfolgen ausschliesslich über HTTPS.

Alle Anfragen an die API des Cloudservice müssen zudem mit einem entsprechenden JWT Token authentifiziert sein.
Praxismitarbeitende werden durch den Cloudservice mittels Basic Authentication authentifiziert.
Bei der Anmeldung mit Benutzername und Passwort liefert der Cloudservice ein JWT Token welches für weitere Anfragen an den Cloudservice weiterverwendet wird.
Sowohl die Credentials für die Basic Authentication als auch das JWT Token werden durch die iOS Applikation im Keystore des Betriebssystems gespeichert.
Das JWT Token wird regelmässig erneuert, indem die Basic Authentication mit den gespeicherten Credentials wiederholt wird.
Der Ablauf für Authentifizierung wird damit unverändert aus dem Vorgängerprojekt übernommen~\cite{ip5}.

Langfristig sollen Authentifizierung und Authorisierung mit OAuth und OpenId-Connect umgesetzt werden.
Diese Integration von OAuth und OpenId-Connect ist nicht Bestandteil dieser Projektarbeit.

\clearpage

\subsubsection{Servicemodell}

Abbildung 7.20 zeigt alle Servicekomponenten, die als Teil der nativen iOS Applikation implementiert werden.

\begin{figure}[h]
    \centering
    \begin{minipage}[b]{1\textwidth}
        \fbox{\includegraphics[width=\textwidth]{/home/joshua/FHNW/dev/IP6/IP6_Bachelorarbeit_Bericht_Cloudbasiertes_Praxisrufsystem/src/graphics/diagramms/Class_Mobile_Client_Draft_V02}}
        \caption{Klassendiagramm Modul SpeechSynthesis}
    \end{minipage}
\end{figure}

\clearpage


\clearpage

\subsection{Sprachsynthese}

Dieses Kapitel beschreibt die Integration von Sprachsynthese in Praxisruf.
Der Fokus liegt dabei auf den Abläufen zum Empfangen von Benachrichtigungen und dem Abrufen der Sprachdaten.
Der Empfang von Benachrichtigungen wird so erweitert, dass der Inhalt empfangener Benachrichtigungen automatisch vorgelesen wird.

\subsubsection{Konfiguration}

Benachrichtigungen für Praxisruf können über das Admin UI konfiguriert werden.
Es kann pro Benachrichtigung Titel, Inhalt, Anzeigetext für Benachrichtigungsbuttons und eine Beschreibung erfasst werden.
Diese Konfiguration wird über die Entität NotificationType verwaltet.
Neu soll auch konfiguriert werden können, ob eine Benachrichtigung für die Sprachsynthese relevant ist.
Dazu wird die Entität NotificationType um ein boolean Flag mit dem Namen ''isTextToSpeech'' erweitert.
Dieses Flag wird beim Versenden einer Benachrichtigung mitgesendet und kann vom Empfänger überprüft werden.
Wenn das Vorlesen von Benachrichtigungen in den lokalen Einstellungen und das Flag auf der Benachrichtigung aktiviert sind, wird die Benachrichtigungen vorgelesen.
Abbildung 7.11 zeigt einen Ausschnitt aus dem Entity Relationship Diagramm der Domäne Configuration.
Dabei sind die Felder, welche für die Sprachsynthese ergänzt werden, grün markiert.

\begin{figure}[h]
    \centering
    \begin{minipage}[b]{0.6\textwidth}
        \includegraphics[width=\textwidth]{/home/joshua/FHNW/dev/IP6/IP6_Bachelorarbeit_Bericht_Cloudbasiertes_Praxisrufsystem/src/graphics/diagramms/erd_t2s_v01.drawio}
        \caption{ERD Ausschnitt - Konfiguration Sprachsynthese}
    \end{minipage}
\end{figure}

Neben dem Feld isTextToSpeech, wird die NotificationType Entity um ein weiteres Feld ''version'' erweitert.
Das Versionsfeld beinhaltet eine Ganzzahl, welche mit jeder Änderung inkrementiert wird.
Der Inhalt dieses Felds wird ebenfalls beim Versenden von Benachrichtigungen mitgesendet.
Auf Client-Seite wird diese Information zur Implementierung eines Cache verwendet.

\subsubsection{Anbindung von Sprachsynthese in Cloudservice}

Dieses Kapitel beschreibt, wie Amazon Polly an den Cloudservice angebunden wird, um das Vorlesen von Benachrichtigungen zu ermöglichen.

Die Anbindung an Amazon Polly erfolgt zentral im Cloudservice.
Sämtliche Anfragen an Amazon Polly werden durch den Cloudservice gemacht.
Empfänger von Benachrichtigungen senden keine direkten Anfragen an Amazon Polly.
Sie kommunizieren stattdessen mit dem Cloudservice.
Dieser führt die Anfrage an Amazon Polly aus und gibt die Resultate an den Anfrager zurück.

Für die Anbindung von Amazon Polly wird der Cloudservice um ein Modul mit dem Namen ''Speech Synthesis'' erweitert.
Dieses Modul muss unabhängig von allen anderen Domänen-Modulen des Cloudservice umgesetzt werden.
Werden Daten aus einer anderen Domäne benötigt, muss die Kommunikation über die API des entsprechenden Moduls gehen.
Diese Trennung ermöglicht es, das Modul in Zukunft einfach aus dem Cloudservice auszubauen und als eigenständigen Microservice zu betreiben.

Die Abhängigkeit zu Amazon Polly als Anbieter soll weitmöglichst minimiert werden.
So kann bei Bedarf einfacher auf einen anderen Provider gewechselt werden.
Um dies zu ermöglichen wird das Interface SpeechSynthesisService definiert.
Dieses gibt eine einzelne Methode vor, welche eine InputStreamResource zurückgibt und eine Universal Unique Id (UUID) als Parameter entgegennimmt.
Der Parameter entspricht der technischen Identifikation des zu synthetisierenden Benachrichtigungstypes (NotificationType).
Die InputStreamResource muss die synthetisierten Sprachdaten enthalten.
Dieses Interface wird von der Komponente, welche die Schnittstelle nach aussen bietet verwendet.
Um einen Anbieter für Sprachsynthese anzubinden, kann dieses Interface implementiert werden und der Schnittstelle zur Verfügung gestellt werden.
Das Klassendiagramm in Abbildung 7.12 gibt einen Überblick über den Aufbau des Moduls Speech Synthesis.

\begin{figure}[h]
    \centering
    \begin{minipage}[b]{1\textwidth}
        \includegraphics[width=\textwidth]{/home/joshua/FHNW/dev/IP6/IP6_Bachelorarbeit_Bericht_Cloudbasiertes_Praxisrufsystem/src/graphics/diagramms/Class_AWS_Polly_Configuration_V02}
        \caption{Klassendiagramm - Modul SpeechSynthesis}
    \end{minipage}
\end{figure}

Für die Anbindung des Providers Amazon Polly wird das Interface SpeechSynthesisService mit der Klasse AwsPollySpeechSynthesisService implementiert.
Amazon stellt einen Java Bibliothek für Amazon Polly zur Verfügung, welcher diese Anbindung ermöglicht~\cite{aws_polly_sdks}.
Diese Bibliothek bietet alle Klassen die für die Anbindung an AWS Polly nötig sind und wird in der Implementierung des SprachSyntheseProviderService verwendet, um den Service anzubinden.

Die Anbindung von Amazon Polly benötigt drei Komponenten.
Als Erstes muss eine Instanz von AWSStaticCredentialsProvider zur Verfügung gestellt werden.
Dieser liefert die Credentials, welche das System berechtigen, Anfragen an AWS Polly zu senden.
Als Zweites muss eine Voice konfiguriert werden.
Die Voice definiert Sprache und Stimme, welche für die Sprachausgabe verwendet wird.
Letztlich muss eine Instanz von AmazonPollyClient konfiguriert werden.
Dieser Client wird verwendet, um Anfragen an AWS Polly zu senden.
Er verwendet den zuvor konfigurierten CredentialsProvider, um die Anfrage mit den entsprechenden Credentials zu ergänzen.
Die zuvor konfigurierte Voice wird bei Anfragen an Amazon Polly mitgesendet, damit die Daten mit den gewünschten Parametern synthetisiert werden.

Der Cloudservice ist als Java-Applikation mit Spring Boot umgesetzt.
Dies ermöglicht es, die notwendigen Komponenten in einer Spring Konfigurationsklasse zu konfigurieren und als Spring Beans zu instanziieren.
Über die Dependency Injection von Spring werden diese Komponenten dem AwsPollySpeechSynthesisService übergeben werden.

Werte, welche für die technische Konfiguration notwendig sind, werden aus der Konfigurationsdatei application.yml geladen.
Sprache und Region werden sich im Rahmen dieses Projektes nie ändern und beinhalten keine sensitiven Informationen.
Sie werden deshalb direkt in der Konfigurationsdatei definiert und mit dem Quellcode des Projektes verwaltet.
Als Credentials für die Anbindung dienen die zwei Schlüssel AccessKey und SecretKey.
Credentials werden nicht direkt in der Konfigurationsdatei gespeichert.
Stattdessen wird ein Platzhalter definiert, welcher die Werte für Credentials aus entsprechend benannten Umgebungsvariablen lädt.
Die Zugangsdaten müssen damit nicht mit dem Quellcode verwaltet werden.

\subsubsection{Sprachsynthese über Cloudservice API}

Endgeräte in Praxisruf müssen Sprachdaten über den Cloudservice beziehen können.
Das Modul Speech Synthesis stellt deshalb eine Schnittstelle zur Verfügung, über welche Sprachdaten abgefragt werden können.
Dabei ist es nicht möglich beliebige Textdaten in Sprachdaten zu verwandeln.
Stattdessen erlaubt die Schnittstelle die Abfrage von Sprachdaten für Inhalt und Sender einer Benachrichtigung.

Als Inhalt einer Benachrichtigung wird das Feld ''title'' aus der Entität NotificationType verwendet.
Der Name des Senders wird dem Feld ''name'' der Entität Client entnommen.
Beide Entitäten sind Teil des Moduls Configuration.
Die entsprechenden Daten müssen deshalb über die API des Configuration-Moduls geladen werden.
Um dies zu ermöglichen, werden die Identifikatoren der relevanten Entitäten zusammen mit Benachrichtigung versendet.

Der Endpunkt zum Bezug von Sprachdaten nimmt die zwei Parameter ''notificationTypeId'' und ''sender'' entgegen.
Diese müssen die technischen Identifikatoren der jeweiligen Entitäten beinhalten.
Anhand dieser Parameter werden der die benötigten Daten von der API des Configuration-Moduls geladen.
Anschliessend wird eine Anfrage an Amazon Polly gesendet um die Textdaten als Sprache zu synthetisieren.
Der zu synthetisierende Text setzt sich dabei aus Inhalt der Benachrichtigung und Name des Senders zusammen.
Die beiden Werte werden dabei durch ein Komma getrennt.
Dadurch wird eine Pause zwischen dem Vorlesen der einzelnen Werte eingefügt.
Die von Polly gelieferten Sprachdaten können anschliessend als Resultat zurückgegeben werden.

Der Endpunkt für die Abfrage von Sprachdaten im Cloud Service wird als Spring RestController umgesetzt.
Die Sprachdaten werden darin als Binärdaten mit Media Type ''audio/mp3'' im Body der Response zurückgegeben.
Der Endpunkt wird kann über Http-Get-Anfragen angesprochen werden.

\subsubsection{Security}

Anfragen an die API des Moduls Speech Synthesis müssen, wie alle Anfragen an die Cloudservice API, authentisiert werden.
Für die Authentisierung wird derselbe Mechanismus wie für Http-Anfragen in allen Cloudservice Modulen verwendet.
Über die Konfiguration des App Modules des Cloudservices wird die Authentifizierung aller Http-Requests überprüft.
Mit dieser Prüfung wird sichergestellt, dass ein gültiges JWT Token im Authentication Header der Anfrage vorhanden ist~\cite{ip5}.
Diese Prüfung wurde im Rahmen des Vorgängerprojektes umgesetzt und wird weiterverwendet.
Die entsprechenden Abläufe sind in dem Kapiteln 5.3.6 und 5.3.7 im Projektbericht ''IP5 Cloudbasiertes Praxisrufsystem'' dokumentiert~\cite{ip5}.

Um die Verschlüsselung der Übertragung von Sprachdaten und Anfragen zwischen Cloudservice und Mobile Client wird für die Übertragung ausschliesslich das Protokoll HTTPS verwendet.
Die Übertragung von Daten zwischen Cloudservice und Amazon Polly ist über Secure Sockets Layer (SSL) geschützt~\cite{aws_polly_encryption_in_transit}.

\subsubsection{Sprachsynthese in iOS App}

In der iOS App müssen empfangene Benachrichtigungen vorgelesen werden können.
Um dies zu ermögli-chen wird eine Anbindung an die Sprachsynthese-API des Cloudservice umgesetzt.
Dazu wird die in Kapitel 7.2 beschriebene Klasse PraxisrufApi erweitert.
Neben dem Abfragen von JSON Daten über HTTP Schnittstellen, muss diese für die Sprachsynthese auch das Herunterladen von Dateien unterstützten.
Dazu wird die Komponente URLSession aus der iOS Standardbibliothek verwendet.
Diese bietet mit URLSession.downloadTask die Möglichkeit Inhalte von einer URL herunterzuladen~\cite{ios_downloadtask}.

Der Service PraxisrufApi wird um eine Methode mit dem Namen ''download'' ergänzt.
Diese ist dafür verantwortlich, eine Anfrage für den Download mit Credentials aus dem iOS Keystore zu ergänzen und die Anfrage zu versenden.
Die Resultate der Anfrage und aufgetretene Fehler werden analog zu anderen Abfragen an eine Callback-Funktion übergeben.
Heruntergeladene Dateien werden von PraxisrufApi in einem temporären Verzeichnis gespeichert.
Das Resultat im Erfolgsfall ist deshalb nicht die heruntergeladene Datei selbst, sondern eine URL welche auf die Datei im temporären Verzeichnis zeigt.

Die Sprachsynthese für Benachrichtigungen muss automatisch ausgeführt werden, nachdem eine relevante Benachrichtigung empfangen wurde.
Der Empfang der Benachrichtigung findet über die Anbindung von Firebase Cloud Messaging im AppDelegate statt.
Die Benachrichtigung wird im AppDelegate empfangen und an die Applikation übergeben.
Die empfangene Benachrichtigung beinhaltet mit dem ''isTextToSpeech'' Flag, die Information, ob sie für die Sprachsynthese relevant ist.

Ist eine Benachrichtigung für Sprachsynthese relevant, werden die Sprachdaten dazu vom Cloudservice bezogen.
Dazu wird ein SpeechSynthesisService implementiert, welcher PraxisrufApi verwendet, um eine Anfrage an den Cloudservice zu senden.
Wurden die Daten erfolgreich geladen, kopiert der SpeechSynthesisService die heruntergeladenen Daten aus dem temporären Downloadverzeichnis in ein permanentes Verzeichnis.
Die Datei wird dabei unter dem Namen $NotificationTypeId.Version.SenderId$ gespeichert.
Sowohl NotificationTypeId als auch Version und SenderId können der empfangenen Benachrichtigung entnommen werden.
Nachdem die Sprachdatei unter dem neuen Namen gespeichert ist, wird ihr Inhalt abgespielt.

Die Namenskonvention für die gespeicherten Sprachdateien, erlaubt es ein Cache auf der Seite der iOS Applikation umzusetzen.
Bevor der SpeechSynthesisService eine Anfrage an den Cloudservice absetzt, prüft er, ob bereits eine Datei mit dem entsprechenden Namen vorhanden ist.
Ist dies der Fall, wird keine Anfrage an den Cloudservice gesendet und es wird die bereits gespeicherte Sprachdatei abgespielt.
Dieses Cache ermöglicht es Anfragen für Sprachsynthese zu minimieren und nach Änderungen trotzdem immer die aktuellsten Daten zu erhalten.

\clearpage

\subsubsection{Laufzeitsicht}

Dieses Kapitel beschreibt die Prozesse für das Vorlesen von Benachrichtigungen.
Dabei wird der Ablauf vom Versenden der Benachrichtigung bis hin zur Ausgabe der Sprachdaten auf Empfängerseite beschrieben.
Abbildung 7.13 stellt den Ablauf dem Empfangen einer Benachrichtigung aus Systemsicht dar.

Um eine Benachrichtigung zu versenden, sendet ein Mobile Client eine Anfrage an den Cloudservice.
Dieser lädt die gespeicherte Konfiguration und findet alle für die gewünschte Benachrichtigung relevanten Empfänger.
Anschliessend erstellt er für jeden Empfänger eine Benachrichtigung und versendet diese über den Messaging Service.
Der Messaging Service stellt die Benachrichtigungen an die Empfänger zu~\cite{ip5}.

Benachrichtigungen werden im Mobile Client über die Anbindung an den Messaging Service im AppDelegate empfangen.
Im AppDelegate werden die Informationen aus der empfangenen Benachrichtigung gelesen und in das interne Model der Mobile Client Applikation überführt.
Anschliessend wird die Benachrichtigung an das Betriebssystem übergeben damit auf dem Gerät ein Benachrichtigungston abgespielt und eine Push-Benachrichtigung angezeigt wird.
Daraufhin wird die Benachrichtigung im internen Model einem NotificationService übergeben.
Dieser fügt die empfangene Benachrichtigung in eine Inbox ein.
Ab diesem Moment ist die Benachrichtigung in der Inbox des Mobile Clients ersichtlich.

\begin{figure}[h]
    \centering
    \begin{minipage}[b]{0.8\textwidth}
        \includegraphics[width=\textwidth]{graphics/diagramms/Sequence_Speech_Synth_System}
        \caption{Sequenzdiagramm - Sprachsynthese auf Systemebene}
    \end{minipage}
\end{figure}


Nachdem eine empfangene Benachrichtigung der Inbox hinzugefügt wurde, wird geprüft ob Sprachsynthese in den lokalen Einstellungen aktiviert ist.
Ist diese deaktiviert, endet die Verarbeitung.
Andernfalls wird geprüft, ob das ''isTextToSpeech'' Flag auf der Benachrichtigung aktiviert ist.
Nur wenn das Flag aktiviert ist, wird die Benachrichtigung an den SpeechSynthesisService übergeben.
Der SpeechSynthesisService prüft als erstes, ob die Sprachdaten für die empfangene Benachrichtigung bereits lokal zur Verfügung stehen.
Dies wird gemacht in dem er überprüft, ob im Applikationsverzeichnis bereits eine Datei für Id, Version und Sender der Benachrichtigung vorhanden ist.
Ist dies der Fall, werden die Inhalte dieser Datei abgespielt und es wird keine Anfrage an den Cloudservice versendet.
Wenn die Daten gar nicht oder nur in einer anderen Version lokal gefunden werden, wird eine Anfrage an den CloudService gesendet.

Sobald Sprachdaten über die Cloudservice API angefragt werden, lädt dieser den Namen des Senders und die Inhalte der Benachrichtigung aus der Konfiguration.
Anschliessend sendet der Cloudservice eine Anfrage an Amazon Polly, um den Titel der Benachrichtigung als Sprachdaten zu synthetisieren.
Die Resultate von Amazon Polly werden als Resultat der Anfrage des Mobile Clients zurückgegeben.
Der Client speichert die empfangenen Daten lokal im Applikationsverzeichnis unter Id und Version verknüpften NotificationType.
Nachdem die Daten gespeichert wurden, wird deren Inhalt abgespielt.

\clearpage

\subsection{Gegensprechanlage}

Mit dem Einbau von synchroner Sprachübertragung wird das Praxisrufsystem um die Funktion Gegensprechanlage erweitert.
Die gewählte Technologie WebRTC erlaubt es Peer-to-Peer Sprachverbindungen zwischen Clients aufzubauen.
Dieses Kapitel beschreibt wie das Praxisrufsystem erweitert wird, um eine konfigurierbare Gegensprechanlage mit WebRTC zu integrieren.

\subsubsection{Konfiguration}

Die Gegensprechanlage wird in den nativen Mobile Client integriert.
Praxismitarbeitende können über Buttons Sprachverbindungen zu anderen Clients aufbauen.
Welche Buttons und damit welche Sprachverbindungen zur Verfügung stehen, wird durch Praxisadministrierende über das Admin UI konfiguriert.
Damit dies möglich ist, sind Änderungen an der Domäne Configuration des Cloudservice sowie am Admin UI notwendig.

Die Konfiguration von Mobile Clients wird in der Domäne Configuration abgebildet.
Zentral sind dabei die beiden Entities Client und ClientConfiguration.
Ein Client repräsentiert ein physisches Endgerät.
Eine ClientConfiguration definiert die Konfiguration eines Gerätes.

Praxisruf bietet bereits heute die Möglichkeit Buttons zu konfigurieren, über welche Benachrichtigungen versendet werden können.
Diese Buttons werden mit der Entität NotificationType konfiguriert, welche wiederum einer ClientConfiguration zugeordnet werden können.
Diese ClientConfiguration wird bei der Anmeldung auf dem Mobile Client geladen und verwendet, um die nötigen Buttons darzustellen.
Für die Konfiguration von Sprachverbindungen wird die Entität CallType erstellt.
Ein CallType beinhaltet den Text, welcher auf dem zugehörigen Button auf Clientseite angezeigt wird und eine Liste von Clients, welche als Ziel der Sprachverbindung verwendet werden.
Abbildung 7.12 zeigt einen Ausschnitt aus dem Entity Relationship Diagramm der Configuration Domäne.
Dabei sind die Teile, die für die Konfiguration von Sprachverbindungen ergänzt werden, grün markiert.

\begin{figure}[h]
    \centering
    \begin{minipage}[b]{0.7\textwidth}
        \includegraphics[width=\textwidth]{/home/joshua/FHNW/dev/IP6/IP6_Bachelorarbeit_Bericht_Cloudbasiertes_Praxisrufsystem/src/graphics/diagramms/erd_intercom_v02.drawio}
        \caption{ERD Ausschnitt - Konfiguration Gegensprechanlage}
    \end{minipage}
\end{figure}

Das Admin UI wird mit Ansichten erweitert, um CallTypes zu erstellen, anzeigen, bearbeiten und löschen.
Gleichzeitig wird der Cloudservice um Rest Endpunkte für das Lesen, Erstellen, Aktualisieren und Löschen von CallTypes erweitert.
Die Ansichten für ClientConfigurations im Admin UI werden so erweitert, dass CallTypes darauf angezeigt, hinzugefügt und entfernt werden können.
Die bestehenden Endpunkte für ClientConfiguration werden entsprechend erweitert.

\clearpage

\subsubsection{Signaling Instanz}

Mit WebRTC werden Peer-To-Peer Verbindungen aufgebaut~\cite{webrtc}.
Damit diese Verbindungen aufgebaut werden können, müssen die beteiligten Geräte Signalmeldungen austauschen können.
Dazu ist eine Instanz notwendig, welche Signale zwischen den Endgeräten vermitteln kann.
Diese Signaling Instanz wird als Teil des Cloudservice implementiert.

%Prinzipiell ist es möglich, den Signalaustausch in dieses Modul zu integrieren.
%Dieser Ansatz hat aber zwei grosse Probleme.
%Erstens ist die Grösse von Medlungen, welche über den gewählten Messaging Service (Firebase Cloud Messaging) ausgetauscht werden können beschränkt.
%Meldungen zum Aufbau von WebRTC Verbindungen können diese Grösse überschreiten.\footnote{cite}
%Zweitens geschieht das Versenden von Benachrichtigungen über einen asynchronen Mechanismus.
%Die Signale für Sprachverbindungen sollen aber synchron ausgetauscht werden.
%Es soll unmittelbar beim Versuch des Verbindungsaufbau klar sein, ob die Verbindung etabliert werden kann oder nicht.
%Diese Id entspricht der technischen Identifikation des entsprechenden Mobile Clients und muss, wenn sich der Mobile Client für eine Verbindung registriert mitgesendet werden.

Der Cloudservice wird um ein neues Modul ''Signaling'' erweitert.
Dieses soll den Austausch von Signalen zwischen Clients ermöglichen.
Gleich wie das Modul für Sprachsynthese wird es unabhängig von den anderen Domänenmodulen im Cloudservice implementiert.
Das Modul Signaling muss dabei zwei Aufgaben übernehmen.
Erstens muss es Mobile Clients die Möglichkeit bieten, sich für Sprachverbindungen zu registrieren.
Zu diesem Zweck müssen Mobile Clients eine Verbindung mit der Signaling Instanz herstellen und trennen können.
Zweitens muss es Signalmeldungen empfangen und an die relevanten Empfänger zustellen können.
Kann eine Signalmeldung nicht zugestellt werden, muss es den betroffenen Empfänger über das verpasste Signal informieren.

Für diese Funktionen wird das Interface ClientConnector definiert.
Dieses definiert die Methoden afterConnectionEstablished und afterConnectionClosed.
Die beiden Methoden werden aufgerufen, wenn eine Verbindung geöffnet bzw.\ geschlossen wurde.
Die Implementierung dieses Interfaces ist dafür verantwortlich verfügbare Verbindungen zu verwalten.
Weiter definiert das Interface die Methode handleSignal.
Diese muss verwendet werden, um ein Signal entgegenzunehmen und an relevante Empfänger weiterzuleiten.

\begin{figure}[h]
    \centering
    \begin{minipage}[b]{1\textwidth}
        \includegraphics[width=\textwidth]{/home/joshua/FHNW/dev/IP6/IP6_Bachelorarbeit_Bericht_Cloudbasiertes_Praxisrufsystem/src/graphics/diagramms/Class_Intercom_Full_V01}
        \caption{Klassendiagramm SpeechSynthesisController}
    \end{minipage}
\end{figure}

Die Klasse SignalingService implementiert das ClientConnector Interface.
Der SignalingService führt eine Liste an verfügbaren Verbindungen.
Dabei wird jede Verbindung durch eine eindeutige Id gekennzeichnet.
Für die Verwaltung von Verbindungen wird die Komponente ConnectionRegistry implementiert.
Diese muss Verbindungen anhand einer Id registrieren und wieder entfernen können.
Weiter müssen Verbindungen anhand ihrer Id gefunden werden können.
Für das Zustellen von Signalen über bekannte Verbindungen wird die Methode handleSignal implementiert.
Jedes Signal muss die Identifikation seines Empfängers beinhalten.
Beim Empfang eines Signals wird kontrolliert, ob die ConnectionRegistry eine Verbindung für die Identifikation des Empfängers enthält.
Ist dies der Fall, wird das Signal über diese Verbindung an den Empfänger übermittelt.
Wenn dies nicht der Fall ist oder wenn das Senden des Signals fehlschlägt, ist der Empfänger nicht erreichbar.
Nicht erreichbare Empfänger werden mit Benachrichtigungen über verpasste Signale informiert.
Dazu wird eine Benachrichtigung über die API des Moduls Notification versendet.

Die Schnittestelle der Signaling Instanz im Cloudservice wird mit Websockets umgesetzt.
Dazu wird die Bibliothek Spring-Boot-Starter-Websocket verwendet.
Es wird ein WebsocketHandler implementiert, welcher unter dem Pfad ''$<$serverUrl$>$/signaling'' erreichbar ist.
Etablierte Verbindungen müssen eindeutig einem Client zugeordnet werden können.
Diese Identifikation wird als Query Parameter bei Verbindungsaufbau mitgegeben.
Der WebsocketHandler definiert Methoden die beim Öffnen und Schliessen von Verbindungen aufgerufen werden.
Diese delegieren das Verwalten der Verbindungen an den SignalingService.
Empfangene Meldungen werden an den SignalingService übergeben und wie oben beschrieben verarbeitet.

\subsubsection{Signaling Security}

Der Zugriff auf den Signaling Service darf nur für Berechtigte möglich sein.
Um dies sicherzustellen, wird der Verbindungsaufbau nur erlaubt, wenn die Anfrage dazu authentisiert ist.
Für die Authentisierung wird derselbe Mechanismus wie für Http Anfragen in den anderen Cloudservice Domänen verwendet werden.
Über die SecurityConfig des Cloudservices wird die Authentifizierung aller Http Requests überprüft.
Mit dieser Prüfung wird sichergestellt, dass ein gültiges JWT Token im Authentication Header der Anfrage vorhanden ist.
Wenn kein gültiges Token vorhanden ist, wird eine entsprechende Fehlermeldung zurückgegeben.
Der Aufbau der Websocketverbindung wird abgebrochen.

Mit der Klasse HttpSessionHandshakeInterceptor kann vor dem Aufbau einer Websocketverbindung geprüft werden, dass der Http Request für den Verbindungsaufbau authentifiziert wurde.
Mit der Verarbeitung des Http Requests durch die SecurityConfig des Cloudservices wurden die Rollen, welche dem Aufrufer zugeteilt ausgelesen.
Im HttpSessionHandshakeInterceptor können diese Informationen ausgewertet und geprüft werden, dass ein Benutzer für den Austausch von Signalmeldungen authorisiert ist.
Ist er dies nicht, wird der Aufbau der Websocketverbindung wird abgebrochen und eine Fehlermeldung zurückgegeben.
Das Praxisrufsystem kennt die zwei Rollen ''ADMIN'' und ''USER''.
Beide Rollen sind berechtigt, dass Rufsystem über den Mobile Client zu verwenden und dürfen damit Signalmeldungen austauschen.

Für den Aufbau von Websockets wird das ausschliesslich das Protokoll Secure WebSockets (WSS) verwendet.
Der Austausch von Signalmeldungen ist damit verschlüsselt.

\clearpage

\subsubsection{Signalmeldungen}

Dieses Kapitel beschreibt die Signalmeldungen, welche für Sprachverbindungen verwendet werden.
Abbildung 7.14 zeigt, wie Signalmeldungen im Mobile Client modeliert werden.

\begin{figure}[h]
    \centering
    \begin{minipage}[b]{0.33\textwidth}
        \fbox{\includegraphics[width=\textwidth]{graphics/diagramms/Class_Signal_V01}}
        \caption{Signal }
    \end{minipage}
\end{figure}

Alle Signalmeldungen beinhalten Identifikation von Sender und Empfänger.
Diese werden vom Signalingserver verwendet, um die Signale korrekt weiterzuleiten.
Type, Payload und Description werden im Mobile Client verwendet, um das Signal korrekt zu verarbeiten und Verbindungen aufzubauen.
Das Flag notificationOnFailedDelivery wird im Cloudservice ausgewertet.
Wenn ein Signal nicht zugestellt werden kann und dieses Flag TRUE ist, wird der Empfänger mit einer asynchrone Benachrichtigung darüber informiert.
Dazu wird das bestehende Notification Modul des Cloudservices verwendet.
Für die Verwaltung von Sprachverbindungen werden die folgenden sechs Typen von Signalmeldungen definiert.
\\ \\
\begin{tabbing}
    Left \= Middle \= Right \= Right \kill
    Offer
    \> \> \> Ein Offer wird vom Initiator der Sprachverbindung an die Empfänger gesendet. Es
    \\\> \> \> beinhaltet die Informationen des Initiators. \\ \\

    Answer
    \> \> \> Eine Answer wird vom Empfänger eines Offers an den Initiator der Sprachverbindung
    \\\> \> \> gesendet. Es beinhaltet SDP Informationen des Empfängers. \\ \\

    Ice Candidate
    \> \> \> Ice Candidate Signal Beinhaltet Informatinoen eines ICE Candidates, die für den Ver-
    \\ \> \> \> bindungsaufbau verwendet werden. Nach Verarbeitung von Offer und Answer
    \\ \> \> \> tauschen Initiator und Empfänger solange Ice Candidate Signale aus, bis sie sich auf
    \\ \> \> \> einen Kandidaten geeinigt haben.\\ \\

    End
    \> \> \> Wird von Empfänger oder Initiator versendet, nachdem die Verbindung durch tippen des
    \\ \> \> \> Auflegen-Button in der Applikation beendet wurde. Empfang dieses Signal führt dazu,
    \\ \> \> \> dass offene Sprachverbindungen zum Sender dieses Signals beendet werden.\\ \\

    Unavailable
    \> \> \> Wenn der Signalingserver ein Signal nicht zustellen kann, wird das Ziel über eine Benach-
    \\ \> \> \> richtigung informiert. Zudem wird ein Unavailable Signal zurück an den Sender gesendet.
    \\ \> \> \> Es wird angezeigt, dass der gewünschte gesprächspartner nicht verfügbar ist. \\ \\

    Decline
    \> \> \> Wird ein Signal empfangen während die Gegensprechanlage in den lokalen Einstellungen
    \\ \> \> \> deaktiviert ist, sendet der Mobile Client ein Decline Signal zurück. Es wird angezeigt,
    \\ \> \> \>  dass der gewünschte gesprächspartner nicht verfügbar ist.
\end{tabbing}

\subsubsection{Anmeldung und Registrierung}

Der Ablauf von Anmeldung und Registrierung funktioniert mit dem neuen nativen Mobile Client grundsätzlich gleich wie zuvor.
Praxismitarbeitende öffnen die Applikation und geben ihr Benutzername und Passwort ein.
Der Mobile Client verwendet diese um sich über Basic Authentication beim Cloudservice anzumelden.
Als Antowrt auf die Anmeldung gibt der Cloudservice ein JWT Token zurück.
Dieses wird für die Authentifizierung aller weiteren Anfragen an den Cloudservice verwendet.
Nachdem die Anmeldung erfolgt ist, wird eine Liste der verfügbaren Konfigurationen geladen.
Der Benutzer wählt die gewünschte Konfiguration aus und bestätigt.

\begin{figure}[h]
    \centering
    \begin{minipage}[b]{0.9\textwidth}
        \fbox{\includegraphics[width=\textwidth]{/home/joshua/FHNW/dev/IP6/IP6_Bachelorarbeit_Bericht_Cloudbasiertes_Praxisrufsystem/src/graphics/diagramms/Sequence_Registration}}
        \caption{Sequenzdiagramm - Anmeldung und Registrierung im Mobile Client}
    \end{minipage}
\end{figure}

Danach wird diese Konfiguration geladen und die Hauptansicht angezeigt.
Die geladene Konfiguration beinhaltet alle Informationen die nötig sind um Buttons für Benachrichtigungen und Sprachverbindungen anzuzeigen.
Im Hintergrund muss sich der Mobile Client nun für Benachrichtigungen und Sprachverbindungen registrieren.
Für Benachrichtigungen registriert er sich zuerst bei Firebase Cloud Messaging.
Er erhält ein Token, welches den Client beim Messaging Service identifiziert.
Dieses Token sendet der Mobile Client zusammen mit der gewählten Konfiguration an den Cloudservice.
Dieser persistiert die Registrierung und kann sie verwenden, um Benachrichtigungen an diesen Client zuzustellen.
Für Sprachverbindungen muss zudem eine Verbindung zum Signaling Modul des Cloudservices aufgebaut werden.
Dazu wird wie in Kapitel 7.4.6 beschrieben eine Websocketverbindung geöffnet.

\clearpage

\subsubsection{Verbindungsaufbau}

Praxismitarbeitende können Sprachverbindungen zu anderen Clients aufbauen indem sie auf den entsprechenden Button in der Praxisruf App tippen.
Zum Zeitpunkt an dem der Button getippt wird, weiss der Mobile Client noch nicht, zu welchen Clients diese Verbindung aufgebaut werden soll.
Als erstes muss deshalb beim Cloudservice angefragt werden, welche Clients mit dem betätigten Button angesprochen werden sollen.
Der Cloudservice bietet dazu einen Endpoint an über den der vollständige CallType des verwendeten Buttons hinterlegt sind geladen werden können.
Nachdem diese Informationen geladen sind, können Sprachverbindungen zu allen relevanten Clients aufgebaut werden.
Dazu müssen Offer, Answer und Ice Candidate Signale ausgetauscht werden.
Der auslösende Client initialisiert die Peer to Peer Verbindung auf seiner Seite und sendet für jeden Gesprächspartner ein Offer.
Der Cloudservice findet die Verbindung der relevanten Empfänger und leitet die Signale über die jeweilige Verbindung weiter.
Die Verbindung auf Empfängerseite wird nach Eingang des Offers initialisiert.
Danach wird eine Answer zurück an den Initiator gesendet, welche über den Signalingserver zugestellt wird,
Der Initiator empfängt die Antwort Signale und ergänzt die notwendigen Verbindungsinformationen.
Abbildung 7.15 visualisiert diesen Ablauf.

\begin{figure}[h]
    \centering
    \begin{minipage}[b]{0.9\textwidth}
        \includegraphics[width=\textwidth]{graphics/diagramms/Sequence_Intercom_Broking_V02}
        \caption{Ablauf Verbindungsaufbau Gegensprechanlage}
    \end{minipage}
\end{figure}

Nachdem die Offer und Answer Meldungen ausgetauscht sind, müssen Ice Candidate Meldungen ausgetauscht werden.
Dieser Austausch ist der um die Übersichtlichkeit zu wahren nicht in Abbildung 7.15 dargestellt.
Er verläuft nach demselben Prinzip wie der Austausch von Offer und Answer.
Sobald sich die Clients auf die Verbindungsinformationen geeinigt haben, besteht die Sprachverbindung.

\clearpage

\subsubsection{Anbindung Mobile Client an Singaling Instanz}

Für den Aufbau von Sprachverbindungen zwischen Mobile Clients müssen mehrere Signalmeldungen ausgetauscht werden.
Der Cloud Service wird im Rahmen dieses Projektes um eine Schnittstelle erweitert, welche dies ermöglicht.
Als Technologie für diese Schnittstelle werden Websockets verwendet.
Der Service PraxisrufApi wird dementsprechend erweitert, um Websocket Verbindungen zu ermöglichen.
Dies beinhaltet den Auf- und Abbau von Websocket Verbindungen, sowie das Senden und Empfangen von Meldungen über diese Verbindung.
Die Verbindung zur Signaling Instanz konstant offen gehalten und im Fehlerfall erneut aufgebaut werden können.

Der Austausch von Signalmeldungen ist der einzige Anwendungsfall in Praxisruf, der Websocketverbindungen benötigt.
Deshalb wird auf eine generische Integration von Verbindungen analog von Http Verbindungen verzichtet.
Es werden die Extension PraxisrufApi+Signaling und das Protokoll PraxisrufApiSignalingDelegate definiert, wie in Abbildung 7.17 dargestellt implementiert.

\begin{figure}[h]
    \centering
    \begin{minipage}[b]{0.9\textwidth}
        \includegraphics[width=\textwidth]{graphics/diagramms/Class_Mobile_Client_Signaling_Connection}
        \caption{Klassendiagramm - Signaling Schnittstelle in Mobile Client}
    \end{minipage}
\end{figure}


Die Extension PraxisrufApi+Signaling ist für die Verbindung zu der Signaling Instanz verantwortlich.
Für die Integration dieser Extension in den Rest der Applikation wird das Protokoll PraxisrufApiSignalingDelegate definiert.

Die Verbindung zu der Signaling Instanz wird direkt nach der Anmeldung im Mobile Client aufgebaut.
Dazu wird die Methode connectSignalingServer verwendet.
Die Identifikation des Clients wird bei dieser Abfrage als Parameter mitgegeben.
So kann die Verbindung von der Signaling Instanz eindeutig einem Client zugeordnet werden.
Nachdem die Verbindung geöffnet ist, dürfen Signale empfangen und verarbeitet werden.
Dies wird durch die Methode listenForSignal initialisiert.
Darin wird der Websocketverbindung signalisiert, dass der Client bereit ist die nächste Meldung zu empfangen.
Sobald eine Meldung empfangen wird, wird diese über PraxisrufApiSignalingDelegate verarbeitet.

Für die Verarbeitung von Meldungen wird die entsprechende Methode des PraxisrufApiSignalingDelegate verwendet.
Wurde eine gültige Signalmeldung empfangen, wird diese über die Methode onSignalReceived verarbeitet.
Wurde hingegen eine ungültige Signalmeldung oder eine Fehlermeldung empfangen wird die Methode onErrorReceived aufgerufen.
Im Fehlerfall wird zudem überprüft ob die Verbindung noch offen verwendbar ist.
Sollte die Verbindung nicht mehr verwendbar sein, wird die Methode onConnectionLost des Delegates aufgerufen.




\clearpage

\subsubsection{Signalverarbeitung im Mobile Client}

Neben austausch von Signalen muss auch die Sprachverbindung selbst verwaltet werden.
Die Details der Sprachverbindung sind vendor spezifisch.
Es wird deshalb eine eigene Klasse CallClient erstellt.
Diese ist für das Verwalten von Verbindungen verantwortlich.
Bei eingehenden Verbindungen muss sie signale empfangen, die Verbindung erstellen.
Wenn nötig Antowrt Signale erstellen und zurückgeben.
Bei ausgehenden Verbindungen muss sie Verbindung initialisieren.
Es muss Signal erstellt und versendet werden.
Weiter müssen Methoden angeboten werden um die Unterhaltung oder das Microphon zu muten.
Und um die Verbindung zu schliessen.
Und um den bei Änderung des internen Status, das UI entsprechend zu aktualisieren.

Um die Integration der Sprachverbindung möglichst unabhängig und auswechselbar zu machen, wird der CalLClient nicht direkt in der View verwendet.
Stattdessen definiert der CallClient ein Delegate Protocol, welches die notwendigen Callbacks definiert.

\begin{figure}[h]
    \centering
    \begin{minipage}[b]{0.9\textwidth}
        \includegraphics[width=\textwidth]{graphics/diagramms/Class_Mobile_Client_Signal_Processing}
        \caption{Klassendiagramm - Signalingverarbeitung in Mobile Client}
    \end{minipage}
\end{figure}

Um Anrufe in der Applikation verwenden zu können müssen CallClient und PraxisrufApi+Signaling in die Benutzeroberfläche integriert werden.
Beide Applikationen definieren ein Delegate Protocol, welches die Funktionen spezifiziert, über welche die Komponenten eingebunden werden können.
Es wird eine weitere Serviceklasse mit dem Namen CallService implementiert, welche beide Delegate Protocols implementiert.
Dieser Service instanziert CallClient und PraxisrufApi+Signaling und registriert sich anschliessend als Delegate bei beiden Instanzen.
Der CallService selbst wird in der View verwendet.
Er nimmt Benutzereingaben entgegen und delegiert die entsprechende Funktionalität an den CallClient und SignalingClient.

Wenn der Benutzer einen Anruf startet, wird die View für aktive Anrufe geladen.
Diese initialisiert den Anruf über den CallService.
Der CallService ruft dazu als erstes den CallClient auf.
Der CallClient initialisiert die lokalen Verbindungsinformationen und erstellt ein Signel, um den Empfänger zu informieren.
Dieses Signal gibt er an den CallService weiter.
Der CallService leitet das Signal an den SignalingClient weiter, welcher das Versenden an den Cloud Service übernimmt.

\clearpage
\begin{figure}[h]
    \centering
    \begin{minipage}[b]{0.8\textwidth}
        \includegraphics[width=\textwidth]{/home/joshua/FHNW/dev/IP6/IP6_Bachelorarbeit_Bericht_Cloudbasiertes_Praxisrufsystem/src/graphics/diagramms/Sequence_MobileClient_Caller_Signaling.drawio}
        \caption{MobileClient - Anruf Starten Signal}
    \end{minipage}
\end{figure}

Das versendete Signal wird über das Signaling Modul des Cloud Service an den Empfänger übermittelt.
Dieser empfängt das Signal über den SingalingClient.
Der SignalingClient gibt das Signal über onSignalReceived an den CallService weiter.
Der CallService aktiviert die Ansicht für aktive Anrufe und leitet das Signal an den CallClient weiter.
Der CallClient initialisiert die lokalen Verbindungsinformationen und erstellt eine Signal zur Bestätigung.
Dieses Signal wird wiederun über den CallService zum SignalingClient weiter zum Cloud Service versendet.
Auf Starterseite, kann diese Bestätigung weiterverarbeitet werden.

\clearpage

\subsubsection{Verpasste Anrufe}

Anrufe über die Gegensprechanlage können nur empfangen werden, solange die Praxisruf App im Vordergrund läuft und beim Signalingserver registriert ist.
Ist dies nicht der Fall, kann der Signalingserver keine Signale an den jeweiligen Empfänger zustellen.
Wenn der Singalingserver ein Signal für einen Empfänger erhält, der nicht verbunden ist wird versucht, diese über eine Benachrichtigung zu informieren.
Benachrichtigungen für verpasste Anrufe, werden gleich wie alle anderen Benachrichtigungen empfangen und in der Inbox angezeigt.
Für das Versenden der Benachrichtigung für verpasste Anrufe wird das bestehende Notification Modul des Cloudservice verwendet.
Dieses wird um einen Endpunkt erweitert, der es erlaubt Benachrichtigungen gezielt an einen einzelnen Client zu versenden.
Der neue Enpunkt nimmt zwei Parameter entgegen:
Die technische Identifikation des Empfängers und die technische Identifikation des relevanten Benachrichtigungstyps.
Der relevante Benachrichtigungstyp muss durch den Praxisadministrator im Admin UI erfasst werden.
Die technische Identifikation dieses Benachrichtigungstyps muss anschliessend in den Umgebungsvariablen des Signalingservices hinterlegt werden.\footnote{Vgl. Installationsanleitung}
Wenn der Empfänger nicht für Benachrichtigungen registriert ist, kann er nicht über den verpassten Anruf informiert werden.
Dasselbe gilt, wenn die Benachrichtigung aus technischen Gründen nicht zugestellt werden kann.
Im Rahmen dieses Projektes wird kein Mechanismus implementiert, um diese fehlgeschlagene Zustellung automatisch zu wiederholen.

\subsubsection{Deaktivierte Anrufe}

Praxismitarbeitende können das Empfangen von Anrufen lokal deaktivieren.
Wenn der Empfang von Anrufen deaktiviert ist, werden Offer Signale nicht mit einer Answer sondern mit einem Declined Signal beantwortet.
Empfängt ein Client ein Declined Signal, erstellt er lokal einen Eintrag in der Inbox und zeigt eine Push Benachrichtigung an um den Benutzer darauf hinzuweisen.

\subsubsection{Verbindungsende}

Praxismitarbeitende können Sprachverbindungen durch einen Button beenden.
Durch antippen des Auflagen-Buttons wird die bestehende Peer to Peer Verbindung zum Gesprächspartner getrennt.
Anschliessend wird ein Signal vom Typ End an die Gesprächspartner gesendet.
Durch Empfang des End-Signals, weiss der Gesprächspartner, dass die Verbindung durch das Gegenüber getrennt wurde und kann die Verbindung seinerseits entfernen.


\clearpage

\input{sections/07_concept/07_05_summary.tex}
