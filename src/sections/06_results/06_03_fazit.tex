\subsection{Fazit}

In diesem Kapitel werden die zentralen Herausforderungen während der Projektarbeit und die Schlussfolgerungen die daraus gezogen werden können beschrieben.

Nativer Mobile Client was herausfordernd.
Insbesondere Anbindung an Umsysteme und effiziente Integration von WebRTC.
Mehraufwand für saubere INtegration (praxisrufapi) hat sich gelohnt.
Hat länger gebraucht, die Anbindung am schluss ist aber recht übersichtlich und gefällt mir.

Fazit: Konzepte Lohnen sich.
Wirklich.
Das detaillierte ausarbeiten von Abläufen auf Fachlicher Ebende, für Connection handling und für Rest Calls hat sich gelohnt.
Ich bin mit der Integration dieser Andbindungen im Mobile Client echt zufrieden.

Implementation an WebRTC war auch eher müsham.
Es gibt native SDKs.
Diese haben allerdings keine nennenswerte Entwicklerdokumentation.
Es finden sich leicht diverse Referenzimplementierungen oder Tutorial Style Github Repositories.
Dabei handelt es sich aber immer nur um einfache Dinger.
Die saubere Integration in ein UI mit controls fehlt oft.
Insbesondere für SwiftUI.
WebRTC ist aber schlussendlich ein gut verbreiteter Standard.
Der auch in allen Grossen Browsern unterstützt wird.
Die Konzepte von WebRTC sind dieselben, unabhängig von der Platform.
Dementsprechend könnena uch diese Resourcen verwendet werden.
Verbindungen funktioneren zuverlässung und soweit stabil.
Ledliglich Problem mit Feedback wenn nahe zusammen.
Hat den Vorteil, dass simpler eigener Signaling Service verwendet werden kann.
Keine Providerbindung und maximale Flexibilität.
Handelt sich aber halt um open source.
Wer weiss, wann google das fallen lässt oder kompatibilität plötzlich nicht mehr da ist.

Fazit: WebRTC ist voll okay für Praxisrufsystem.
Für kommerzielle Nutzung sollte aber ein Notfallszenario definiert werden.
Dies beinhaltet zeitnahe evaluation bei neuen iOS Releases, damit alles kompatibel ist.
Und ein Szenario wie WebRTC bei Bedarf ersetzt werden kann.
Szenario war im Rahmen dieses Projekts nicht möglich.
Integration in Mobile App ist aber so, dass die Andbindung ins UI wiederverwndet werden kann, auch wenn andere Call Technologie verwendet wird.
Letztlich gibt es mit open source auch sowas wie provider bindung halt anders.
Darum egal ob provider oder open source: Notfallplan gehört dazu.

Covid war auch eine Challange.
Im Methoden Teil wurde angedacht, dass scrum mässig zusammengesessen und getestet wird.
Das hat aus zwei gründen nicht ganz wie erwartet funktioniert.
Einerseits, ist der Konzept teil zu lang.
Nicht länger als angedacht, aber halt doch lang.
Meetings mussten grösstenteils remote statfinden.
Das hat Demonstartionen und Absprachen deutlich erschwert.
Anforderungen wurden am Anfang gemacht, das ist auch gut so.
Persönlichere Meetings hätten aber vlt direkteres Feedback ermöglicht, so dass direkter auf Bedürfnisse hätte eingegangen werden können.
Letztlich bin ich selbst am Covid erkrankt.
Genau in der Zeit in der ich vorgenommen hatte, Zeit für das Projekt zu investieren.
Das hat zu Verzögerungen geführt.
Insgesamt trotzdem erreicht.
Aber es könnte besser sein.
Anforderungen waren als Minimum gedacht, mit raum für mehr.

Fazit: Puffer sind nötig.
Es wurde Zeit für Polishing eingeplant aber nicht genug.
Künftig: Puffer explizit als Puffer einbauen und nicht als Zeit in der man erwartet noch mehr machen zu können.
Mehr Zeit für Testing
Mehr Zeit für Polishing

Insgesamt bin ich mit dem Resultat dieser Arbeit sehr zufrieden.
Ich bin sehr zufrieden mit der Systemarchitektur.
Überzeugt, dass diese verwendet werden kann um ein gutes, kommerzielles Produkt zu erstellen.
Ich bin weiter zufrieden mit dem Aufbau des Mobile Clients.
Besonders da es mein erster ist.
Besonders Anbindung umsysteme und integration in UI.
Gleichzeitig hätte ich mir gwünscht weiter zu kommen.
Es wurden gerade die minimalen Anforderungen unmgesetzt, die am Anfang definiert wurden.
Eigentlich hätte ich mehr gewollt.

Unterm Strich: Ein guter Prototyp der als Basis für eine kommerzialisierung eines Cloudbasierten Praxisrufsystems dienen kann.
