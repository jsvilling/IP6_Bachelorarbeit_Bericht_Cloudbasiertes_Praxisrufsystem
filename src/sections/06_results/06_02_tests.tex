\subsection{Tests}

\subsubsection{Benutzertests}

Lorem ipsum

\clearpage

\subsubsection{Testplan migrierte Funktionalität}

Im Rahmen des Projektes IP5 Cloudbasiertes Praxisrufsystem wurde ein finaler Testplan definiert.
Diese Tests wurden zum Abschluss des Projektes durchgeführt, um die Funktionalität des Systems abschliessend zu testen.
Um zu gewährleisten, dass alle Funktionen aus dem alten (Shared Platform) Mobile CLient auch im neuen (Native) Mobile Client zur Verfügung stehen wurden diese Tests zum Abschluss dieses Projekts wiederholt.
Die detaillierte Definition der Testszenarien sind im Anhang E des Projektberichts IP5 Cloudbasiertes Praxisrufsystem beschrieben.\cite{ip5}
Folgendes Protokoll gibt eine Übersicht über die Tests

Folgendes Protokoll zeigt den Stand der letzten Ausführung der Tests am xx.xx.2022:

\begin{table}[h]
    \centering
    \begin{tabular}{|l|p{11cm}|c|c|}
        \hline
        \textbf{Szenario} & \textbf{Beschreibung}                                                                                                                                  & \textbf{Resultat} \\
        \hline
        S01         & Benachrichtigung versenden - Empfänger konfiguriert   & tbd\\
        \hline
        S02         & Benachrichtigung versenden - kein Empfänger & tbd\\
        \hline
        S03         & Benachrichtigung empfangen.  & tbd\\
        \hline
        S04         & Fehler beim Versenden anzeigen.  & tbd\\
        \hline
        S05         & Wiederholen im Fehlerfall bestätigen.  & tbd\\
        \hline
        S06         & Wiederholen im Fehlerfall abbrechen.  & tbd\\
        \hline
        S07         & Audiosignal bei Benachrichtigung.   & tbd\\
        \hline
        S08         & Push Benachrichtigung im Hintergrund.  & tbd\\
        \hline
        S09         & Erinnerungston für nicht Quittierte Benachrichtigungen.   & tbd\\
        \hline
        S10         & Start Mobile Client - nicht angemeldet   & tbd\\
        \hline
        S11         & Start Mobile Client  - angemeldet & tbd\\
        \hline
        S12         & Anmelden mit korrekten Daten.   & tbd\\
        \hline
        S13         & Anmeldung mit ungültigen Daten.   & tbd\\
        \hline
        S14         & Konfiguration Wählen   & tbd\\
        \hline
        S15         & Abmelden.   & tbd\\
        \hline
        S16         & Admin UI - Anmeldung mit korrekten Daten   & tbd\\
        \hline
        S17         & Admin UI - Anmeldung mit ungültigen Daten   & tbd\\
        \hline
        S18         & Admin UI - Konfiguration Verwalten   & tbd\\
        \hline
    \end{tabular}\label{tab:testplan_migration}
\end{table}
\clearpage

\subsubsection{Testplan Sprachsynthese}

\begin{table}[h]
    \centering
    \begin{tabular}{|l|p{11cm}|c|c|}
        \hline
        \textbf{Szenario} & \textbf{Beschreibung}                                                                                                                                  & \textbf{Resultat} \\
        \hline
        S19         & Benachrichtigung vorlesen - Sprachsynthese aktiviert und Benachrichtigung relevant & tbd\\
        \hline
        S20         & Benachrichtigung nicht vorlesen - Sprachsynthese aktiviert und Benachrichtigung nicht relevant & tbd\\
        \hline
        S21         & Lokale Einstellung - Sprachsynthese deaktiviert und Benachrichtigung relevant  & tbd\\
        \hline
        S22         & Lokale Einstellung - Sprachsynthese deaktiviert und Benachrichtigung nicht relevant  & tbd\\
        \hline
        S23         & Benachrichtigung verwalten - Relevanz Sprachsynthese kann im Admin UI aktiviert / deaktiviert werden  & tbd\\
        \hline
        S24         & Benachrichtigung empfangen - Änderung an Typ in Admin UI wird sofort angewendet   & tbd\\
        \hline
    \end{tabular}\label{tab:testplan_sprachsynthese}
\end{table}

\clearpage

\subsubsection{Testplan Gegensprechanlage}
\begin{table}[h]
    \centering
    \begin{tabular}{|l|p{11cm}|c|c|}
        \hline
        \textbf{Szenario} & \textbf{Beschreibung}                                                                                                                                  & \textbf{Resultat} \\
        \hline
        S25         & Gegensprechanlage Buttons nach Anmeldung anzeigen & tbd\\
        \hline
        S26         & Verbindungsaufbau - Gegenüber ist Verfügbar & tbd\\
        \hline
        S27         & Verbindungsaufbau - Gegenüber ist nicht Verfügbar & tbd\\
        \hline
        S28         & Verbindungsaufbau - Gegenüber hat Gegensprechanlage deaktiviert & tbd\\
        \hline
        S29         & Verbindungsaufbau - Benachrichtigungston & tbd\\
        \hline
        S30         & Verbindungsaufbau - Automatische Annahme & tbd\\
        \hline
        S31         & Unterhaltung 1:1 - Unterhaltung in Echtzeit möglich & tbd\\
        \hline
        S32         & Unterhaltung 1:n - Unterhaltung in Echtzeit möglich & tbd\\
        \hline
        S33         & Verbindungsaufbau 1:n & tbd\\
        \hline
        S34         & Inbox - Vergangene Sprachverbindungen & tbd\\
        \hline
        S35         & Inbox - Verpasste Sprachverbindungen & tbd\\
        \hline
        S36         & Inbox - Abgelehnte Unterhaltungen & tbd\\
        \hline
        S37         & Verbindung trennen durch Empfänger & tbd\\
        \hline
        S38         & Verbindung durch Initiator & tbd\\
        \hline
        S39         & Austreten aus Gruppenunterhaltung & tbd\\
        \hline
        S40         & Konfiguration über Admin UI & tbd\\
        \hline
    \end{tabular}\label{tab:testplan_gegensprechanlage}
\end{table}
\clearpage

\subsubsection{Testplan Performance}

\begin{table}[h]
    \centering
    \begin{tabular}{|l|p{11cm}|c|c|}
        \hline
        \textbf{Kriterium} & \textbf{Beschreibung}                                                                                                                                  & \textbf{Resultat} \\
        \hline
        P01         & Zeit bis Benachrichtigung ankommt im Schnitt < 5s & tbd\\
        \hline
        P02         & Vorlesen von Benachrichtigung < 5s nach Benachrichtigungston (ohne Cache) & tbd\\
        \hline
        P03         & Vorlesen von Benachrichtigung < 1s nach Benachrichtigungston (mit Cache) & tbd\\
        \hline
        P04         & Verbindungsaufbau Sprachverbindung <5 s  & tbd\\
        \hline
        P05         & Verzögerung bei Sprachverbindung klein genug für normale Unterhaltungen & tbd\\
        \hline
        P06         & Ressourcenverbrauch der Applikation bleibt über Zeit konstant & tbd\\
        \hline
    \end{tabular}\label{tab:testplan_performance}
\end{table}
\clearpage
