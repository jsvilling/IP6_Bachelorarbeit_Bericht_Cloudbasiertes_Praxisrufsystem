\section{Schluss}

Im Rahmen dieser Projektarbeit wurde Sprachübertragung und Sprachsynthese in ein cloudbasiertes Praxisrufsystem integriert.
Die umgesetzte Lösung basiert auf dem Praxisrufsystem das im Rahmen des Projektes ''Cloudbasiertes Praxisrufsystem'' umgesetzt wurde.\cite{ip5}
Das erweiterte System besteht aus einer Mobilen Applikation, einem Cloudservice und einer Web-Applikation.
Die Mobile Applikation wurde neu als native Applikation für iOS entwickelt.
Sie ersetzt die Shared Platform Applikation, welche im Rahmen des Vorgängerprojektes entwickelt wurde.
Dabei wurden sämtliche Funktionen und Anbindungen an Umsysteme auch in der neuen Applikation implementiert.
Mit diesem Projekt neu konzipiert und umgesetzt wurden die Sprachsynthese für empfangene Benachrichtigung mit ''AWS Polly '' sowie die Integration einer Gegensprechanlage über Peer To Peer Verbindungen.

Das umgesetzte Praxisrufsystem kann zum Austausch von Informationen in einem Praxisumfeld verwendet werden.
Als Endgeräte dienen dabei iOS Tablets.
Über die Mobile Applikation des Systems ist es möglich Sprachverbindungen zu einem oder mehreren anderen Clients aufzubauen.
Eingehende Sprachverbindungen werden automatisch angenommen.
Das System unterstützt weiter das Versenden und Empfangen von Benachrichtigungen.
Dies wird einerseits verwendet, um nicht erreichbare Empfänger über verpasste Sprachverbindungen zu informieren.
Weiter bietet die Applikation Praxismitarbeitenden die Möglichkeit vorkonfigurierte Benachrichtigungen an andere Clients zu versenden.
Der Inhalt von Benachrichtigungen kann dabei beim Empfang automatisch vorgelesen werden.
Sowohl empfangene Benachrichtigungen als auch verpasste und vergangene Anrufe, werden gesammelt und in einer Inbox angezeigt.
Empfangene Benachrichtigungen und verpasste Anrufe müssen von Praxismitarbeitenden quittiert werden.
Sind unquittierte Elemente in der Inbox, ertönt in regelmässigen Abständen ein Erinnerungston.

Das umgesetzte System deckt die wesentlichen Anforderungen eines cloudbasierten Praxisrufsystems ab.
Für eine kommerzielle Nutzung des Systems sind aber zusätzliche Erweiterungen notwendig.
Praxisruf unterstützt in der aktuellen Version die Authentifzierung mittels Json Web Tokens.
Ausstellung der Tokens wird dabei allerdings durch Praxisruf selbst gemacht.
Es wird empfohlen vor der kommerziellen Nutzung einen externen Identity Provider anzubinden und Authentifizierung/Authorisierung nach OpenID Connect umzusetzen.
Weiter ist Praxisruf heute nur beschränkt mandantenfähig.
Praxismitarbeitende haben nur Zugriff auf Konfigurationen, welche dem verwendeten Benutzer zugewiesen sind.
Praxisadministratoren können allerdings immer alle bekannten Konfigurationen über das Admin UI verwalten.
Weiter wird bei der Auswertung der Konfiguration für das Zustellen von Benachrichtigungen und beim der Signalvermittlung für Sprachverbindungen keine Zuordnung an Benutzer oder Mandat geprüft.
Um Praxisruf produktiv bei mehreren Kunden einzusetzen, muss es Mandantenfähigkeit implementiert werden.
Letztlich sind heute einfache Mechanismen für das Wiederholen von Benachrichtigungen und Wiederaufbau von verlorenen Verbindungen implementiert.
Für den produktiven Betrieb können und müssen diese allerdings noch erweitert werden.
Insbesondere der Wiederaufbau von bestehenden Sprachverbindungen bei Verbindungsverlust ist für eine kommerzielle Nutzung unerlässlich.

Die grösste Gefahr für die produktive Nutzung von Praxisruf ist allerdings, das es bis heute nie in grösserem Umfang produktiv eingesetzt wurde.
In der aktuellen Form sollte Praxisruf nicht im grossen Stil produktiv eingesetzt werden.
Es bietet allerdings alle Funktionen, die ein Praxisrufsystem benötigt.
Dementsprechend ist es möglich, das System in einem Pilotbetrieb einzusetzten.
So kann Testfeedback von Benutzern eingeholt werden und es können Performance Metriken gesammelt werden.
Diese können weitere Einblicke darauf geben, welche Teile des Cloudservices separat als skalierbare Microservices deployed werden sollen.
Wird Praxisruf mit den Erkentnissen aus einem Pilotbetrieb ergänzt und die Funktionen Mandantenfähigkeit und OpenId Connect implementiert, kann es kommerziell und produktiv genutzt werden.

Insgesamt bin ich mit den Konzepten und Ergebnissen, die aus dieser Arbeit hervorgegangen sind zufrieden.
Ich bin überzeugt, dass die erarbeiteten Konzepte und das umgesetzte System eine solide Grundlage für ein kommerziell erfolgreiches cloudbasiertes Praxisrufsystem bilden.

\clearpage
