\section{Schluss}

Im Rahmen dieser Projektarbeit wurden Peer-To-Peer Sprachverbindungen und Sprachsynthese in ein cloudbasiertes Praxisrufsystem integriert.
Die umgesetzte Lösung basiert auf dem Praxisrufsystem das im Rahmen des Projektes ''IP5 Cloudbasiertes Praxisrufsystem''~\cite{ip5} umgesetzt wurde.
Das Praxisrufsystem kann zum Austausch von Informationen in einem Praxisumfeld verwendet werden.
Kommunikation ist dabei über Textnachrichten und Sprachverbindungen zwischen mehreren Teilnehmern möglich.

Das erweiterte System besteht aus einer Mobilen Applikation, einer Serverkomponente und einer Webapplikation.
Die Mobile Applikation wurde neu als native Applikation für iOS entwickelt.
Sie ersetzt die Shared Platform Applikation, welche im Rahmen des Vorgängerprojektes entwickelt wurde.
Dabei wurden sämtliche bestehende Funktionen und Anbindungen an Umsysteme in die neue Applikation migriert.
Mit diesem Projekt neu konzipiert und umgesetzt wurden die Sprachsynthese für empfangene Benachrichtigung mit Amazon Polly sowie die Integration einer Gegensprechanlage über Peer-To-Peer Sprachverbindungen.
Die Serverkomponente und Webapplikation wurden erweitert, um diese Funktionen im System verwenden und konfigurieren zu können.

Die mobile Applikation dient als Benutzeroberfläche für das Rufsystem.
In der Applikation können über vorkonfigurierte Buttons Sprachverbindungen aufgebaut von Textnachrichten versendet werden.
Empfangene Benachrichtigungen werden als Push-Benachrichtigungen angezeigt und in eine Inbox eingefügt.
Durch die Anbindung von Amazon Polly, kann der Inhalt von empfangen Textnachrichten automatisch vorgelesen werden.
Eingehende Sprachverbindungen werden automatisch angenommen.
Dadurch kann das Rufsystem als konfigurierbare Gegensprechanlage verwendet werden.
Kann ein Empfänger für Sprachverbindungen nicht erreicht werden, wird er mit einer Textnachricht über die verpasste Unterhaltung informiert.
Sowohl empfangene Benachrichtigungen als auch verpasste und vergangene Anrufe, werden gesammelt und in einer Inbox angezeigt.
Empfangene Benachrichtigungen und verpasste Anrufe müssen von Praxismitarbeitenden quittiert werden.
Sind unquittierte Elemente in der Inbox, ertönt in regelmässigen Abständen ein Erinnerungston.

Das umgesetzte System deckt die wesentlichen Anforderungen eines cloudbasierten Praxisrufsystems ab.
Für eine kommerzielle Nutzung des Systems sind aber zusätzliche Erweiterungen notwendig.
Es wird empfohlen vor der produktiven Nutzung des Systems einen externen Identity Provider anzubinden und Authentifizierung/Authorisierung nach OpenID-Connect umzusetzen.
Weiter ist das umgesetzte Praxisrufsystem heute nur beschränkt mandantenfähig.
Praxismitarbeitende haben nur Zugriff auf Konfigurationen, welche dem verwendeten Benutzer zugewiesen sind.
Praxisadministratoren können allerdings immer alle bekannten Konfigurationen verwalten.
Weiter wird bei der Auswertung der Konfiguration für das Zustellen von Textnachrichten und bei der Signalvermittlung für den Aufbau von Sprachverbindungen keine Zuordnung an Benutzer oder Mandat geprüft.
Um Praxisruf produktiv bei mehreren Kunden einzusetzen, muss eine Trennung von Mandanten ermöglicht werden.
Letztlich sind heute einfache Mechanismen für das Wiederholen von Benachrichtigungen und Wiederaufbau von verlorenen Verbindungen implementiert.
Für den produktiven Betrieb können diese allerdings noch erweitert werden.
Es könnten z.B.\ angezeigt werden, wenn eine Verbindung nicht mehr repariert werden kann und der Benutzer sich neu anmelden muss.

Eine weitere Gefahr für die produktive Nutzung des Systems ist, dass es noch nicht über einen längeren Zeitraum in einem realistischen, produktiven Umfeld getestet wurde.
Es wird empfohlen, das System als Prototypen in einem Pilotbetrieb zu testen.
So kann Testfeedback von Praxismitarbeitenden eingeholt werden und es können Performancemetriken gesammelt werden.
Anhand der Resultate dieser Tests können nötige Erweiterungen und Optimierungen am System identifiziert werden.

Insgesamt bin ich mit den Konzepten und Ergebnissen, die aus dieser Arbeit hervorgegangen sind zufrieden.
Die erarbeiteten Konzepte bilden eine solide Grundlage für die Architektur eines cloudbasierten Praxisrufsystems.
Das umgesetzte Rufsystem erfüllt die Anforderungen eines cloudbasiertes Praxisrufsystem.
Es bildet einen Prototypen, welcher für die Entwicklung eines kommerziell erfolgreichen Praxisrufsystems erweitert werden kann.

\clearpage
