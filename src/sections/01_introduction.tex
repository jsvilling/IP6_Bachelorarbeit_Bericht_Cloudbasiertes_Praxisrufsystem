\section{Einleitung}
"Ärzte und Zahnärzte haben den Anspruch in Ihren Praxen ein Rufsystem einzusetzen.
Dieses Rufsystem ermöglicht, dass der behandelnde Arzt über einen Knopfdruck Hilfe anfordern oder Behandlungsmaterial bestellen kann.
Zusätzlich bieten die meisten Rufsysteme die Möglichkeit eine Gegensprechfunktion zu integrieren.
Ein durchgeführte Marktanalyse hat gezeigt, dass die meisten auf dem Markt kommerziell erhältlichen Rufsysteme auf proprietären Standards beruhen und ein veraltetes Bussystem oder analoge Funktechnologie zur Signalübermittlung einsetzen.
Weiter können diese Systeme nicht in ein TCP/IP-Netzwerk integriert werden und über eine API extern angesteuert werden.

Im Rahmen dieser Arbeit soll ein Cloudbasiertes Praxisrufsystem entwickelt werden.
Pro Behandlungszimmer wird ein Android oder IOS basiertes Tablet installiert.

Auf diese Tablet kann die zu entwickelnde App installiert und betrieben werden.
Die App deckt dabei die folgenden Ziele ab:

\begin{itemize}
    \item Evaluation Frameworks für die Übertragung von Sprachinformationen (1:1 und 1:m)
    \item Erweiterung SW-Architektur für die Übertragung von Sprachdaten
    \item Definitoin und Implementierung Text-to-Speach Funktion
    \item Implementierung Sprachübertragung inklusive Gegensprechfunktion
    \item Durchführung von Funktions- und Performancetests
\end{itemize}

Die Hauptproblemstellung dieser Arbeit ist die sichere und effiziente Übertragung von Sprach- und Textmeldungen zwischen den einzelnen Tablets.
Dabei soll es möglich sein, dass die App einen Unicast, Broadcast und Mutlicast Übertragung der Daten ermöglicht.
Über eine offene Systemarchitektur müssen die Kommunikationsbuttons in der App frei konfiguriert und parametrisiert werden können."\footnote{Ausgangslage, Ziele und Problemstellung im Originaltext der Aufgabenstellung}\cite{aufgabenstellung}

%TODO: Mention IP5 als Basis
%TODO: Gefahren
%TODO: Geplanter Projektablauf
%TODO: Aufbau Bericht

\clearpage
