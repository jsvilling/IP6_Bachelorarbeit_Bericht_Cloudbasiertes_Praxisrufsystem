\section*{Management Summary}

''Ärzte und Zahnärzte haben den Anspruch in Ihren Praxen ein Rufsystem einzusetzen.
Dieses Rufsystem ermöglicht, dass der behandelnde Arzt über einen Knopfdruck Hilfe anfordern oder Behandlungsmaterial bestellen kann.
Heute kommerziell erhältlichen Rufsysteme beruhen meistens auf proprietären Standards und veralteten Technologien.''\cite{aufgabenstellung}
Mit dem Projekt ''IP5 Cloudbasiertes Praxisrufsystem'' wurde ein modernes Rufsystem ''Praxisruf'' mit reduziertem Funktionsumfang umgesetzt.
Die Problemstellung für dieses Projekt besteht darin, das bestehende Praxisrufsystem zu erweitern und fehlende Funktionen zu ergänzen.

Das erweiterte Praxisrufsystem hat zwei Hauptfunktionen.
Es kann als Gegensprechanlage verwendet werden und bietet die Möglichkeit Benachrichtigungen in einer Praxis zu versenden.
Das Rufsystem wird von Endbenutzern über IPads bedient.
Die dazu entwickelte Applikation ermöglicht es, Sprachverbindungen zwischen zwei oder mehr Teilnehmern aufzubauen und Benachrichtigungen zu versenden.
Der Inhalt von empfangenen Benachrichtigungen kann bei entsprechender Konfiguration automatisch vorgelesen werden.
Welche Sprachverbindungen und Benachrichtigungen zur Verfügung stehen, wird von Administratoren über eine Weboberfläche konfiguriert.

Nach Projektabschluss stehen drei Optionen für das weitere Vorgehen offen:

\textbf{1. Produktiver Einsatz}

Mit den Erweiterungen, die in diesem Projekt umgesetzt wurden, erfüllt Praxisruf die Mindestanforderungen an ein modernes Praxisrufsystem.
Dementsprechend könnte das System bereits heute als produktives Rufsystem in einem Praxisumfeld eingesetzt werden.
Praxisruf allerdings noch nicht in einem produktiven Umfeld getestet.
Es muss verifiziert werden, dass das System mit der Last, die in einem produktiven Umfeld entsteht umgehen kann und den Qualitätsansprüchen von Praxismitarbeitenden entspricht.
Weiter hat das System noch Lücken in den Bereichen Benutzerfreundlichkeit und Konfigurierbarkeit.
Aufgrund dieser Einschränkung wird davon abgeraten, Praxisruf produktiv einzusetzen.

\textbf{2. Test in Pilotbetrieb}

Um Praxisruf in einem produktiven Umfeld zu testen, könnte es als Pilotprojekt in einzelnen Praxen eingesetzt werden.
In diesem Pilotprojekt könnten Rückmeldung von Praxismitarbeitenden gesammelt und Performancemetriken gemessen werden.
Aufgrund dieser Ergebnisse kann evaluiert werden, welche Erweiterungen an Praxisruf vorgenommen werden müssen.

\textbf{3. Weiterentwicklung für kommerzielle Nutzung}

Praxisruf ist heute darauf ausgelegt, in genau einer Praxis eingesetzt zu werden.
Um es als kommerziell erhältliches System anbieten zu können muss es um Mandantenfähigkeit erweitert werden.
Es muss möglich sein mehrere Systemkonfigurationen zu verwalten, wobei diese strikt voneinander getrennt bleiben.
Weiter wird empfohlen die Berechtigungsprüfung in Praxisruf zu erweitern und einen externen Identity-Provider anzubinden.

\textbf{Empfehlung}

Es wird empfohlen mit der Option zwei ''Test in Pilotbetrieb'' fortzufahren.
Durch Pilotbetrieb in einem realistischen Umfeld, kann garantiert werden, dass Praxisruf den Qualitätsanspruchen der Anwender enstpricht.
Weiter kann die Verwendung des Systems beobachtet werden und in hinsicht Performance und Benutzerfreundlichkeit wo nötig verbessert werden.
Sollte Praxisruf als kommerzielles System angeboten werden, wird zudem empfohlen die Option drei ''Weiterentwicklung für kommerzielle Nutzung'' zu verfolgen.
Die Erweiterung um Mandantenfähigkeit und Anbindung eines externen Identity-Providers sind für die kommerzielle Nutzung des Systems unerlässlich.

\clearpage
