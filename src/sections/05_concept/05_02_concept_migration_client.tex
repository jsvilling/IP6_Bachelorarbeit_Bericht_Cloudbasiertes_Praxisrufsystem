\subsection{Migration Benachrichtigungen}

Mit IP5 wurde bereits ein Client umgesetzt.
Dieser muss für IP6 migiert werden.
Hier wird beschrieben, wie die bestehenden Anforderungen mit dem nativen client umgesetzt werden können.

\subsubsection*{Benutzeroberfläche}
SwiftUI bietet alles was man braucht.


\subsubsection*{Anbindung Cloudservice}
Anbindung an REST Schnittstellen ist mit SwiftUI natürlich Möglich.
Ein zentraler API Service wird erstellt.
Pro Domain die angesprochen wird, wird eine Extension erstellt.
Der Api Service macht den Rest call, setzt authentication.
Es muss für jeden Api Call ein Callback mitgegeben werden, dass bei completion ausgeführt wird.
Dabei muss dieses Callback den Erfolgs und den Fehlerfall behandeln. \\

Integration in die Benutzeroberfläche funktionert über einen zwischengeschalteten Service (ViewModel?).
Dieses verwendet @ObservableObject um die View über den SwiftUI Lifecycle zu aktualisieren. \\

Sämtliche Calls und Abläufe können mit diesen Mitteln analog zu IP5 umgesetzt werden.


\subsubsection*{Anbindung Firebase}

Die Anbindung von Firebase ist nicht rein mit SwiftUI möglich.
Wir benötigen die Lifecycle Integration zu iOS, die mit den AppDelegates von UIKit möglich ist.
Auch mit SwiftUI können AppDelegates verwendet werden.
Die Anbindung an Firebase Messaging wird dementsprechend mit AppDelegates gelöst.
Die Anbindung erfolgt damit wie in der offiziellen Dokumentation vorgesehen.
Damit die Abhängigkeit zu AppDelegates minimiert ist, sollen innerhalb der AppDelegates nur minimale Logik ausgeführt werden.
Die echte Logik wird an unabhängige Services delegiert.


\subsubsection*{Scheduled Reminder für Inbox}
IOS Development unterstützt scheduled tasks.

\clearpage
