\subsection{Gegensprechanlage}

Mit dem Einbau von synchroner Sprachübertragung wird Praxisruf um die Funktion einer Gegensprechanalge erweitert.
Die gewählte Technologie WebRTC erlaubt es Peer to Peer Sprachverbindungen zwischen Clients aufzubauen.
Dieses Kapitel beschreibt wie das Praxisrufsystem erweitert wird, um eine konfigurierbare Gegensprechanlage mit WebRTC zu integrieren.

\subsubsection{Konfiguration}

Praxisruf wird um die Funktion einer Gegensprechanlage erweitert.
Dabei soll von einem Administrator zentral konfiguriert werden können, zwischen welchen Clients Sprachverbindungen aufgebaut werden können.
Damit dies möglich ist, sind Änderungen an der Configuration Domain des Cloudservice von Praxisruf notwendig.

Praxisruf bietet bereits heute die Möglichkeit Buttons zu konfigurieren, über welche Benachrichtigungen versendet werden können.
Diese Buttons werden mit der Entität NotificationType konfiguriert, welche wiederum einer ClientConfiguration zugeordnet werden können.
Diese ClientConfiguration wird bei der Anmeldung auf dem Mobile Client geladen und verwendet um die nötigen Buttons darzustellen.
Analog dazu wird für den Aufbau von Sprachverbindungen eine Entität CallType erstellt.
Ein CallType beinhaltet den Text, welcher auf dem zugehörigen Button auf Clientseite angezeigt wird und eine Liste von Clients, welche als Ziel der Sprachverbindung verwendet werden.

Die Konfiguration eines Clients wird durch die bestehende Entität ClientConfiguration abgebildet.
Diese Konfiguration wird um eine Liste von CallTypes erweitert.
So kann für jeden Client definiert werden, welche Sprachverbindungen aufgebaut werden können.

\begin{figure}[h]
    \centering
    \begin{minipage}[b]{0.7\textwidth}
        \includegraphics[width=\textwidth]{/home/joshua/FHNW/dev/IP6/IP6_Bachelorarbeit_Bericht_Cloudbasiertes_Praxisrufsystem/src/graphics/diagramms/erd_intercom_v02.drawio}
        \caption{ERD Ausschnitt - Konfiguration Gegensprechanalge}
    \end{minipage}
\end{figure}

Das Admin UI wird mit Ansichten erweitert, um CallTypes zu erstellen, anzeigen, bearbeiten und löschen.
Gleichzeitig wird der Cloudservice um Rest Endpunkte für das Lesen, Erstellen, Aktualisieren und Löschen von CallTypes erweitert.

Die Ansichten für ClientConfigurations im Admin UI werden so erweitert, dass CallTypes darauf angezeigt, hinzugefügt und entferntw erden können.
Die bestehenden Endpunkte für ClientConfiguration werden entsprechend erweitert.

\clearpage

\subsubsection{Signaling Server}

Um Sprachverbindungen zwischen Clients aufzubauen, müssen diese Nachrichten austauschen können.
Die Verbindung wird durch den Sender mit einem Offer initialisiert.
Dieses muss an den Empfänger übermittelt werden.
Dieser antwortet schliesslich mit einer Answer, welche an den Sender übermittelt werden ist.
Bevor die Verbindung aufgebaut ist, kennen die beiden Clients sich gegenseitig noch nicht.
Es braucht deshalb eine Instanz, welche beide Seiten kennt und die Übermittlung der Daten übernehmen kann.
Bei Praxisruf fällt diese Aufgabe dem Cloudservice zu.
Um dies zu ermöglichen wird Cloudservice umeine Schnittstelle erweitert, die es ermöglicht langlebige Verbindungen
zu öffnen und registrieren.
Sobald ein Client sich angemeldet hat, baut er eine Verbindung zum CloudService auf.
Beim Verbindungsaufbau gibt der Client seine technische Identifikation mit.
Der Cloudservice kann damit intern eine der verfügbaren Verbindungen und den dazugehörigen Identifikatoren führen.
Diese Liste kann beim Verbindungsaufbau verwendet werden, um die Offers und Answers an die jeweiligen Clients zu übermitteln.

Dieses Konzept wird im intercom Modul des Cloudservice umgesetzt.
Um die Funktionalität zu ermöglichen werden zwei Komponenten benötigt.
Diese kapseln die Funktionalität die unabhängig von der Technologie zum Verbindungsaufbau notwendig ist.
Um sicherzustellen, dass diese Unabhägigkeit bleibt, werden hier die Interfaces für diese beiden Komponenten spezifiziert.
Es braucht erstens eine Komponente, welche Verbindungen etabliert und Nachrichten über diese Verbindungen senden kann.
Diese Funktionalität wird mit der Komponente ClientConnector umgesetzt.

\lstinputlisting[caption=ClientConnector.java,language=java,label={lst:ClientConnector.java}]{listings/ClientConnector.java}

\clearpage
Weiter braucht es eine Komponente, welche Buch führt über bekannte Verbindungen.
Diese muss Verbindungen anhand einer id registrieren und Verbindungen wieder entfernen können.

\lstinputlisting[caption=ClientConnector.java,language=java,label={lst:ClientConnector.java}]{listings/ConnectionRegistry.java}

\clearpage
\subsubsection{Anmeldung und Registrierung}

Der Ablauf von Anmeldung und Registrierung funktioniert damit grundsätzlich wie bisher.
Er wird aber um eine zusätzliche Registrierung über eine permanente Verbindung zum Intercom Modul ergänzt.

\begin{figure}[h]
    \centering
    \begin{minipage}[b]{0.9\textwidth}
        \includegraphics[width=\textwidth]{/home/joshua/FHNW/dev/IP6/IP6_Bachelorarbeit_Bericht_Cloudbasiertes_Praxisrufsystem/src/graphics/diagramms/Sequence_Registration}
        \caption{Mockup Home}
    \end{minipage}
\end{figure}


\clearpage
\subsubsection{Verbindungsaufbau}

\begin{figure}[h]
    \centering
    \begin{minipage}[b]{0.9\textwidth}
        \includegraphics[width=\textwidth]{graphics/diagramms/Sequence_Intercom_Broking_V02}
        \caption{Ablauf Verbindungsaufbau Gegensprechanalge}
    \end{minipage}
\end{figure}

Wenn der Praxismitarbeitende im Mobile Client einen Button der Gegensprechanlage tippt, soll eine Sprachverbindung zu anderen Clients aufgebaut werden.
Zum Zeitpunkt an dem der Button getippt wird, weiss der Mobile Client nicht, zu welchen Clients diese Verbindung aufgebaut werden soll.
Als erstes muss deshalb beim Cloudservice angefragt werden, welche Clients mit dem betätigten Button angesprochen werden sollen.
Der Cloudservice bietet dazu einen Endpoint an über den die technischen Identifikatoren die in der CallGroup des verwendeten Buttons hinterlegt sind geladen werden.

Der Mobile Client kennt nun die technischen Identifikatoren der Clients, zu denen eine Verbindung aufgebaut werden soll.
Um die Peer To Peer Verbindung zu diesen Clients aufzubauen, müssen nun weitere Nachrichten mit dem Cloudservice ausgetauscht werden.
Alle verfügbaren Clients haben sich bei der Anmeldung mit dem Cloudservice verbunden.
Der Cloudservice führt eine Liste über die verfügbaren Verbindungen und die dazugehörigen technischen Identifikatoren.
Über diese Verbindungen werden nun die Nachrichten ausgetauscht die zum aufbauen der WebRTC Verbindung benötigt werden.
Der Austausch dieser Nachrichten folgt dem Interactive Connection Establishment Protokoll (ICE).
Der Client initialisiert für jede der erhaltenen client Ids einen Rtc Connection.
Dies dient als Endpunkt der Connection auf seiner Seite.
Anschliessend Sendet der Client ein Anfrage an den Cloudservice.
Dieses Anfrage beinhaltet das ICE Offer und die Client Ids des von Ausgangs- und Ziel-Client.
Der Cloudservice findet die Verbindung des Zieles anhand der Client Id und leitet das ICE Offer und Ausgangs Client Id über die registrierte Verbindung an den Empfänger weiter.
Wenn der Empfänger das ICE Offer erhält, initialisiert er auf seiner Seite die Rtc Connection und sendet eine Anfrage mit ICE Antwort und originaler Ausgangs Client Id an den Server zurück.
Dieser leitet die Anfrage dann analog dem ICE Offer an die Verbindung des ausgehenden Clients weiter.
Sobald dieser die Antwort erhält, ist die Rtc Connection bestätigt und die Sprachverbindung ist initialisiert.

\clearpage
\subsubsection{Verpasste und vergangene Anrufe}

Verpasste geben Push Benachrichtigung
Verpasste müssen quittiert werden.
Vergangene erscheinen in Inbox.
Vergangene müssen nicht quittiert werden.

\subsubsection{Unterhaltung}

Mute Button
Audio Off Button
End Call Button


\subsubsection{Verbindungsende}

Button antippen

\clearpage
