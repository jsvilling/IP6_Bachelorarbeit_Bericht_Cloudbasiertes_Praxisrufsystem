\subsection{Bestehende Rufsysteme}

Im Vorfeld dieses Projektes wurde eine Marktanalyse zu kommerziell erhältlichen Rufsystemen durchgeführt.
Die Resultate dieser Analyse sind in der Aufgabenstellung dieses Projektes zusammengefasst:
Die meisten kommerziell erhältlichen Rufsysteme basieren auf proprietären Standards und setzen veraltete Funktechnologien ein.
Diese Systeme sind weder über TCP/IP-Netzwerke integrierbar, noch über externe APIs ansteuerbar~\cite{aufgabenstellung}.
Bestehende Systeme sind deshalb oft aufwändig zu installieren und konfigurieren.
Erweiterung und Veränderungen am System können nur mit grossem Aufwand vorgenommen werden.

Ein cloudbasiertes Praxisrufsystem kann diese Probleme lösen.
Durch die Verwaltung der Konfiguration in Clouddiensten können Schnittstellen zur Integration von Endgeräten geboten werden.
Weiter können Schnittstellen geschaffen werden, welche die Kommunikation anhand dieser Konfiguration vermitteln.
Endgeräte können diese Schnittstellen verwenden, um ihre Konfiguration zu beziehen und mit anderen Endgeräten zu kommunizieren.
Ein solches Rufsystem kann einfach verändert und erweitert werden.
Die Konfiguration kann über die Schnittstellen der Clouddienste verwaltet werden.
Neue Endgeräte können über diese Schnittstellen einfach in das System integriert werden.
Clouddienste können unabhängig von den Endgeräten skaliert werden und über die Schnittstellen in bestehende Netzwerke integriert werden.
