\subsection{Vorarbeit Cloudbasiertes Praxisrufsystem}

Mit dem Vorgängerprojekt ''Cloudbasiertes Praxisrufsystem'' wurde ein cloudbasiertes Rufsystem mit dem Namen Praxisruf entwickelt.
Dieses ermöglicht es, vorkonfigurierte Benachrichtigungen zu versenden und empfangen.
Das System unterstützt allerdings noch keine Sprachverbindungen oder Sprachausgabe.

Praxisruf umfasst eine zentrale Serverkomponente, eine Weboberfläche und eine mobile Applikation für iOS und Android Tablets.
Die mobile Applikation dient als Benutzeroberfläche für Praxismitarbeitende.
Sie ermöglicht es vorkonfigurierte Benachrichtigung zu senden und empfangen.
Die Konfiguration des Systems wird mit der Serverkomponente ''Cloudservice'' verwaltet.
Der Cloudservice bietet dazu eine API an, über welche Konfigurationen erfasst, gelesen und bearbeitet werden können.
Die Weboberfläche ''Admin UI'' bietet eine Benutzeroberfläche, um Konfigurationen zu verwalten.
Das Admin UI verwendet dazu die API des Cloudservice.
Die API des Cloudservice wird weiter von der mobilen Applikation verwendet, um die Konfiguration zu laden und Benachrichtigungen zu versenden.

Das Senden und Empfangen von Benachrichtigungen in Praxisruf wird durch Firebase Cloud Messaging (FCM) ermöglicht.
Sowohl in der mobilen Applikation als auch im Cloudservice ist eine Anbindung an FCM implementiert.
Dabei wird die Anbindung auf der Seite der App ausschliesslich zum Empfangen von Meldungen verwendet.
Das Versenden von Benachrichtigungen wird an den Cloudservice delegiert.
Dazu wird die API des Cloudservice verwendet.
Der Cloudservice wertet die Konfiguration aus, um relevante Empfänger zu identifizieren.
Anschliessend benutzt er die FCM-Anbindung, um die Benachrichtigung an alle relevanten Empfänger zuzustellen.

Sämtliche Infrastruktur für Praxisruf ist mit Amazon Webservices (AWS) eingerichtet.
Der Java-Applikation Cloudservice wird mit einer AWS Elastic Beanstalk Instanz betrieben.
Die Weboberfläche Admin UI wird mit AWS Amplify betrieben~\cite{ip5}.
