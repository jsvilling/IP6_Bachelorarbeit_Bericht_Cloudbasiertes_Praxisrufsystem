\section{Ausgangslage}

Dieses Kapitel beschreibt die Ausgangslage zu Beginn dieser Projektarbeit.
Es wird eine Übersicht zu bestehenden Lösungen und den Problemen, welche diese haben gegeben.
Dabei wird auch beschrieben, wie diese Probleme in einem cloudbasierten Praxisrufsystem gelöst werden.
Weiter wird ein Überblick zu dem im Vorgängerprojekt erarbeiteten cloudbasierten Praxisrufsystem gegeben.

\subsection{Bestehende Rufsysteme}

Im Vorfeld dieses Projektes wurde eine Marktanalyse zu kommerziell erhältlichen Rufsystemen durchgeführt.
Die Resultate dieser Analyse sind in der Aufgabenstellung dieses Projektes zusammengefasst:
Die meisten kommerziell erhältlichen Rufsysteme basieren auf proprietären Standards und setzen veraltete Funktechnologien ein.
Diese Systeme sind weder über TCP/IP-Netzwerke integrierbar, noch über externe APIs ansteuerbar~\cite{aufgabenstellung}.
Bestehende Systeme sind deshalb oft aufwändig zu installieren und konfigurieren.
Erweiterung und Veränderungen am System können nur mit grossem Aufwand vorgenommen werden.

Ein cloudbasiertes Praxisrufsystem kann diese Probleme lösen.
Durch die Verwaltung der Konfiguration in Clouddiensten können Schnittstellen zur Integration von Endgeräten geboten werden.
Weiter können Schnittstellen geschaffen werden, welche die Kommunikation anhand dieser Konfiguration vermitteln.
Endgeräte können diese Schnittstellen verwenden, um ihre Konfiguration zu beziehen und mit anderen Endgeräten zu kommunizieren.
Ein solches Rufsystem kann einfach verändert und erweitert werden.
Die Konfiguration kann über die Schnittstellen der Clouddienste verwaltet werden.
Neue Endgeräte können über diese Schnittstellen einfach in das System integriert werden.
Clouddienste können unabhängig von den Endgeräten skaliert werden und über die Schnittstellen in bestehende Netzwerke integriert werden.

\subsection{Vorarbeit Cloudbasiertes Praxisrufsystem}

Mit dem Vorgängerprojekt ''Cloudbasiertes Praxisrufsystem'' wurde ein cloudbasiertes Rufsystem mit dem Namen Praxisruf entwickelt.
Dieses ermöglicht es, vorkonfigurierte Benachrichtigungen zu versenden und empfangen.
Das System unterstützt allerdings noch keine Sprachverbindungen oder Sprachausgabe.

Praxisruf umfasst eine zentrale Serverkomponente, eine Weboberfläche und eine mobile Applikation für iOS und Android Tablets.
Die mobile Applikation dient als Benutzeroberfläche für Praxismitarbeitende.
Sie ermöglicht es vorkonfigurierte Benachrichtigung zu senden und empfangen.
Die Konfiguration des Systems wird mit der Serverkomponente ''Cloudservice'' verwaltet.
Der Cloudservice bietet dazu eine API an, über welche Konfigurationen erfasst, gelesen und bearbeitet werden können.
Die Weboberfläche ''Admin UI'' bietet eine Benutzeroberfläche, um Konfigurationen zu verwalten.
Das Admin UI verwendet dazu die API des Cloudservice.
Die API des Cloudservice wird weiter von der mobilen Applikation verwendet, um die Konfiguration zu laden und Benachrichtigungen zu versenden.

Das Senden und Empfangen von Benachrichtigungen in Praxisruf wird durch Firebase Cloud Messaging (FCM) ermöglicht.
Sowohl in der mobilen Applikation als auch im Cloudservice ist eine Anbindung an FCM implementiert.
Dabei wird die Anbindung auf der Seite der App ausschliesslich zum Empfangen von Meldungen verwendet.
Das Versenden von Benachrichtigungen wird an den Cloudservice delegiert.
Dazu wird die API des Cloudservice verwendet.
Der Cloudservice wertet die Konfiguration aus, um relevante Empfänger zu identifizieren.
Anschliessend benutzt er die FCM-Anbindung, um die Benachrichtigung an alle relevanten Empfänger zuzustellen.

Sämtliche Infrastruktur für Praxisruf ist mit Amazon Webservices (AWS) eingerichtet.
Der Cloudservice besteht aus einer einzelnen Java-Applikation, welche mit einer AWS Elastic Beanstalk Instanz betrieben.
Das Admin UI ist eine simple Javascript Applikation.
Sie wird mit AWS Amplify betrieben~\cite{ip5}.



%\subsection{Vergleichbare Anwendungen}
%
%Wie in Kapitel 4.1 beschrieben, gibt es keine kommerziell erhältlichen, cloudbasierte Praxisrufsysteme.
%Es gibt allerdings andere Anwendungen, welche ähnliche Technologien einsetzten und ähnliche Probleme lösen.
%Dieses Kapitel beschreibt, wie diese Sprachübertragung und Sprachsynthese lösen und zeigt auf, wieso diese Lösungen sich nicht als Praxisrufsystem eignen.
%
%\subsubsection{Sprachübertragung}
%
%Die zentrale Aufgabenstellung dieses Projektes dreht sich um die Integration von Sprachverbindungen in ein cloudbasiertes Praxisrufsystem.
%Als Endgerät dafür dienen iOS Tablets.
%
%
%
%
%Sprachverbindungen in mobilen Geräten sind weit verbreitet.
%Neben traditionellen
%
%Gibt keine cloudbasierten oder mobile applikation basierten.
%Am nächsten kommen meeting applikationen, telefon oder chat apps wie whatsapp und discord.
%Ganz andere anforderungen.
%Lange gespräche, Chats etc.
%
%WebRTC ist weit verbreitet und wird von fast allem grossen im Hintergrund verwendet.
%''https://bloggeek.me/massive-applications-using-webrtc/''
%
%\subsubsection{Sprachsynthese}
%Ist schwer.
%Macht man nicht selbst.
%Wird ganz viel als Service angeboten.
%Evaluation beschäftigt sich mit welcher anbieter.
%Konzept beschäftigt sich mit effizienter einbindung.
%
%\subsubsection{Mobile Client}
%Native Apps gibts viele.
%Aber keine die Praxisruf können.
%Wichtigste Anforderungen hier sind, dass bestehende Funktionen unterstützt werden.
%und dass Technologien für t2s und p2p technologie unterstützt werden.

\clearpage

