\section{Ausgangslage und Umfeld}

\subsection{Vorarbeit Cloudbasiertes Praxisrufsystem}

Durch das Projekt ''IP5 Cloudbasiertes Praxisrufsystem'' wurde ein erstes cloudbasiertes Praxisrufsystem entwickelt.
Dieses System ermöglicht es vorkonfigurierte Benachrichtigungen zwischen Endgeräten zu versenden.
Sprachverbindungen und Sprachsynthese für Benachrichtigungen werden von diesem System nicht unterstützt.

Das bestehende System umfasst eine zentralen Serverkomponente, eine Weboberfläche und eine mobile Applikation.
Die mobile Applikation dient als Endgeräte für Praxismitarbeitende.
Sie ermöglicht es vorkonfigurierte Benachrichtigung zu senden und empfangen.
Die Konfiguration des Systems wird mit der Serverkomponente ''Cloudservice'' verwaltet.
Der Cloudservice bietet dazu eine REST Api an, über welche Konfigurationen erfasst und gelesen werden können.
Diese Api wird von der Weboberfläche ''Admin UI'' angesprochen, um Konfigurationen zu erstellen und bearbeiten.
Sie wird weiter von der mobilen Applikation verwendet, um die Konfiguration der Applikation zu laden.

Das Senden und Empfangen von Benachrichtigungen im bestehenden System wird durch Firebase Cloud Messaging (Firebase Cloud Messaging) ermöglicht.
Sowohl in der Mobile Client als auch im Cloudservice ist eine Anbindung an FCM implementiert.
Dabei wird die Anbindung auf der Seite des Mobile Clients ausschliesslich zum Empfangen von Meldungen verwendet.
Das Versenden von Benachrichtigungen wird an den Cloudservice delegiert.
Dazu sendet ein Mobile Client eine Anfrage an die REST Api des Cloudservice.
Dieser wertet die Konfiguration aus und benutzt die FCM Anbindung um eine Benachrichtigung an alle relevanten Empfänger zuzustellen.

Sämtliche Infrastruktur für das bestehende Praxisrufsystem ist mit Amazon Webservices (AWS) eingerichtet.
Der Cloudservice besteht aus einer einzelnen Java-Applikation.
Diese wird mit einer AWS Elastic Beanstalk Instanz betrieben.
Das Admin UI ist eine simple Javascript Applikation.
Sie wird mit AWS Amplify betrieben.\cite{ip5}

\subsection{Vergleichbare Systeme}

Was es auf dem Markt schon gibt.
Halt eben nicht viel.

\subsubsection{Sprachübertragung}

Gibt keine cloudbasierten oder mobile applikation basierten.
Am nächsten kommen meeting applikationen, telefon oder chat apps wie whatsapp und discord.
Ganz andere anforderungen.
Lange gespräche, Chats etc.

WebRTC ist weit verbreitet und wird von fast allem grossen im Hintergrund verwendet.
''https://bloggeek.me/massive-applications-using-webrtc/''

\subsubsection{Sprachsynthese}
Ist schwer.
Macht man nicht selbst.
Wird ganz viel als Service angeboten.
Evaluation beschäftigt sich mit welcher anbieter.
Konzept beschäftigt sich mit effizienter einbindung.

\subsubsection{Mobile Client}
Native Apps gibts viele.
Aber keine die Praxisruf können.
Wichtigste Anforderungen hier sind, dass bestehende Funktionen unterstützt werden.
und dass Technologien für t2s und p2p technologie unterstützt werden.

\clearpage

