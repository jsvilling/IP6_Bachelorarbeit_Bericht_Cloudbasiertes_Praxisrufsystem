\section{Ausgangslage und Umfeld}

\subsection{Vorarbeit Cloudbasiertes Praxisrufsystem}

Bestehender Funktionsumfang.
Shared Mobile Client.
Ist nicht so das Wahre.
Neu native Mobile Client.
Integration bestehender Funktionalität.
Integration t2s
Intgegration p2p

\subsection{Vergleichbare Systeme}

Was es auf dem Markt schon gibt.
Halt eben nicht viel.

\subsubsection{Sprachübertragung}

Gibt keine cloudbasierten oder mobile applikation basierten.
Am nächsten kommen meeting applikationen, telefon oder chat apps wie whatsapp und discord.
Ganz andere anforderungen.
Lange gespräche, Chats etc.

WebRTC ist weit verbreitet und wird von fast allem grossen im Hintergrund verwendet.
''https://bloggeek.me/massive-applications-using-webrtc/''

\subsubsection{Sprachsynthese}
Ist schwer.
Macht man nicht selbst.
Wird ganz viel als Service angeboten.
Evaluation beschäftigt sich mit welcher anbieter.
Konzept beschäftigt sich mit effizienter einbindung.

\subsubsection{Mobile Client}
Native Apps gibts viele.
Aber keine die Praxisruf können.
Wichtigste Anforderungen hier sind, dass bestehende Funktionen unterstützt werden.
und dass Technologien für t2s und p2p technologie unterstützt werden.

\clearpage

